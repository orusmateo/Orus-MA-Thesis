\chapter{Discussion of contributions}
In this section, I consider the implications of realizing Semantic Forms and Query Isomorphs to a fuller potential. To recontextualize this discussion, I reintroduce the problem. The climate crisis continues to accelerate faster than the deployment of climate knowledge can keep up, so we find ourselves in a crisis of understanding. We must work toward new research paradigms and frameworks when facing the vast amount, interdependent nature, and exponential growth of information in the work of Sustainability Transitions Knowledge Activation. We must differentiate the various modes of KA to lead with lower complexity and ecological cost when possible, working from knowledge \textbf{surfacing}, to \textbf{synthesis}, to \textbf{translation}, to \textbf{production} only after the others have been used (KSSTP). Knowledge Production is and will continue to be necessary, but it will be limited due to its high complexity and cost. 
\index[terms]{Knowledge Production (KP)}
\index[terms]{Query Isomorphs}

My thesis presents methods for activating large texts in the climate crisis as a means of Knowledge Activation (KA). KA is both subject and method in this thesis. My literature review and geometric analysis were KA to develop models, or forms, from theory. Visuospatial Systematic Combining was KA for developing theory and methodology using form. In short, this thesis engaged in the inter-informing visuospatial epistemological methods for arriving at form through meaning and for arriving at meaning through form. 

\section{C1. Semantic Forms}


\noindent\textbf{In RQ1.1 I asked:}

\textit{What forms of spatial information visualization are there, and how can they inform the composition of network graphs?}

I proposed C1, Semantic Forms, a taxonomy of three-dimensional topic model compositions for Human-in-the-Loop Computational Analysis of Texts and Graphs (HITL CATG), Human Analysis of Texts and Graphs (HATG), or both.
\index[terms]{Semantic Forms}

\vspace{1em}

\subsection{Vedic visuospatial culture}
My survey of symbols provided a pivotal part of my thesis in connecting two-dimensional ideograms to three-dimensional versions of themselves. This interbeing of the Sri Yantra and the Meru Chakra inspired the cone Semantic Form. The etymology of the word yantra as ``machine” \citep[p. 28]{buhnemann_mandalas_2003} is activated beyond its two-dimensionality into three-dimensional Meru Chakra-inspired \citep[p. 31]{buhnemann_mandalas_2003} cone Semantic Form ``Literary Machines” \citep{nelson_literary_1981} for ``knowledge engineering” \citep[p. 8]{wielinga_kads_1992} as a pivotal part my TCA Workspace ``studio laboratory of knowledge design” \citep[p. 197]{drucker_graphesis_2014}.
\index[terms]{TCA Workspace}
\index[terms]{Sri Yantra}
\index[terms]{Meru Chakra}
\index[terms]{interbeing}

While I do not invoke the associations to warfare in the etymology of the word \textit{yantra} \citep[p. 28]{buhnemann_mandalas_2003}, I do propose that this work can facilitate genitive disagreement and anti-oppressive resistance.  

\subsection{Geometry and Topology}

Semantic Forms are presented as geometric forms, but they encompass topological versatility. The ways their form can be used to interpret and reveal semantic relationships are not limited to a particular size of the data set, nor is it limited to a particular ratio of more consilient terms to less consilient terms. Therefore, their shape is a heuristic for the ways they work with semantic relationships, not a prescription. Furthermore, Semantic Forms can be nested within and overlap with each other, further demonstrating the requirement for a multi-mathematical approach.

The computational operationalization of Semantic Forms will benefit from calculating their geometric topology. For example, the calculation of angles between edge vertices (Degree Difference) will inform how “similar or dissimilar neighbouring vertices are with respect to some quantity” \citep[p. 2]{farzam_degree_2020} (Global Assortativity). Homophily, which measures “the tendency to associate with like-minded or otherwise similar people” \citep[p. 2]{farzam_degree_2020}, is already calculated using Degree Difference and Global Assortativity \citep[p. 2]{farzam_degree_2020}.
\index[terms]{Semantic Forms} \index[terms]{Query Isomorphs} 
\index[terms]{topology}

\subsection{Dimensional addition}

Semantic Forms are derived from a Systematic Combination approach to dimensional addition of Semantic Shapes. Semantic Shapes are the two-dimensional units of network graph composition that are combined into Semantic Forms. 
\index[terms]{Systematic Combining (SC)}

Ware and Mitchell report that the move from two-dimensional to three-dimensional node graphs increases “the size of the graph that can be “read”” by about one order of magnitude \citep[p. 10]{ware_visualizing_2008}. Tversky’s investigations on the neuroscience of the visuospatial indicate that brains evolved to handle position in space before they evolved the ability to handle language. In Tversky’s words, “spatial thinking is the foundation of all thought” and “the foundation for spatial thought is also the foundation for conceptual thought” \citep{tversky_barbara_2022}.
\index[people]{Tversky, Barbara}

Considering interfaces for the visuospatial that rely on our sense of sight, the pivotal middle area that includes human visuospatial ability and higher-dimensional graphs is the third dimension (with or without movement) since we can sense no further. LLMs embed many billions of pieces of information \citep[p. 1]{brown_language_2020} as tensors, or `directions’ in high-dimensional vector space \citep{sanderson_attention_2024}. We rely on CATG, and hopefully, more HITL CATG, when it comes to Language Models. 
\index[terms]{Large Language Model (LLM)}

Drucker’s ``visual forms of knowledge production" \citep{drucker_graphesis_2014}, visual reasoning and visual argument  \citep{tversky_barbara_2022}, are also, then, visuospatial forms of Knowledge Activation when developed in tandem with Tversky’s work and KSSTP. 
\index[people]{Tversky, Barbara}


\subsection{Three-dimensional composition}

Indexing forms for their affordances in interpreting and revealing semantic relationships with HATG, HITL CATG, or both is supported by their use in various research areas. Grant et al. propose that the meta-analysis literature review can be illustrated using a funnel plot \citep[p. 94-95]{grant_typology_2009}. Taylor’s cones of plausibility even show the presence of double-cones \citep[p. 14]{taylor_creating_1990} and provide the basis of Bezold and Hancock’s futures cone \citep[p. 73]{bezold_overview_1993}. Specifically within the area of climate, Kjøde’s \textit{Entanglement of Systemic Design and Sustainability Transitions uses Systematic Combining} uses Systematic Combinations composed with the cone \citep[p. 123]{kjode_entanglement_2024}, sphere  \citep[p. 125]{kjode_entanglement_2024}, and cylinder  \citep[p. 144]{kjode_entanglement_2024}. Lenat et al.’s illustration of upper ontology in relation to middle ontology in \autoref{fig:13} \citep[p. 7]{lenat_harnessing_2010}, which relates to various domain models, depicts a mountain range of meaning. I propose that their cone-like composition, individually and as a group, is remarkable because it captures the form of Adler’s “syntopical reading” \citep[p. xi]{adler_great_1952-2}, consilience \citep{hepburn_scientific_2021,wilson_consilience_1999}, and Peirce’s abduction \citetext{\citealp[p. 106]{peirce_pragmatism_1960}; \citealp[p. 20]{sowa_challenge_2004}}. 
\index[terms]{Systemic Design} \index[terms]{abduction}
\index[people]{Peirce, Charles Sanders}
\index[people]{Adler, Mortimer J.}
\index[people]{Kjøde, Svein Gunnar}
\index[terms]{Systematic Combining (SC)}
\index[people]{Bezold, Clement}
\index[people]{Wilson, Edward O.}
\index[terms]{syntopical reading}
\index[terms]{consilience}


In my research, I have not found an explicit index of three-dimensional forms used to interpret and reveal semantic relationships. I aim for my Semantic Forms to be applied in (a) computational approaches like global topological synchronization (GTS) \citep{bianconi_topology_2024,wang_global_2024} using filtration \citep{giusti_twos_2016}, bundling \citep[p. 1]{holten_forcedirected_2009}, Iterative Embedding and ReWeighting (IERW) \citep[p. 2]{kovacs_iterative_2024}, and my magnetic approach to node grouping; (b) manual node placement in Systemic Design approaches like Systems Oriented Design \citep{sevaldson_designing_2022}, gigamapping \citep{sevaldson_giga-mapping_2011}, Systematic Combining \citep[p. 556]{dubois_systematic_2002} \citep[p. 46]{kjode_entanglement_2024}, Boundary Critique \citep{midgley_theory_1998}; and, (c) in some combination of both where HATG can us used independently or in tandem with HITL CATG.
\index[terms]{Systemic Design}
\index[people]{Sevaldson, Birger}
\index[people]{Kjøde, Svein Gunnar}
\index[terms]{Systematic Combining (SC)}
\index[people]{Gadde, Lars-Erik}
\index[people]{Dubois, Anna}
\index[terms]{gigamapping}
\index[terms]{Semantic Forms}
\index[terms]{Systems Oriented Design (SOD)}
\index[terms]{Boundary Critique}


\section{C2. Query Isomorphs}

\noindent\textbf{In RQ1.2 I asked:}

\textit{How can Computational Analysis of Texts and Graphs (CATG) identify isomorphic semantic structures within large network graphs?}

I proposed C2, Query Isomorphs as a means of Topological Capta Analysis (TCA) in HITL CATG using small graph chunks. To relate Query Isomorphs to C1. Semantic Forms, Semantic Forms necessarily include Query Isomorphs, but Query Isomorphs can be part of any network graph regardless of whether or not that graph is or uses a Semantic Form.
\index[terms]{Query Isomorphs}
\index[terms]{Semantic Forms}


\subsection{Topological network query graphlets}

Query Isomorphs are necessarily topological in that they interpret and reveal semantic relationships across reconfigurable nested hierarchies through the scalable and variable distances of the connections between individual nodes. This section draws on my literature review gap analysis to explain the need to use visible and high-dimensional graphs in Query Isomorphs.

As part of my literature review, I searched for precedents of database query using graphlets or ``network subgraphs” with ``a small number of nodes” 
\citep[p. 5-6]{przulj_modeling_2004} in various permutations \citep[p. 3]{sarajlic_graphlet-based_2016}. On this search, I found the VivoMind Analogy Engine (VAE). The VAE uses Sowa’s conceptual and Query Graphs to match labels, subgraphs and graph transformations \citep[p. 22]{sowa_analogical_2003}, revealing information from larger networks \citep[p. 313]{sowa_conceptual_1984}. 

Query Isomorphs would need to interpret and reveal meaning in low-dimensional formats, and as simplicial complexes, ``higher-order networks that encode higher-order topology and dynamics of complex systems” \citep[p. 1]{wang_global_2024}. Tracking a Query Isomorph across network graphs, vector graphs, and dimensions would require topological methods like TDA \citep[p. 1]{aktas_persistence_2019}, and TCA by extension. While topological graph analysis methods can facilitate managing high information complexity, they remain detached from the embodied interface experience. Query Isomorphs cannot be completely detached from the geometry necessary for front-facing interface; so tracking Query Isomorph edge angles would require geometric topology approaches like Degree Difference \citep{farzam_degree_2020}. 

As a note on emerging AI technologies, vector graphs have the potential to be a productive middle ground for topological application of Query Isomorphs as a form of semantic isomorphological query. The TerminusDB platform, for example, uses vector search to enable ``efficient and accurate similarity-based querying” and “clustering algorithms to gain valuable insights from massive datasets,” modelling data ``in a more semantically meaningful manner, enabling a deeper understanding of the connections between different entities” \citep{terminusdb_enterprise_2023}. 

There is a broad horizon for applying Query Isomorphs, but it will require a multi-mathematical approach that incorporates topology and geometry. As a result of my literature review gap analysis, I propose that while Query Isomorphs are necessarily topological by working across hierarchy, scale, and dimension, Query Isomorphs must use both visible geometric interface and calculation methods and `invisible' high-dimensional graphs.

\index[people]{Sowa, John F.}
\index[terms]{analogy} 
\index[terms]{simplicial complex} 
\index[terms]{topology}
\index[terms]{Query Isomorphs}
\index[terms]{Topological Data Analysis (TDA)}
\index[terms]{Topological Capta Analysis (TCA)}



\subsection{Weighting}

A common problem in weighting is the loss of information through filtration used to assess hierarchical structure in weighted networks. Giusti et al. that in ``In some situations, like measurements of correlation or coherence of activity, the resulting network has edges between every pair of nodes and it is common to \textit{threshold} the network to obtain some sparser, unweighted network whose edges", revealing connections that are important to the researcher \citep[p. 11]{giusti_twos_2016}. By leveraging persistent homology \citep[p. 11]{giusti_twos_2016} in TDA, and TCA by extension, the multi-scale topological features of data can be analyzed using Semantic Forms while preserving critical information that may be lost through conventional thresholding methods. 

%Furthermore, since Query Isomorphs resemble chemical molecules in various ways, point weighting can be applied to capture and reveal the semantic relationships of ideas by representing these molecules- or \textit{datacules} -as ``a union of balls in Euclidean space” \citep[p. 14]{otter_roadmap_2017}. 

\index[terms]{Query Isomorphs}
\index[terms]{Topological Data Analysis (TDA)}
\index[terms]{Topological Capta Analysis (TCA)}

\subsection{Interdisciplinarity powered by HITL CATG}

By employing Query Isomorphs in HITL CATG KA, researchers can query for isomorphic network graphs inside larger text graphs, facilitating detection and surfacing of analogous knowledge structures across different disciplines' datasets and captasets. 
\index[terms]{Query Isomorphs}
\index[terms]{capta}
\index[terms]{captasets}

In conclusion, Query Isomorphs could offer a powerful approach for operationalizing the dimensional versatility of TCA in HITL CATG for interdisciplinary KA by finding isomorphic semantic structures within large network graphs.
\section{C3. Ontological Semantic Network Summaries (OSNS)}
\noindent\textbf{In RQ1.3 I asked:}

\textit{What methods can reveal the relationships between fundamental ideas in texts, graphs, or Large Language Models (LLMs)?}

I proposed C3, Ontological Semantic Network Summaries (OSNS) to reveal ontological relationships between ideas in a given body of research using HITL CATG, HATG, or both.
\index[terms]{Ontological Semantic Network Summaries (OSNS)}

\vspace{1em}
\index[terms]{Large Language Model (LLM)}

OSNS is a method to interpret and reveal ontological relationships using semantic networks in HATG, HITL CATG, or both. OSNS, like any other network graph form, can be integrated with Semantic Form and Query Isomorph approaches for KA. 
\index[terms]{Ontological Semantic Network Summaries (OSNS)}
\index[terms]{Query Isomorphs}

Many refer to philosophers like Aristotle and Porphyry as guides for how we approach ontology or the ``systematic account of Existence” \citep[p. 1]{gruber_toward_1995} Ontology is fundamental in computer science, in which “what “exists” is that which can be represented” \citep[p. 1]{gruber_toward_1995}. As a definition that engages both philosophy and computer science, ontology can be defined as “an explicit specification of a conceptualization” \citep[p. 1]{gruber_toward_1995}.
\index[people]{Aristotle}
\index[people]{Porphyry}

Naming and relating what we consider to exist is elusive yet fundamental to this thesis's various modes of reasoning, like Adler’s “syntopical reading” \citep[p. xi]{adler_great_1952-2}, consilience \citep{hepburn_scientific_2021,wilson_consilience_1999}, and Peirce’s abduction \citetext{\citealp[p. 106]{peirce_pragmatism_1960}; \citealp[p. 20]{sowa_challenge_2004}}. Using OSNS, researchers can examine syntopical, consilient, or abductive terms to each other in an overview of how key terms relate, including the inheritance of key properties like in the Tree of Porphyry. In this way, OSNS can serve to categorize and compare large texts, groups of texts, or LMs.

\index[people]{Peirce, Charles Sanders}
\index[people]{Adler, Mortimer J.}
\index[people]{Sowa, John F.}
\index[people]{Wilson, Edward O.}
\index[terms]{abduction}
\index[terms]{syntopical reading}
\index[terms]{consilience}
\index[terms]{Ontological Semantic Network Summaries (OSNS)}
\index[terms]{Tree of Porphyry}


\subsection{AI}

OSNS could be used to bring people further `into the loop' during language model training, and to compare language models after they are trained. OSNS would allow researchers to assess the depth and breadth of an LM’s proficiencies, gaps, and biases in a given language domain. 
\index[terms]{Large Language Model (LLM)}
\index[terms]{Language Model (LM)}
\index[terms]{Ontological Semantic Network Summaries (OSNS)}

In short, I propose OSNS as a tool to interpret and reveal relationships between fundamental ideas in texts, graphs, and LMs to navigate constellations of information more effectively in KA. 
\index[terms]{Ontological Semantic Network Summaries (OSNS)}


\section{C4. Symbol-setting}

\noindent\textbf{In RQ1.4 I asked:}

\textit{Given the semantic versatility of symbols, how can a practice of collaborative symbol-making support Knowledge Production?}
\index[terms]{Knowledge Production (KP)}

I proposed C4, Symbol-setting, a method for expanding the semiotic range of Knowledge Production using symbol co-creation in HITL CATG, HATG, or both.
\index[terms]{Knowledge Production (KP)}
\index[terms]{Symbol-setting}

\vspace{1em}
\index[terms]{Large Language Model (LLM)}

As I have discussed, Knowledge Activation happens in multiple modes, with and without written words. Visuospatial forms of Knowledge Activation can use the wide range of meaning-making methods available to us to interpret and reveal meaning which written words may not be able to capture effectively.

I expand Pangaro’s ``Conversation to Agree on Goals” \citep[p. 185]{pangaro_design_2011} and the Equity-Centred Community Design framework’s ``language setting” \citep[p. 8-11]{creative_reaction_lab_equity-centered_2018} with my proposal to use the wider variety of tools available in visuospatial epistemology, or Symbol-setting. Similar to Sevaldson’s perspective on Systems Oriented Design, I propose that Symbol-setting is an open-source process design ``methodology without a method” that does not prescribe specific techniques \citep[p. 30]{sevaldson_designing_2022}.
\index[people]{Sevaldson, Birger}
\index[terms]{Symbol-setting}
\index[terms]{Systems Oriented Design (SOD)}
\index[people]{Pangaro, Paul}
\index[terms]{visuospatial epistemology}
\index[terms]{Symbol-setting}

Nonetheless, I do offer a starting point and suggest that practitioners inform themselves of symbol composition \citep{liungman_thought_1995}, information visualization composition \citep{vital_how_2018,ribecca_data_2017}, isomorphology \citep{anderson_drawing_2018}, Systems Oriented Design \citep[p. 343]{sevaldson_designing_2022}, gigamapping \citetext{\citealp{sevaldson_giga-mapping_2011}; \citealp[p. 26]{sevaldson_designing_2022}}, synthesis mapping \citetext{\citealp{jones_synthesis_2016}; \citealp[p. 129]{jones_synthesis_2017}}, Systematic Combining \citetext{\citealp[p. 554]{dubois_systematic_2002}; 
\citealp{ kjode_entanglement_2024}}, Boundary Critique \citep{midgley_theory_1998}, Semantic Forms, Query Isomorphs, OSNS, and the various contributions of my thesis. 
\index[people]{Sevaldson, Birger}
\index[people]{Kjøde, Svein Gunnar}
\index[terms]{Systematic Combining (SC)}
\index[terms]{Boundary Critique}
\index[people]{Midgley, Gerald}
\index[people]{Liungman, Carl G.}
\index[people]{Vital, Anna}
\index[terms]{gigamapping}
\index[people]{Ribecca, Severino}
\index[terms]{morphology}
\index[terms]{isomorphology}
\index[terms]{Ontological Semantic Network Summaries (OSNS)}
\index[terms]{Systems Oriented Design (SOD)}

Regarding the larger practices of three-dimensional analysis of words, Liungman’s \textit{Thought Signs} \citep{liungman_thought_1995} would make an excellent case study for the virtual visuospatial analysis of symbols. \textit{Thought Signs} is particularly ready for such an analysis because of its interrelated classification criteria. Another starting point for such an analysis is the MNIST handwritten digits database \citep{lecun_mnist_2012} and its use in the Embedding Projector in TensorFlow \citep{smilkov_embedding_2016}. Plotting the relationships of symbols into three-dimensional space informed by their composition would allow a researcher to discover new semantic relationships and categorizations within and between symbols.  
\index[people]{Liungman, Carl G.}

In short, Symbol-setting serves as a method for the semiotic expansion of the linguistics of Knowledge Activation and co-production. Symbol-setting can foster interdisciplinary understanding while emphasizing inclusion and community.
\index[terms]{Symbol-setting}



\section{C5. Terroir of Text and Graphs (TTG) }

\noindent\textbf{In RQ1.5 I asked:}

\textit{How can CATG reveal relationships between place and text?}

I proposed C5, Terroir of Text and Graphs (TTG), a method of HITL CATG that uses TCA to interpret and reveal semantic relationships between (a) texts and graphs, and (b) the features and systems of ecological place.

\vspace{1em}

 Beyond the shift to Small Language Models which use fewer resources to function, Language Models can be improved by being informed of ecological dynamics of the place where they are being used. To foster more sustainable ecological development that works more deeply across many ways of knowing, AI-assisted KSSTP of TTG must honour and be informed by Indigenous wisdom and its keepers whenever possible. 
 \index[terms]{Large Language Model (LLM)} 
\index[terms]{Small Language Model (SLM)} 
\index[terms]{Language Model (LM)} 

TTG aims to support experts in ecologically sustainable ways of living whose insights are vital to Sustainability Transitions and climate resilience. Centering Indigenous wisdom, TTG can be used for Knowledge Activation (Knowledge Surfacing, Synthesis, Translation, and Production) of resource use. Indigenous researchers from the Abundant Intelligences research network have enthusiastically received my research on computational methods for examining the relationship between place and ancestral wisdom.
\index[terms]{Indigeneity}
\index[terms]{Terroir of Text and Graphs (TTG)}
\index[terms]{climate resilience}

TTG is compatible with Semantic Forms, Query Isomorphs, and OSNS as graph-based TCA. Furthermore, as a form of Symbol-setting, TTG interprets and reveals the symbolic and material relationships that inform how multiple communities can come to understand their shared and related environments. In this way, TTG reveals the connection of collective knowledge systems to place to symbol, story, and material culture. HITL CATG, in the mode of TTG, can provide insight into the tangible and intangible elements of culture interpreted as words and diagrams.
\index[terms]{Terroir of Text and Graphs (TTG)}
\index[terms]{Symbol-setting}
\index[terms]{Ontological Semantic Network Summaries (OSNS)}
\index[terms]{Query Isomorphs}





\section{C6. TCA Researcher Grouping}
\index[terms]{Indigeneity}
\index[terms]{Terroir of Text and Graphs (TTG)}


\noindent\textbf{In RQ1.6 I asked:}

\textit{What strategies can improve university knowledge management to accelerate research?}

I proposed C6, TCA Researcher Grouping, a proposal to use TCA for grouping research collaborators more effectively using HITL CATG, HATG, or both.
\index[terms]{TCA Researcher Grouping}

\vspace{1em}

Among the various ways researchers and institutions can help or hinder Sustainability Transitions Knowledge Activation is how we get to know each other and collaborate. Current bibliographic platforms seem to be limited to computational classification using metadata categories only, which limits their ability to group authors and works together based on research content. In my tests, I found that even the platforms capable of revealing terms shared by researchers, like the AI-assisted InfraNodus, were not equipped to analyze large numbers of authors and their respective bodies of work.

TCA of research databases has the potential to reveal commonalities between authors that can catalyze new types of research, simultaneously increasing the impact of Knowledge Activation and the quality of the researcher experience. The versatile methods I propose in this thesis work to connect research across conventionally siloed disciplines by supporting the people who make Knowledge Activation happen. 

Current PKM tools like Obsidian are effective for managing an individual's information. Still, my testing has shown that they struggle to scale to the demands of larger datasets and complex repositories. TCA Researcher Grouping would build on the hyperlinked format of PKM but would necessarily have to operate with the versatility of scale and graph dimension, hence my urgency in proposing the application of a Topological Capta Analysis.
\index[terms]{TCA Researcher Grouping}
\index[terms]{Topological Capta Analysis (TCA)}


In the context of the climate crisis, it is fundamental that we develop information technology that involves all the various aspects of KSSTP (KA) to support our researchers and better leverage research resources. TCA Researcher Grouping seeks to build on the capabilities of existing PKM tools like Obsidian and topic modelling tools like InfraNodus to improve upon them. Developing knowledge management tools for institutions like TCA Researcher Grouping is a critical step towards increasing the effectiveness of interdisciplinary Sustainability Transitions Knowledge Activation.
\section{C7. TCA Workspace}
\noindent\textbf{In RQ1.7 I asked:}
\index[terms]{TCA Researcher Grouping}

\textit{What sort of knowledge work software would I want to build to incorporate my various findings and use spatial information visualization to identify isomorphic semantic structures, reveal the relationships between fundamental ideas, build on collaborative symbol-making, reveal relationships between place and text, and improve university knowledge management?}

I proposed C7, TCA Workspace, a proposal for a collaborative HITL CATG + HATG platform to:

\begin{enumerate}[label=\textbf{(\alph*)}, leftmargin=2em]
    \item House all my thesis contributions (\textit{Semantic Forms}, \textit{Query Isomorphs}, \textit{OSNS}, \textit{Symbol-setting}, \textit{TTG}, and \textit{TCA Researcher Grouping}).
    \item Facilitate their combined use with Systemic Design methods for visuospatial reasoning encountered through my literature review, such as gigamapping \citetext{\citealp{sevaldson_giga-mapping_2011}; \citealp[p. 26]{sevaldson_designing_2022}} and Systematic Combining \citetext{\citealp[p. 554]{dubois_systematic_2002}; \citealp{kjode_entanglement_2024}}.
\end{enumerate}
\index[terms]{Systemic Design}
\index[people]{Sevaldson, Birger}
\index[terms]{gigamapping}
\index[people]{Gadde, Lars-Erik}
\index[people]{Dubois, Anna}
\index[terms]{TCA Researcher Grouping}
\index[terms]{Symbol-setting}


TCA Workspace is my proposed starting point for a shared effort between Systemic Design methods and topic modelling by using Semantic Forms, Query Isomorphs, and Topological Capta Analysis. 
\index[terms]{Systemic Design}
\index[terms]{Query Isomorphs}
\index[terms]{Semantic Forms}

\subsection{Design and philosophy}

I propose TCA Workspace as a humanistic and post-structuralist interface, emphasizing the interpretative nature of language, information, and knowledge in KA. Inspired by thinkers such as Eagleton, Barthes, Drucker, and Thích Nhất Hạnh, TCA Workspace works with text and graphs by multidimensional means to reveal and interpret meaning. 
\index[people]{Drucker, Johanna}
\index[people]{Barthes, Roland}
\index[people]{Eagleton, Terry}
\index[people]{Thích Nhất Hạnh}

TCA Workspace works within an ontological ethos that aims to engineer and implement a ``diagrammatic and constellationary rhetoric” within an ``infinitely extensible field” of new legibility conventions \citep[p. 197]{drucker_graphesis_2014}. It strives to move beyond disciplines that are ``antithetical to interpretation”, focusing instead on revealing the ``constructedness of knowledge” \citep[p. 178] {drucker_graphesis_2014}.
\index[people]{Drucker, Johanna}


TCA Workspace seeks to increase human agency over climate complexity by developing cybernetic methods that better leverage Tversky’s neuroscience of the visuospatial \citep{tversky_barbara_2022} and Drucker’s interpretation of humanistic information interfaces. By using ``techniques of semantic web, topic maps, network diagrams, and other computational means”, TCA Workspace seeks to `spatialize’ arguments and ``relations among units of thought” for reconfiguration of their ``constellationary form” \citep[p. 158] {drucker_graphesis_2014}.
\index[people]{Drucker, Johanna}
\index[people]{Tversky, Barbara}


\subsection{TCA Workspace as a tool for addressing climate complexity}
Sustainability Transition is among the most significant challenges of our time, if not the greatest. Any work in any field is contextualized by or oriented towards improving or worsening it. Furthermore, a choice not to address ST is a choice in itself. To echo Finkelstein and Gran Fury's warning against complacency, ``Silence = Death” \citep{finkelstein_after_2018}. 

The rapidly growing amount of information in the many disparate fields of ST \citep{grin_transitions_2011} makes developing research platforms like TCA Workspace more urgent to manage information overload. I aim for TCA Workspace to function as a ``studio laboratory of knowledge design” \citep[p. 197]{drucker_graphesis_2014} that advances the collective understanding necessary for mitigating the climate crisis using the various modes of KA.
\index[people]{Drucker, Johanna}



\subsection{Interbeing of fluid ontologies}
By proposing the development of TCA Workspace I aim to develop methods that further embrace ``fluid ontologies" and diverse classifications, affording a new and richer way of representing knowledge structures that reveal rather than conceal their ``constructedness" \citep[p. 178-179]{drucker_graphesis_2014}. TCA Workspace de-emphasizes monolithic definition to visuospatialize how ideas exhibit the interpenetrating unity-and-diversity of interbeing \citep[p. 80]{nhat_hanh_world_2008}. 
\index[people]{Thích Nhất Hạnh}
\index[terms]{interbeing}
\index[people]{Drucker, Johanna}


\subsection{Humanist post-structuralist design}
Eagleton’s description of text can be read as a vivid characterization of Query Isomorphs, Semantic Forms, and TCA Workspace: ``The `writable' text, usually a modernist one, has no determinate meaning, no settled signifieds, but is plural and diffuse, an inexhaustible tissue or galaxy of signifiers, a seamless weave of codes and fragments of codes, through which the critic may cut his own errant path” \citep[p. 119]{eagleton_literary_2006}. Similarly, Barthes addresses decentralization and dimensionality: “We know now that a text is not a line of words releasing a single `theological' meaning (the `message' of the AuthorGod) but a multi-dimensional space in which a variety of writings, none of them original, blend and clash. The text is a tissue of quotations drawn from the innumerable centres of culture” \citep[p. 146]{barthes_image_1977}. 
\index[people]{Thích Nhất Hạnh}
\index[people]{Barthes, Roland}
\index[terms]{interbeing}
\index[people]{Eagleton, Terry}
\index[terms]{Query Isomorphs}
\index[terms]{Semantic Forms}

As a critical design for interpretative and humanistic interfaces, TCA Workspace facilitates interpretation by embracing ``inconsistency among types of knowledge representation, classification, fluid ontologies, and navigation” \citep[p. 178]{drucker_graphesis_2014}. TCA Workspace represents an adaptive transformation of scholarly tools that exemplify a more dynamic, relational, and interpretative approach to humanistic information design.
\index[terms]{TCA Workspace}
\index[people]{Drucker, Johanna}

To summarize TCA Workspace by turning towards language more explicitly about interface, I recall when Johanna Drucker mused the following: ``Are we merely part of an emerging constellation of potentialities for realization of aspects of knowledge design and interpretative acts that are closer to our once-sensible reading of natural and cultural landscapes? Perhaps we are reawakening habits of associative and spatialized knowledge we once read and through which we knew ourselves. We may yet awaken the cognitive potential of our interpretative condition of being, as constructs that express themselves in forms, contingently, only to be remade again, across the distributed condition of knowing” \citep[p. 192]{drucker_graphesis_2014}. 
\index[people]{Drucker, Johanna}
\index[terms]{TCA Workspace}


\subsection{The future of TCA Workspace}

I aim for TCA Workspace to iteratively improve Visuospatial Knowledge Activation research technology, in concert with other emerging digital scholarship methods, to develop new ways of engaging with complexity. 
\index[terms]{TCA Workspace}

Saint-Martin and Drucker anticipated a future in which researchers contribute to a new language for interpreting graphical relationships in digital spaces \citetext{\citealp[p. 54]{drucker_graphesis_2014}; \citealp[p. 225]{saint-martin_semiotics_1990}}. I propose the development of TCA Workspace to build that future, with the increasing urgency to defend the ecologically vulnerable.
\index[terms]{Visuospatial Knowledge Activation (VKA)} 
\index[terms]{TCA Workspace}
\index[people]{Drucker, Johanna}
\index[people]{Saint-Martin, Fernande}

Drucker can be read as capturing the epistemological impetus of TCA Workspace when she urged: ``We have to find graphical conventions to show uncertainty and ambiguity in digital models, not just because these are conditions of knowledge production in our disciplines, but because the very model of knowledge itself that gets embodied in the process has values whose cultural authority matters very much” \citep[p. 190-191]{drucker_graphesis_2014}. 
\index[terms]{TCA Workspace}
\index[people]{Drucker, Johanna}
\index[terms]{Knowledge Production (KP)}

I propose that Saint-Martin’s ``semantic system of topological semiotics” \citep[p. 225]{saint-martin_semiotics_1990} and Drucker’s ``constellationary field” \citep[p. 196]{drucker_graphesis_2014} capture the `space’ in TCA Workspace. TCA Workspace would facilitate a multi-mathematical framework for the Computational Analysis of Texts and Graphs (CATG). In it, Semantic Form ``thought forms” \citep[p. 196]{drucker_graphesis_2014} would be interpreted and revealed with Query Isomorphs and TCA. This approach would allow us to study the ways thought and form co-inform each other as ``content” and ``configuration of knowledge” \citep[p. 196]{drucker_graphesis_2014}. The ``organizing orders of graphical expression that take on their own legibility” \citep[p. 196]{drucker_graphesis_2014} would be visuospatialized as the semantic relationships revealed with HATG and HITL CATG. TCA Workspace would integrate semantic topological semiotics and constellationary fields to analyze texts and graphs and more efficiently reveal co-informing interdisciplinary knowledge structures.
\index[terms]{TCA Workspace}
\index[people]{Drucker, Johanna}
\index[people]{Saint-Martin, Fernande}
\index[terms]{Query Isomorphs}


\subsection{Living networks of decentralized knowing}

Current tools like keyword searches and statistical digital scholarship methods are limited in their effectiveness for navigating hyperspecialized jargon-laden texts and graphs across diverse disciplines. They often fail to reveal latent, transdisciplinary, and interdisciplinary insights. 

With TCA Workspace, I aim to bridge innovation paths across disciplines, enabling specialists to identify novel approaches latent in other disciplines which they might otherwise overlook. We may already possess life-saving Sustainability Transition insight buried in our embarrassment of informational riches, concealed by context, form, and perhaps intention. My aim is for researchers to use TCA Workspace as a cybernetic mycelium that reconstitutes the matter left behind after the ``death” of an idea to activate words and graphs ``across the distributed condition of knowing” \citep[p. 192]{drucker_graphesis_2014}. 
\index[terms]{cybernetic rhizome}	
\index[terms]{rhizome} 

We must examine the ``detritus” of Knowledge Production to solve complex challenges and create insights. TCA Workspace can catalyze new forms of knowledge resurfacing, synthesis, translation, and production (as necessary) to reactivate the dormant ideas in our web of semantic interpretation into living networks of decentralized STKA.
\index[terms]{Knowledge Production (KP)}


\subsection{Addressing information overload through interdisciplinary collaboration}

In this age of massive data, we need new tools to facilitate functional interdisciplinary collaborations. We need KA interfaces to unify knowledge work across multiple ways of knowing. By developing a shared terminology across disciplines with methods like abductive and syntopically consilient language-setting, TCA Workspace would activate knowledge across siloes.

Furthermore, using AI has the potential to reduce the cost of Knowledge Production by Surfacing, Synthesizing, and Translating knowledge that is already contained in knowledge artefacts beyond text-centric modalities, including architecture, agriculture, petroglyphs, weaving, song, and other forms of wisdom encoding.
\index[terms]{gigamapping}
\index[terms]{Knowledge Production (KP)}
\index[terms]{Symbol-setting}
\index[terms]{Semantic Forms}

\subsection{Humanistic design elements of TCA Workspace}

TCA Workspace and other forms of computational text analysis exist in the legacy of Vannevar Bush’s memex \citep{bush_as_1945}, Ted Nelson’s hypertext \citep{nelson_literary_1981}, and Tim Berners-Lee’s World-Wide Web \citep{berners-lee_world-wide_1992}, among many others. By proposing TCA Workspace I seek to build on these foundational technologies as part of what Drucker refers to as the ``incunabula”—or cradle— of humanistic information design \citep[p. 176]{drucker_graphesis_2014}.
\index[terms]{TCA Workspace} 
\index[people]{Drucker, Johanna}

To comment on the current state of information software since Drucker's \textit{Graphesis} (2014), bibliometric platforms like Obsidian and Litmaps are among the emerging ``new conventions that do not rely on book structures" \citep[p. 176]{drucker_graphesis_2014} and leverage ``informational derivatives of data mining, analytics, visualization, and display” to represent networked relationships of academic dialogue, as Drucker describes \citep[p. 176]{drucker_graphesis_2014}. However, TCA Workspace would move a step further toward ``the creation of an interface that is meant to expose and support the activity of interpretation, rather than to display finished forms" \citep[p. 179]{drucker_graphesis_2014}.
\index[terms]{TCA Workspace}
\index[people]{Drucker, Johanna}



\subsection{Conclusion to the discussion of TCA Workspace}

TCA Workspace is an approach to bibliometric analysis that aligns with Drucker's vision of humanistic information design, focusing on dynamic, relational scholarly activity and interpretative processes. I believe current open-source PKM hyperlinked writing platforms like Obsidian are most of the way to Drucker’s ``future book”, which will ``call to the vast repositories of knowledge, images, interpretation, and interactive platforms.” \citep[p. 175]{drucker_graphesis_2014}. Books, PKM or otherwise, will continue to be ``an interface, a richly networked portal, organized along lines of inquiry in which primary source materials, secondary interpretations, witnesses and evidence, are all available, incorporated, [and] made accessible for use” \citep[p. 175]{drucker_graphesis_2014}. 
Current PKM ``books of the future” can be especially productive when using Luhmann's Zettelkasten method \citep{ahrens_how_2017,cevolini_forgetting_2016,luhmann_kommunikation_1981}\footnote{For the list of tools I used in my knowledge management system, including its Zettelkasten tools like Obsidian, see Appendix \autoref{Appendix Resources and Tools}.} in conjunction with reference managers like Zotero.
\index[terms]{TCA Workspace} 
\index[people]{Drucker, Johanna}
\index[people]{Luhmann, Niklas}


By proposing TCA Workspace I aim to push ``the ``book" of the future" further forward using Topological Capta Analysis of texts and graphs. Developing these and other methods of hyperlinked information decentralization can help all researchers access new tools for a shift analogous to Hegel’s, extolled by Derrida as ``the last philosopher of the book, and the first thinker of writing” \citep[p. 26]{derrida_grammatology_1997}. My proposal for TCA Workspace is about developing interfaces that shift emphasis from the book toward the wider decentralized rhizome \citep{deleuze_thousand_2007}\footnote{Characteristics of the rhizome are articulated in Deleuze and Guattari’s \textit{A thousand plateaus} ``A rhizome has no beginning or end; it is always in the middle, between things, interbeing, \textit{intermezzo}. The tree is filiation, but the rhizome is alliance, uniquely alliance” \citep[p. 25]{deleuze_thousand_2007}.} of interconnected writing and graphing. Ultimately, by proposing TCA Workspace, I aim to facilitate more agency over climate complexity through interfaces that manage more dynamic visuospatialized relationships between capta.
\index[terms]{visuospatialization}
\index[terms]{TCA Workspace} 
\index[people]{Drucker, Johanna}
\index[people]{Deleuze, Gilles}
\index[people]{Guattari, Félix}
\index[terms]{interbeing}
\index[terms]{rhizome} 
\index[terms]{capta}