\chapter{Discussion of contributions}
In this section, I consider the sorts of implications to realizing Semantic Forms and Query Isomorphs to fuller potential.

To recontextualize this discussion I re-introduce the problem. The climate crisis continues to accelerate faster than deployment of climate knowledge can keep up, so we find ourselves in a crisis of understanding. We must work toward new research paradigms and frameworks to strategize when facing the vast amount, interdependent nature, and exponential growth of information in the work of Knowledge Activation (KA). We must differentiate the various modes of KA to lead with lower complexity and ecological cost when possible working from knowledge \textbf{surfacing}, to \textbf{synthesis}, to \textbf{translation}, to \textbf{production} only after the others have been used (KSSTP). Knowledge Production is and will continue to be necessary, but will be limited due to its high complexity and cost. 

My thesis presents methods for activating large texts in the climate crisis as a means of Knowledge Activation (KA). KA is both subject and method in this thesis. My literature review and geometric analysis were KA to develop models, or forms, from theory. Visuospatial Systematic Combining was KA for developing theory and methodology using form. In short, this thesis engaged in the inter-informing visuospatial epistemological methods for arriving at form through meaning, and for arriving at meaning through form. 

\section{C1. Semantic Forms}


\noindent\textbf{In RQ1.1 I asked:}

\textit{What forms of spatial information visualization are there, and how can they inform the composition of network graphs?}

As a result of my research, I proposed \textbf{C1, Semantic Forms}, a taxonomy of three-dimensional topic model compositions for Human-in-the-Loop Computational Analysis of Texts and Graphs (HITL CATG), Human Analysis of Texts and Graphs (HATG), or both.

\vspace{1em}

\subsection{Vedic visuospatial culture}
My survey of symbols provided a pivotal part of my thesis in connecting two-dimensional ideograms to three-dimensional versions of themselves. The cone Semantic Form was inspired by this interbeing of the Sri Yantra and the Meru Chakra. The etymology of the word yantra as “machine” \citep[p. 28]{buhnemann_mandalas_2003} is activated beyond its two-dimensionality into three-dimensional Meru Chakra-inspired \citep[p. 31]{buhnemann_mandalas_2003} cone Semantic Form “Literary Machines” \citep{nelson_literary_1981} for “knowledge engineering” \citep[p. 8]{wielinga_kads_1992} as a pivotal part my TCA Workspace “studio laboratory of knowledge design” \citep[p. 197]{drucker_graphesis_2014}.
\index[terms]{TCA Workspace}
\index[terms]{Sri Yantra}
\index[terms]{Meru Chakra}
\index[terms]{interbeing}

While I do not invoke the associations to warfare in the etymology of the word \textit{yantra} \citep[p. 28]{buhnemann_mandalas_2003}, I do propose that this work can facilitate genitive disagreement and anti-oppressive resistance.  

\subsection{Geometry and Topology}

Semantic Forms are presented as geometric forms, but they encompass topological versatility. The ways their form can be used to interpret and reveal semantic relationships is not limited to a particular size of data set, nor is it limited to a particular ratio of more consilient terms to less consilient terms. Therefore their shape is a heuristic for the ways they work with semantic relationships, not a prescription. Furthermore, Semantic Forms can be nested within each other, and overlapping between each other, which further demonstrates the requirement for a multi-mathematical approach.

The computational operationalization of Semantic Forms will benefit from calculating the geometric topology. For example, the calculation of angles between edge vertices (Degree Difference), will inform how “similar or dissimilar neighbouring vertices are with respect to some quantity” \citep[p.2]{farzam_degree_2020} (Global Assortativity). Homophily, which measures “the tendency to associate with like-minded or otherwise similar people” \citep[p.2]{farzam_degree_2020}, is already calculated using Degree Difference and Global Assortativity \citep[p.2]{farzam_degree_2020}.
\index[terms]{Semantic Forms} \index[terms]{Query Isomorphs} 

\subsection{Dimensional addition}

Semantic Forms are derived from a Systematic Combination approach to dimensional addition of Semantic Shapes. Semantic Shapes are the two-dimensional units of network graph composition that are combined into Semantic Forms. 
\index[terms]{Systematic Combining (SC)}

Ware and Mitchell report that the move from two-dimensional to three-dimensional node graphs increase “the size of the graph that can be “read”” by “roughly an order of magnitude” \citep[p. 10]{ware_visualizing_2008}. Tversky’s investigations on the neuroscience of the visuospatial indicate that brains evolved to handle position in space before they evolved the ability to handle language. In Tversky’s words, that “spatial thinking is the foundation of all thought” and that “the foundation for spatial thought is also the foundation for conceptual thought” \citep{tversky_barbara_2022}.
\index[people]{Tversky, Barbara}

Considering interfaces for the visuospatial that rely on our sense of sight, the pivotal middle area that includes human visuospatial ability and higher-dimensional graphs is the third dimension (with or without movement) since we can sense no further. LLMs embed many billions of pieces of information \citep[p. 1]{brown_language_2020} as tensors, or ‘directions’ in high-dimensional vector space \citep{sanderson_attention_2024}. We rely on CATG, and hopefully more HITL CATG, when it comes to Language Models. 
\index[terms]{Large Language Model (LLM)}

Drucker’s visual forms of knowledge production \citep{drucker_graphesis_2014}, visual reasoning and visual argument  \citep{tversky_barbara_2022}, are also, then, visuospatial forms of Knowledge Activation when developed in tandem to Tversky’s work and KSSTP. 
\index[people]{Tversky, Barbara}

\subsection{Three-dimensional composition}

The value of indexing forms for their various capacities for interpreting and revealing semantic relationships in HATG, HITL CATG, or both, is supported by their use in various areas of research. Grant et al. propose that the meta-analysis literature review can be illustrated using a funnel plot \citep[p. 94-95]{grant_typology_2009}. Taylor’s cones of plausibility even show the presence of double-cones \citep[p. 14]{taylor_creating_1990}, and provide the basis of Bezold and Hancock’s futures cone \citep[p. 73]{bezold_overview_1993}. Specifically within the area of climate, Kjøde’s on the \textit{Entanglement of Systemic Design and Sustainability Transitions uses Systematic Combining} composed with the cone \citep[p. 123]{kjode_entanglement_2024}, sphere  \citep[p. 125]{kjode_entanglement_2024}, and cylinder  \citep[p. 144]{kjode_entanglement_2024}. Lenat et al.’s illustration of upper ontology in relation to a middle ontology that relates to various domain models depicts a sort of mountain range of meaning. I propose that their cone-like composition, both individually and as a group, is remarkable because it captures the form of Adler’s “syntopical reading” \citep[p. xi]{adler_great_1952-2}, Consilience \citep{hepburn_scientific_2021,wilson_consilience_1999}, and Peirce’s abduction \citep[p. 106]{peirce_pragmatism_1960} \citep[p. 20]{sowa_challenge_2004}. 
\index[terms]{Systemic Design} \index[terms]{abduction}
\index[people]{Peirce, Charles Sanders}
\index[people]{Adler, Mortimer J.}
\index[people]{Kjøde, Svein Gunnar}
\index[terms]{Systematic Combining (SC)}


I have not found an explicit index of three-dimensional forms used to to interpret and reveal semantic relationships. I aim for my Semantic Forms to be applied in (a) computational approaches like global topological synchronization (GTS) using filtration \citep[p. 1]{kovacs_iterative_2024} \citep{giusti_twos_2016}, bundling \citep[p. 1]{holten_forcedirected_2009}, weighting \citep[p. 2]{kovacs_iterative_2024}, and my magnetic approach to node grouping; (b) manual node placement in Systemic Design approaches like Systems Oriented Design \citep{sevaldson_designing_2022}, gigamapping \citep{sevaldson_giga-mapping_2011}, Systematic Combining \citep[p. 556]{dubois_systematic_2002} \citep[p. 46]{kjode_entanglement_2024}, Boundary Critique \citep{midgley_theory_1998}; and, (c) in some combination of both where HATG can us used independently or in tandem with HITL CATG.
\index[terms]{Systemic Design}
\index[people]{Sevaldson, Birger}
\index[people]{Kjøde, Svein Gunnar}
\index[terms]{Systematic Combining (SC)}
\index[people]{Gadde, Lars-Erik}
\index[people]{Dubois, Anna}
\index[terms]{gigamapping}

My proposal for Semantic Forms, and the ongoing practice of defining them aims to benefit design, climate science, and the many various fields related to them. Inspired by the work of Manuel Lima and his works The book of trees \citep{lima_book_2014} and The book of circles \citep{lima_book_2017}, it is my hope that the use of three-dimensional information visualizations will lend themselves to books of cones, spheres, cylinders, and tori.
\section{C2. Query Isomorphs}

\noindent\textbf{In RQ1.2 I asked:}

\textit{How can Computational Analysis of Texts and Graphs (CATG) identify isomorphic semantic structures within large network graphs?}

As a result of my research, I proposed \textbf{C2, Query Isomorphs}, as a means of Topological Capta Analysis (TCA) in HITL CATG using small graph chunks.

\vspace{1em}


Query Isomorphs are smaller graph chunks used in HITL CATG. Semantic Forms necessarily include Query Isomorphs, but Query Isomorphs can be part of any network graph regardless of whether or not that graph is a Semantic Form or uses Semantic Forms. 


\subsection{Topological network query graphlets}

Query Isomorphs are necessarily topological in that they interpret and reveal semantic relationships across nested hierarchies through the connections between individual nodes. These connections can form graphlets or “network subgraphs” with “a small number of nodes” 
\citep[p. 5-6]{przulj_modeling_2004} in various permutations \citep[p. 3]{sarajlic_graphlet-based_2016}. 

Considering the multi-dimensional range of TDA, I propose Query Isomorphs interpret and reveal meaning in low-dimensional formats, and as simplicial complexes, “higher-order networks that encode higher-order topology and dynamics of complex systems” \citep[p. 1]{wang_global_2024}. 
\index[terms]{simplicial complex} 

The earliest research I found which leverages the graph isomorphology for query that I incorporate into my Query Isomorphs is the VivoMind Analogy Engine (VAE). The VAE uses Sowa’s conceptual graphs and query graphs to match labels, subgraphs and graph transformations \citep[p. 22]{sowa_analogical_2003}. 
\index[people]{Sowa, John F.}

The potential for Query Isomorphs in AI-powered semantic isomorphological query is demonstrated in TerminusDB’s platform. Their vector search enables “efficient and accurate similarity-based querying” and “clustering algorithms to gain valuable insights from massive datasets,” modeling data “in a more semantically meaningful manner, enabling a deeper understanding of the connections between different entities” \citep{terminusdb_enterprise_2023}.

These graphlets can be imbued with entity-relationship information including directed vectors in keeping with Chen’s ER diagram notation \citep{chen_entity_1976,rodina_chen_2024}. These query graphs can be used to reveal information from larger networks \citep[p. 313]{sowa_conceptual_1984}. Graphlets can be interpreted and revealed in large networks across dimensions using TDA \citep[p. 1]{aktas_persistence_2019}. Furthermore, Query Isomorphs will benefit from development using geometric topology approaches like Degree Difference \citep{farzam_degree_2020}.

My gap analysis of my literature review identifies aspects that are present in one area of literature but absent in another. I propose joining the various qualities of ER, query graphs, visuospatial forms of knowledge production, and TDA into Query Isomorphs as visuospatial directed query graphlets that can be to queries in HITL CATG. 
\index[people]{Sowa, John F.}


\subsection{Weighting}

A common problem in weighting is the loss of information through filtration used to assess hierarchical structure in weighted networks. “In some situations, like measurements of correlation or coherence of activity, the resulting network has edges between every pair of nodes and it is common to threshold the network to obtain some sparser, unweighted network whose edges correspond to “significant” connections” \citep[p. 11]{giusti_twos_2016}. By leveraging persistent homology \citep[p. 11]{giusti_twos_2016} in TDA, the multi-scale topological features of data can be analyzed using Semantic Forms, while preserving critical information that might be lost through conventional thresholding methods. 

Furthermore, since Query Isomorphs resemble chemical molecules in various ways, point weighting can be applied to capture and reveal semantic relationship of ideas by representing them these molecules or datacules as “a union of balls in Euclidean space” \citep[p. 14]{otter_roadmap_2017}. 

\subsection{Interdisciplinarity powered by HITL CATG}

By employing Query Isomorphs in HITL CATG KA, researchers can query for isomorphic network graphs inside larger text graphs, facilitating the ability to detect and reveal analogous knowledge structures across different disciplines' datasets. 

In conclusion, Query Isomorphs offer a powerful approach for operationalizing dimensional versatility of TDA in HITL CATG for interdisciplinary KA by finding isomorphic semantic structures within large network graphs.
\section{C3. Ontological Semantic Network Graphing (OSNS) }
\noindent\textbf{In RQ1.3 I asked:}

\textit{What methods can reveal the relationships between fundamental ideas in texts, graphs, or Large Language Models (LLMs)?}

As a result of my research, I proposed \textbf{C3, Ontological Semantic Network Summaries (OSNS)}, as a means of revealing ontological relationships between ideas in a given body of research using HITL CATG, HATG, or both.

\vspace{1em}
\index[terms]{Large Language Model (LLM)}

OSNS is a method to interpret and reveal ontological relationships using semantic networks in HATG, HITL CATG, or both. OSNS, like any other form of network graph, can be integrated with Semantic Form and Query Isomorph approaches for KA. 

We owe a great deal to philosophers like Aristotle and Porphyry as references for how we approach ontology, or the “systematic account of Existence” \citep[p. 1]{gruber_toward_1995} Ontology is fundamental in computer science, in which “what “exists” is that which can be represented” \citep[p. 1]{gruber_toward_1995}. As a definition that can encompass philosophy and computer science, then, ontology can be considered “an explicit specification of a conceptualization” \citep[p. 1]{gruber_toward_1995}.
\index[people]{Aristotle}
\index[people]{Porphyry}

Naming and relating the things we consider to exist is illusive yet fundamental to the various modes of reasoning named in this thesis like Adler’s “syntopical reading” \citep[p. xi]{adler_great_1952-2}, Consilience \citep{hepburn_scientific_2021,wilson_consilience_1999}, and Peirce’s abduction \citep[p. 106]{peirce_pragmatism_1960} \citep[p. 20]{sowa_challenge_2004}.  By using OSNS researchers can examine syntopical, consilient, or abductive terms to each other in an overview of how key terms relate, including the inheritance of key properties like in the Tree of Porphyry. In this way OSNS can serve to categorize and compare texts.
\index[terms]{abduction}
\index[people]{Peirce, Charles Sanders}
\index[people]{Adler, Mortimer J.}
\index[people]{Sowa, John F.}


\subsection{AI}

I propose OSNS can also serve to bring people more in the loop during language model training, and to compare language models after they are trained. OSNS, then allows researchers to assess the depth and breadth of an LLM’s proficiencies, gaps, and biases in a given language domain. 
\index[terms]{Large Language Model (LLM)}

In short, I propose OSNS as a tool to interpret and reveal ontological positioning of texts and LLMs to navigate constellations of information more effectively in KA. 



\section{C4. Symbol-setting}

\noindent\textbf{In RQ1.4 I asked:}

\textit{Given the semantic versatility of symbols, how can a practice of collaborative symbol-making support knowledge production?}

As a result of my research, I proposed \textbf{C4, Symbol-setting}, a method for expanding the semiotic range of knowledge production using symbol co-creation in HITL CATG, HATG, or both.

\vspace{1em}
\index[terms]{Large Language Model (LLM)}

As we’ve discussed, Knowledge Activation happens in multiple modes, with and without written words. Visuospatial forms of Knowledge Activation can use the wide range of meaning-making methods available to us to interpret and reveal meaning that written words may not be able to capture effectively.

I expand Pangaro’s “Conversation to Agree on Goals” \citep[p. 185]{pangaro_design_2011} and the Equity-Centred Community Design framework’s “language setting” \citep[p. 8-9]{creative_reaction_lab_equity-centered_2018} with my proposal to use the wider variety of tools available in visuospatial epistemology, or Symbol-setting. 
\index[people]{Pangaro, Paul}
\index[terms]{visuospatial epistemology}

Similarly to Sevaldson’s perspective on Systems Oriented Design, I propose that Symbol-setting is an open-source process design “methodology without a method” that does not prescribe specific techniques. \citep[p. 30]{sevaldson_designing_2022}.
\index[people]{Sevaldson, Birger}

Nonetheless, I do offer a starting point with and suggest that practitioners inform themselves of symbol composition \citep{liungman_thought_1995}, information visualization composition \citep{vital_how_2018,ribecca_data_2017}, isomorphology \citep{anderson_drawing_2018}, Systems Oriented Design \citep[p. 343]{sevaldson_designing_2022}, gigamapping \citep{sevaldson_giga-mapping_2011} \citep[p. 26]{sevaldson_designing_2022}, synthesis mapping \citep{jones_synthesis_2016} 
\citep[p. 129]{jones_synthesis_2017}, Systematic Combining \citep[p. 554]{dubois_systematic_2002} 
\citep{ kjode_entanglement_2024}, Boundary Critique \citep{midgley_theory_1998}, Semantic Forms, Query Isomorphs, OSNS, and the various contributions of my thesis. 
\index[people]{Sevaldson, Birger}
\index[people]{Kjøde, Svein Gunnar}
\index[terms]{Systematic Combining (SC)}
\index[terms]{Boundary Critique}
\index[people]{Midgley, Gerald}
\index[people]{Liungman, Carl G.}
\index[people]{Vital, Anna}
\index[terms]{gigamapping}


In relation to the larger practices of three-dimensional analysis of words, Liungman’s \textit{Thought Signs} \citep{liungman_thought_1995} make an excellent case study for the spatial analysis of symbols. \textit{Thought Signs} is particularly ready for such an analysis because of its interrelated classification criteria. Another starting point for such an analysis is the MNIST handwritten digits database \citep{lecun_mnist_2012} and its use in the Embedding Projector on TensorFlow \citep{smilkov_embedding_2016}. Plotting the relationships of symbols into three-dimensional space would allow a researcher to discover new semantic patterns and categorizations within and between symbols and their compositions.  
\index[people]{Liungman, Carl G.}

In short, Symbol-setting serves as a method for the semiotic expansion of the linguistics of Knowledge Activation and co-production. Symbol-setting can foster interdisciplinary understanding while emphasizing inclusion and community.
\section{C5. Terroir of Text and Graphs (TTG) }

\noindent\textbf{In RQ1.5 I asked:}

\textit{How can CATG reveal relationships between place and text?}

As a result of my research, I proposed \textbf{C5, Terroir of Text and Graphs (TTG)}, a method of HITL CATG that uses TCA to interpret and reveal semantic relationships between (a) texts and graphs, and (b) the features and systems of ecological place.

\vspace{1em}


Terroir is a term used to describe the unique qualities of taste imparted to wine by environmental factors of a specific region. In the context of text and graphs, TTG serves as an analogy \citep[p. 20, 28]{sowa_challenge_2004} to ground the “constellations field” of “thought forms” \citep[p. 196]{drucker_graphesis_2014}. TTG is a means of revealing how semantic relationships of texts and graphs are co-influencing with their ecological and cultural context. 
\index[people]{Drucker, Johanna}

TTG aims to support experts of ecologically sustainable ways of living whose insights are vital to ST and climate resilience. Centering Indigenous people, TTG can be used to interpret and reveal dynamics of land-based practices that can inform better settler sustainability practices. Indigenous knowledge keepers could benefit from engaging with new methods of relating place to ancestral wisdom. HITL CATG in the form of TTG can provide analysis of the tangible and intangible elements of culture that can be interpreted by language and diagram. 

Existing HATG reveals some early connections between the ways environmental place shapes language. In India, “rounded forms and wavy lines […] characterize most of the southern scripts of subsequent centuries up to modern times” \citep[p. 39]{salomon_indian_1998}. These forms, which vary so drastically from the letters of the Latin alphabet used for this thesis, are “usually explained as a result of the exigencies of writing with a stylus on palm leaves” \citep[p. 39]{salomon_indian_1998}. TTG allows us to consider the interplay of place and text as a central element of cultural insight, adding grounded depth to KA for ST.

Of course, many other types of relationships between place and language exist. So, I propose TDA as a method to better understand the relationship between the ecological and conceptual as a means of improving the translation of information from one ecologically-located discipline or tradition to another.  Using ecological TDA more broadly we can expect TTG to reveal many interrelated characteristics of a particular thought tradition which are co-informed by characteristics of their place.

TTG can foster the integration of diverse wisdom lineages, including anthropo-symbiotic\citep{fonseca_anthropo-symbiotic_2022} Indigenous ways of knowing, into the broader scope of Systems Thinking (ST). In this way, TTG is simultaneously a practice of ecojustice by elevating Indigenous and traditional knowledge systems, while also providing KA that informs ST. A TTG-based HITL CATG of fields such as ethnobotany, ethnoecology, biocultural memory, and plant-human co-evolution represents a critical direction for revealing otherwise overlooked patterns and connections, and the various other processes of KA.

Like any other form of CATG, TTG can be integrated with Semantic Forms, Query Isomorphs, OSNS, and Symbol-setting approaches to KA. As a form of Symbol-setting, TTG interprets reveals the symbolic and material relationships that inform how multiple communities can come to understand their shared and related environments. In this way, TTG reveals the connection of collective knowledge systems to place to symbol, story, and material culture. 

TTG is particularly useful for supporting Indigenous research about AI. By examining the significance of place in Knowledge Activation and situating  KA within the specific cultural and ecological narratives that shape it, TTG offers a way to build AI technologies that respect and incorporate Indigenous epistemologies. Using TTG can help build a richer understanding of the ways our words and graphs interpret and reveal relationships to place, and in doing so foster a deeper connection between ancestral narrative and ST-informed technological development.
\section{C6. TCA Researcher Grouping}


\noindent\textbf{In RQ1.6 I asked:}

\textit{What strategies can improve university knowledge management to accelerate research?}

As a result of my research, I proposed \textbf{C6, TCA Researcher Grouping}, a proposal to use TCA for grouping research collaborators more effectively using HITL CATG, HATG, or both.

\vspace{1em}


Among the various ways researchers and institutions can help or hinder STKA is how we get to know each other to collaborate. Current bibliographic platforms tend to be limited by superficial information which limits their ability to pair people together based on aspects of their research which may yet be implicit. Even the platforms capable of revealing research terms shared by researchers, like the AI-assisted InfraNodus, were not equipped to analyze large numbers of researchers and their respective bodies of work in my testing.

TDA of research databases can catalyze better connections among researchers, increasing the impact of KA, and the quality of the researcher experience. The versatility of the variety of methods I propose in this thesis–Semantic Forms, Query Isomorphs, OSNS, and TTG among them–looks to connect research across conventionally siloed disciplines by supporting the people that make KA happen.   

Current PKM tools, such as Obsidian, are effective for individual information management but my testing has shown that they struggle to scale up to the demands of larger datasets and complex repositories. TCA Researcher Grouping would build on the hyperlinked format of PKM, but would necessarily have to operate with a versatility of scale, hence my urgency to proposal to apply TDA. 

To borrow an ecological metaphor, TCA Researcher Grouping seeks to catalize solutions through better disciplinary cross-pollination in KA. To further develop the ecological metaphor in traditional agricultural practices like permaculture, TDA in TCA Researcher Grouping is a sort of rewilding beneath the written page. 

In the context of the climate crisis, it is fundamental that we develop information technology that involves all the various aspects of KSSTP (KA) to better leverage our research resources. TCA Researcher Grouping seeks to build on the capabilities of existing PKM tools like Obsidian, and topic modeling tools like InfraNodus, and improve upon them. Developing knowledge management tools for institutions, like TCA Researcher Grouping, is a critical step towards better STKA across disciplines. 
\section{C7. TCA Workspace}
\noindent\textbf{In RQ1.7 I asked:}

\textit{What sort of knowledge work software would I want to build to incorporate my various findings and use spatial information visualization to identify isomorphic semantic structures, reveal the relationships between fundamental ideas, build on collaborative symbol-making, reveal relationships between place and text, and improve university knowledge management?}

As a result of my research, I proposed \textbf{C7, TCA Workspace}, a proposal for a collaborative HITL CATG + HATG platform to:

\begin{enumerate}[label=\textbf{(\alph*)}, leftmargin=2em]
    \item \textbf{House all my thesis contributions} (\textit{Semantic Forms}, \textit{Query Isomorphs}, \textit{OSNS}, \textit{Symbol-setting}, \textit{TTG}, and \textit{TCA Researcher Grouping}).
    \item \textbf{Facilitate their combined use} with Systemic Design methods for visuospatial reasoning discovered through my literature review, such as gigamapping \citep{sevaldson_giga-mapping_2011}, \citep[p.~26]{sevaldson_designing_2022} and Systematic Combining \citep[p.~554]{dubois_systematic_2002}, \citep{kjode_entanglement_2024}.
\end{enumerate}
\index[terms]{Systemic Design}
\index[people]{Sevaldson, Birger}
\index[terms]{gigamapping}

TCA Workspace is my proposed starting point for a shared effort between Systemic Design methods and topic modeling by using Semantic Forms, Query Isomorphs, and Topological Capta Analysis. 
\index[terms]{Systemic Design}

\subsection{Design and philosophy}

I propose TCA Workspace as a humanistic and post-structuralist interface, emphasizing the interpretative nature of language, information, and knowledge in KA. Inspired by thinkers such as Eagleton, Barthes, Drucker, and Thích Nhất Hạnh, TCA Workspace treats texts as multidimensional spaces where writing and graphs can intersect to interpret and reveal meaning. 
\index[people]{Drucker, Johanna}
\index[people]{Barthes, Roland}
\index[people]{Eagleton, Terry}
\index[people]{Thích Nhất Hạnh}

TCA Workspace embodies an ontological ethos that aims to engineer and implement a “diagrammatic and constellationary rhetoric” within an “infinitely extensible field” of new legibility conventions \citep[p. 197]{drucker_graphesis_2014}. It strives to move beyond disciplines that are “antithetical to interpretation”, focusing instead on revealing the “constructedness of knowledge” \citep[p. 178] {drucker_graphesis_2014}.
\index[people]{Drucker, Johanna}


TCA Workspace seeks to respond to climate complexity by developing cybernetic forms that better leverage Tversky’s neuroscience of the visuospatial \citep{tversky_barbara_2022} and bring new life to Drucker’s interpretation of humanistic information interfaces. By using “techniques of semantic web, topic maps, network diagrams, and other computational means” TCA Workspace seeks to ‘spatialize’ arguments and “relations among units of thought” for reconfiguration of their “constellationary form” \citep[p. 158] {drucker_graphesis_2014}.
\index[people]{Drucker, Johanna}
\index[people]{Tversky, Barbara}



\subsection{Integration of theoretical perspectives}
TCA Workspace aims to embrace fluid ontologies and diverse classifications, affording a new and richer way of representing knowledge structures that reveal rather than conceal their "constructedness" \citep[p. 178]{drucker_graphesis_2014}. These various perspectives align with the fundamental principles of TCA Workspace, in which knowledge is not understood to be static or definitive, but more like an interwoven network of meanings that can be dynamically interpreted.
\index[people]{Drucker, Johanna}

Barthe writes that “The text is a tissue of quotations drawn from the innumerable centres of culture” \citep[p. 146]{barthes_image_1977}. TCA Workspace is less about definition and more about the way ideas exhibit ‘interbeing’, the interpenetrating unity-and-diversity of being \citep[p. 80]{nhat_hanh_world_2008}. 
\index[people]{Thích Nhất Hạnh}
\index[people]{Barthes, Roland}
\index[terms]{interbeing}

\subsection{TCA Workspace as a tool for addressing climate complexity}

Climate is among the greatest challenges of our time, if not the greatest challenge. So, any development in any field is contextualized by or oriented towards addressing it, whether we choose to or not, and a choice not to is a choice also. “Silence = Death” as Finkelstein and Gran Fury so briefly put it \citep{finkelstein_after_2018}. 

The growing and massive amount of information in the many disparate fields of ST \citep{grin_transitions_2011} makes the development of research platforms like TCA Workspace all the more relevant. As a “studio laboratory of knowledge design” \citep[p. 197]{drucker_graphesis_2014} I aim for to TCA Workspace advance collective understanding necessary for mitigating the climate crisis using the various modes of KA. 
\index[people]{Drucker, Johanna}


\subsection{Humanist post-structuralist design}

Eagleton’s description vividly characterizes Query Isomorphs, Semantic Forms, and TCA Workspace: “The 'writable' text, usually a modernist one, has no determinate meaning, no settled signifieds, but is plural and diffuse, an inexhaustible tissue or galaxy of signifiers, a seamless weave of codes and fragments of codes, through which the critic may cut his own errant path” \citep[p. 119]{eagleton_literary_2006}. Similarly Barthes addresses decentralization and dimensionality: “We know now that a text is not a line of words releasing a single 'theological' meaning (the 'message' of the AuthorGod) but a multi-dimensional space in which a variety of writings, none of them original, blend and clash. The text is a tissue of quotations drawn from the innumerable centres of culture” \citep[p. 146]{barthes_image_1977}. From a post-structuralist non-theistic Buddhist ecological perspective, Thích-Nhất-Hạnh articulates the concept of  ‘interbeing’, the interpenetrating unity-and-diversity of being \citep[p. 80]{nhat_hanh_world_2008} \footnote{“There is no phenomenon in the universe that does not intimately concern us, from a pebble resting at the bottom of the ocean to the movement of a galaxy millions of light years away. All phenomena are interdependent. When we think of a speck of dust, a flower, or a human being, our thinking cannot break loose from the idea of a self, of a solid, permanent thing. We see a line drawn between one and many, this and that. When we truly realize the interdependent nature of the dust, the flower, and the human being, we see that unity cannot exist without diversity. Unity and diversity interpenetrate each other freely. Unity is diversity, and diversity is unity. This is the principle of interbeing” \citep[p. 80-81]{nhat_hanh_world_2008}.}, which aligns with the philosophy of TCA Workspace.
\index[people]{Thích Nhất Hạnh}
\index[people]{Barthes, Roland}
\index[terms]{interbeing}

To summarize TCA Workspace by turning towards language more explicitly about interface, I recall when Johanna Drucker mused the following: 
\index[terms]{TCA Workspace}

“Are we merely part of an emerging constellation of potentialities for realization of aspects of knowledge design and interpretative acts that are closer to our once-sensible reading of natural and cultural landscapes? Perhaps we are reawakening habits of associative and spatialized knowledge we once read and through which we knew ourselves. We may yet awaken the cognitive potential of our interpretative condition of being, as constructs that express themselves in forms, contingently, only to be remade again, across the distributed condition of knowing” \citep[p. 192]{drucker_graphesis_2014}. 
\index[people]{Drucker, Johanna}

As a critical design for interpretative and humanistic interfaces, TCA Workspace facilitates interpretative activity by embracing “inconsistency among types of knowledge representation, classification, fluid ontologies, and navigation” \citep[p. 178]{drucker_graphesis_2014}. TCA Workspace represents an evolution in scholarly tools, exemplifying a more dynamic, relational, and interpretative approach to humanistic information design.
\index[terms]{TCA Workspace}
\index[people]{Drucker, Johanna}


\subsection{The future of TCA Workspace}

TCA Workspace’s ambition is not just to be a static tool but to evolve in tandem with emerging methods in digital humanities as a larger practice of making Visuospatial Knowledge Activation interface. TCA Workspace aims to be a key part of the future Saint-Martin and Drucker envision where topological concepts are integrated into semantic and textual analysis. In it, we contribute to a new language for interpreting graphical relationships in digital spaces \citep[p. 225]{saint-martin_semiotics_1990} \citep[p. 54]{drucker_graphesis_2014}.
\index[terms]{Visuospatial Knowledge Activation (VKA)} \index[terms]{TCA Workspace}
\index[people]{Drucker, Johanna}
\index[people]{Saint-Martin, Fernande}

Drucker captures the epistemological impetus of TCA Workspace and urges: “We have to find graphical conventions to show uncertainty and ambiguity in digital models, not just because these are conditions of Knowledge Production in our disciplines, but because the very model of knowledge itself that gets embodied in the process has values whose cultural authority matters very much” \citep[p. 190-191]{drucker_graphesis_2014}. 
\index[terms]{TCA Workspace}
\index[people]{Drucker, Johanna}

I propose that Saint-Martin’s “semantic system of topological semiotics” \citep[p. 225]{saint-martin_semiotics_1990} and Drucker’s “constellationary field” \citep[p. 196]{drucker_graphesis_2014} capture the ‘space’ in TCA Workspace. My proposed platform, TCA Workspace, facilitates a multi-mathematical framework for the Computational Analysis of Texts and Graphs (CATG). Semantic Form “thought forms” \citep[p. 196]{drucker_graphesis_2014} can be interpreted and revealed with Query Isomorphs and TDA. This approach can allow us to study the ways thought and form co-inform each other as “content” and “configuration of knowledge” \citep[p. 196]{drucker_graphesis_2014}. The “organizing orders of graphical expression that take on their own legibility” \citep[p. 196]{drucker_graphesis_2014} as the semantic relationships revealed with HATG and HITL CATG. More generally, TCA Workspace integrates semantic topological semiotics and constellationary fields to analyze texts and graphs and find the knowledge structures and connections within them.
\index[terms]{TCA Workspace}
\index[people]{Drucker, Johanna}
\index[people]{Saint-Martin, Fernande}


\subsection{Living networks of decentralized knowing}

Current tools like keyword searches and statistical digital scholarship methods are limited in their effectiveness to navigate hyperspecialized, jargon-laden texts and graphs across diverse disciplines. They often fail to reveal latent transdisciplinary or interdisciplinary insights. 

With TCA Workspace I aim to bridge innovation paths across disciplines, enabling specialists to identify novel approaches they might otherwise overlook. This is especially critical in STKA now, when we may already possess the insights we need buried in our embarrassment of informational riches concealed by context or by form. My aim is for researchers’ use of TCA Workspace to act as a cybernetic mycelium that reconstitutes the matter left behind after the “death” of an idea, activating words and graphs “across the distributed condition of knowing” \citep[p. 192]{drucker_graphesis_2014}. 
\index[terms]{cybernetic rhizome}	
\index[terms]{rhizome} 

To solve complex challenges, we must examine the “detritus” of Knowledge Production and understand what is valuable to create insights. KA with TCA Workspace can catalyze new forms of knowledge resurfacing, synthesis, translation, and production (as necessary), to reactivate the dormant ideas in our web of semantic interpretation into living networks of decentralized knowing. 


\subsection{Addressing information overload through interdisciplinary collaboration}

In the age of massive data, we need new tools to facilitate functional interdisciplinary collaborations. More intuitive interfaces need to be developed with discipline-based approaches for representing network graphs. By developing a shared terminology with syntopical language-setting, built into the TCA Workspace platform for creating shared visual models made with Symbol-setting via gigamapping and Systematic Combining, and by querying our models through increased dimensionality and TDA of Semantic Forms using Query Isomorphs, we can amplify the impact of knowledge in diverse academic disciplines. Furthermore, with AI, we can reduce the cost of Knowledge Production by surfacing, synthesizing, and translating knowledge that has already been produced in texts, marginalia, and commentaries into the other forms of KA.
\index[terms]{gigamapping}

\subsection{Humanistic design elements of TCA Workspace}

TCA Workspace and any form of computational text analysis are indebted to the work of Vannevar Bush’s memex \citep{bush_as_1945}, Ted Nelson’s hypertext \citep{nelson_literary_1981}, Tim Berners-Lee’s World-Wide Web. TCA Workspace, however, seeks to build on these foundational technologies, as part of what Drucker identifies as the “incunabula” of humanistic information design \citep[p. 176]{drucker_graphesis_2014}.
\index[terms]{TCA Workspace} 
\index[people]{Drucker, Johanna}

To comment on the current state of information software since Graphesis (2014), bibliometric platforms like Obsidian and Litmaps are among the emerging new software implementations for text interpretation that do not depend on the structure of the book \citep[p. 176]{drucker_graphesis_2014}. TCA Workspace seeks to build beyond existing information technology to further empower scholarly activity as relational and dynamic, emphasizing process over product.
\index[terms]{TCA Workspace}
\index[people]{Drucker, Johanna}

TCA Workspace leverages informational derivatives from data mining, analytics, and visualization to represent networked relations and scholarly exchange, as Drucker describes \citep[p. 176]{drucker_graphesis_2014}. In line with Drucker's advocacy for interfaces that support interpretation rather than displaying finished forms, TCA Workspace prioritizes the activity of interpretation and the richness of process \citep[p. 178-179]{drucker_graphesis_2014}. 
\index[terms]{TCA Workspace}
\index[people]{Drucker, Johanna}

\subsection{Conclusion to the TCA Workspace discussion}

TCA Workspace is an approach to bibliometric analysis that aligns with Drucker's vision of humanistic information design, focusing on dynamic, relational scholarly activity and interpretative processes. I believe current open source PKM hyperlinked writing platforms like Obsidian are most of the way there to Drucker’s “book of the future”, which will “call to the vast repositories of knowledge, images, interpretation, and interactive platforms.” \citep[p. 175]{drucker_graphesis_2014}. Books, PKM or otherwise, will continue to be “an interface, a richly networked portal, organized along lines of inquiry in which primary source materials, secondary interpretations, witnesses and evidence, are all available, incorporated, made accessible for use” \citep[p. 175]{drucker_graphesis_2014}. 
Current PKM “books of the future” can be especially productive when using Luhmann's Zettelkasten method \citep{luhmann_kommunikation_1981,cevolini_forgetting_2016,ahrens_how_2017}\footnote{For the list of tools I used in my knowledge management system, including its Zettelkasten tools like Obsidian, see Appendix 11.} in conjunction with reference managers like Zotero.
\index[terms]{TCA Workspace} 
\index[people]{Drucker, Johanna}
\index[people]{Luhmann, Niklas}


TCA Workspace, however, takes us beyond the book of the future using Topological Capta Analysis in the Computational Analysis of Texts and Graphs (CATG). A hyperlinked decentralization of information can help us all access perspective analogous to Hegel’s, extolled by Derrida as “the last philosopher of the book, and the first thinker of writing” \citep[p. 26]{derrida_grammatology_1997}. My thesis is about developing interfaces that shifts emphasis from from the book toward the wider decentralized rhizome \citep{deleuze_thousand_2007}\footnote{Characteristics of the rhizome are articulated in Deleuze and Guattari’s \textit{A thousand plateaus} “A rhizome has no beginning or end; it is always in the middle, between things, interbeing, \textit{intermezzo}. The tree is filiation, but the rhizome is alliance, uniquely alliance” \citep[p. 25]{deleuze_thousand_2007}.} of interconnected writing and graphing. Ultimately, TCA Workspace aims to facilitate a more dynamic relationship between data, visuospatialization, and interpretation. 
\index[terms]{TCA Workspace} 
\index[people]{Drucker, Johanna}
\index[people]{Deleuze, Gilles}
\index[people]{Guattari, Félix}
\index[terms]{interbeing}
\index[terms]{rhizome} 