%\begin{savequote}[75mm]
%Nulla facilisi. In vel sem. Morbi id urna in diam dignissim feugiat. %Proin molestie tortor eu velit. Aliquam erat volutpat. Nullam ultrices, diam tempus vulputate egestas, eros pede varius leo.
%\qauthor{Quoteauthor Lastname}
%\end{savequote}
%methodological prototype
%Experiments in Information visualization as anticipatory designs of visuospatial topic modelling, made as a means of climate crisis mitigation and accelerating interdiscipilinary research as a whole.
%I demonstrate my methodological prototype's function through interface mockups, i.e. Ch 5
\chapter{Introduction}

We are in a crisis of understanding. The magnitude and complexity of the climate crisis and its many related crises demand that we change the tools and methods we use to think. This thesis proposes a novel framework for activating large texts, particularly in the critical domain of climate resilience. In this thesis, I propose a response to the climate crisis that builds on the ways we reason with symbols and graphs as a means to accelerate interdisciplinary knowledge work. 
\index[terms]{climate resilience}

The climate crisis is accelerating, which imposes a shrinking timeline for innovating ways to face it. Furthermore, while we may need to create new tools and methods, we may already have solutions for climate crisis mitigation that lie inactive in the silos of research repositories and their marginalia.

My thesis positions itself at the intersection of urgent global need and emerging technological possibility. At a moment in history when time saved in research could mean the survival of innumerable humans and other forms of life, the development of better ways to work with knowledge could hardly be more urgent. 

\section{Background and context}

\subsection{Research Problems and Significance}

Our crisis of understanding demands new solutions, and fast. However, the growing amount of information makes it impossible to consider all data. As Drucker notes, ``The expansion of access to any and all stored data that can be repurposed and remediated nearly boggles the mind” \citep[p. 194]{drucker_graphesis_2014}.
\index[people]{Drucker, Johanna}

Working through all available information in every discipline to find climate resilience solutions is likely impossible or would require an unsustainable amount of time and resources. The endeavour of ``developing the ability to cope with much larger amounts of information" \citep[p. 34]{sevaldson_designing_2022} can and should include developing tools that increase the efficiency of managing information complexity and facilitate conceptual translation between disciplines. Knowledge Activation, or activating connections between disciplines using computationally assisted methods of analyzing text and graphs, has the earth-changing potential to reveal solutions to global issues like the climate crisis. Paradigm-shifting solutions for climate resilience almost certainly exist already, and it is critical that we endeavour to find and apply them. Some new knowledge production will inevitably be necessary to minimize harm during Sustainability Transitions, but we may only need to activate the knowledge we already have.
\index[people]{Sevaldson, Birger}
\index[terms]{climate resilience}

\subsection{Foundations of key ideas}

\subsubsection{Isomorphism}
%Alternative thesis title: Data isomorphism and the climate crisis
Computational tools can accelerate climate resilience knowledge work in many disciplines. However, the magnitude of addressing global climate requires integrating disciplinary approaches for solutions that have yet to be imagined. In addition to being curious researchers, we must proactively interpret the shape of knowledge and identify latent solutions that fit key-in-lock into challenges of our primary disciplines, from within and outside our areas of expertise. By developing systems that help interpret the shape of information, we can more quickly find `keys' with the form we need for a given `lock'.

In my analogy of keys, this sameness of form is also called isomorphism as coined by chemist Eilhard Mitscherlich in the early 19th century
\citetext{\citealp[p. 239]{Berzelius1821Loethrohr}; \citealp{the_editors_of_encyclopaedia_britannica_eilhardt_2024}; \citealp [p. 427-437]{Mitscherlich1820Abhandlungen}; \citealp[p. 239]{Mitscherlich1837Lehrbuch1}}. The word isomorph is composed of the Greek words \textit{ἴσος} (\textit{ísos}, ‘equal’) and \textit{μορφή} (\textit{morphḗ}, ‘form’) \citetext{%
  \citealp[s.v.~\textit{ἴσος}]{liddell__1996-5}; %
  \citealp[s.v.~\textit{μορφή}]{liddell__1996-6}%
}.
\index[people]{Mitscherlich, Eilhard}
\index[terms]{isomorph}
\index[terms]{isomorphism}



\subsubsection{Data}
 To continue development of tools for data isomorphism, and further co-inform ``engineering capability with an imaginative sensibility" \cite[p. 195]{drucker_graphesis_2014} in `\textit{humanistic} forms of knowledge production" \citep[p. 10]{drucker_graphesis_2014}, we must contend with the ontological and epistemological implications of approaching knowledge with word \textit{data,} informed by its etymology.

The conventional use of the English word data traces to Latin \textit{datum} (‘something given, granted’) \citep{oxford_english_dictionary_datum_2024}, which de-emphasizes the interpretive agency of the receiver. By contrast, \textit{capta} (‘things taken’) derives from Latin \textit{capere} (‘to take, receive, acquire’) \citep{ashdowne_capere_2013}, shifting agency toward the receiver’s choice in working with knowledge.
\index[terms]{capta}
\index[terms]{data}

Drucker defines capta as ``a systematic expression of information understood as constructed, as phenomena perceived according to principles of observer-dependent interpretation" \citep[p. 131]{drucker_graphesis_2014}. Drucker adds that by ``qualifying any metric as a factor of some condition, the character of the ``information” shifts from self-evident ``fact” to constructed interpretation motivated by a human agenda" \citep[p. 131]{drucker_graphesis_2014}.

In my thesis, I apply the term \textit{capta} in key anchor terms. For example, to refer to Topological \textit{Data} Analysis (TDA), I propose the term Topological \textit{Capta} Analysis (TCA) instead. Although I generally agree with using the term \textit{capta} and its emphasis on interpretative knowledge construction, I deemed it impractical in writing this thesis to adapt each quotation that uses the term data, due to the broad range of discussions and sources I chose. For this reason, most quotations which use the word data are included without renaming or additional comment. Nevertheless, while I value the disambiguation of data and capta, my experience of knowledge involves a combination of both how I am changed by the `given-ness' events (data) \textit{and} my active interpretation of those events (capta).


\index[terms]{Topological Data Analysis (TDA)}
\index[terms]{Topological Capta Analysis (TCA)}
\index[terms]{capta}
\index[terms]{data}
%\index[terms]{humanistic forms of knowledge production}

\subsection{Research Questions}

Working within this background and context, I ask the following primary research question and seven secondary research questions:

\begin{enumerate}
    \item[\textbf{RQ1}] \textit{What philosophical, mathematical, and computational approaches to textual/graphical analysis can accelerate knowledge work?}
    \item[\textbf{RQ1.1}] \textit{What forms of spatial information visualization are there, and how can they inform the composition of network graphs?}
    \item[\textbf{RQ1.2}] \textit{How can Computational Analysis of Texts and Graphs (CATG) identify isomorphic semantic structures within large network graphs?}
    \item[\textbf{RQ1.3}] \textit{What methods can reveal the relationships between fundamental ideas in texts, graphs, or Large Language Models (LLMs)?}
    \item[\textbf{RQ1.4}] \textit{Given the semantic versatility of symbols, how can a practice of collaborative symbol-making support knowledge production?}
    \item[\textbf{RQ1.5}] \textit{How can CATG reveal relationships between place and text?}
    \item[\textbf{RQ1.6}] \textit{What strategies can improve university knowledge management to accelerate research?}
    \item[\textbf{RQ1.7}] \textit{What sort of knowledge work software would I want to build to incorporate my various findings and use spatial information visualization to identify isomorphic semantic structures, reveal the relationships between fundamental ideas, build on collaborative symbol-making, reveal relationships between place and text, and improve university knowledge management?}
\end{enumerate}


\subsection{Thesis Statement and Contributions}

The climate crisis continues to accelerate faster than climate resilience research and operationalization can keep up. Novel approaches for Knowledge Activation can catalyze paradigmatic shifts in Sustainability Transitions (ST). The seven contributions of this thesis are a set of such novel approaches proposed for visuospatial Knowledge Activation (KA) using human-in-the-loop (HITL) Computational Analysis of Texts and Graphs (CATG), Human Analysis of Texts and Graphs (HATG), or both. These seven contributions would be used for the surfacing, synthesis, translation, and production of knowledge:
\index[terms]{climate resilience}

\begin{enumerate}
    \item[\textbf{C1}] \textit{Semantic Forms}, a taxonomy of three-dimensional topic model compositions for HITL CATG, HATG, or both.
    \item[\textbf{C2}] \textit{Query Isomorphs} as a means of Topological Capta Analysis (TCA) in HITL CATG using small graph chunks.
    \item[\textbf{C3}] \textit{Ontological Semantic Network Summaries (OSNS)} as a means of revealing ontological relationships between ideas in a given body of research using HITL CATG, HATG, or both.
    \item[\textbf{C4}] \textit{Symbol-setting}, a method for expanding the semiotic range of knowledge production using symbol co-creation in HITL CATG, HATG, or both.
    \item[\textbf{C5}] \textit{Terroir of Text and Graphs (TTG)}, a method of HITL CATG that uses TCA to interpret and reveal semantic relationships between (a) texts and graphs, and (b) the features and systems of ecological place.
    \item[\textbf{C6}] \textit{TCA Researcher Grouping}, a proposal to use TCA for grouping research collaborators more effectively using HITL CATG, HATG, or both; finally,
    \item[\textbf{C7}] \textit{TCA Workspace}, a methodological prototype of a collaborative platform for HITL CATG and HATG to:
    \begin{enumerate}
        \item[(a)] integrate all my thesis contributions (\textit{Semantic Forms}, \textit{Query Isomorphs}, \textit{OSNS}, \textit{Symbol-setting}, \textit{TTG}, and \textit{TCA Researcher Grouping}).
        \item[(b)] facilitate computational interface with Systemic Design methods for visuospatial reasoning encountered through my literature review, such as gigamapping 
        \citetext{\citealp{sevaldson_giga-mapping_2011}; \citealp[p. 26]{sevaldson_designing_2022}} and Systematic Combining \citetext{\citealp[p.554]{dubois_systematic_2002}, \citealp{kjode_entanglement_2024}}.
    \end{enumerate}
\end{enumerate}
\index[people]{Gadde, Lars-Erik}
\index[people]{Dubois, Anna}
\index[terms]{gigamapping}
\index[terms]{Systematic Combining (SC)}
        

The core of this thesis is a set of two method proposals for working with graphs of text, Semantic Forms and Query Isomorphs, integrated into one methodological prototype: TCA Workspace. Developing the TCA Workspace software platform has the potential to accelerate the surfacing and interpretation of interdisciplinary climate resilience knowledge using the computational analysis of texts and graphs. TCA Workspace would integrate Semantic Forms for the three-dimensional modelling of text graphs, Query Isomorphs for Topological Capta Analysis, and interdisciplinary symbolic co-creation as Symbol-setting. Furthermore, TCA Workspace would work co-benefit existing Systemic Design methods for managing complexity including gigamapping \citetext{\citealp{sevaldson_giga-mapping_2011}; \citealp[p. 26]{sevaldson_designing_2022}}, and Systematic Combining \citetext{\citealp{dubois_systematic_2002}; \citealp{kjode_entanglement_2024}}. Epistemologically, TCA Workspace would function as a particle accelerator of ideas, where researchers would work with texts and graphs as activated webs of thought.
\index[terms]{gigamapping}
\index[terms]{Systematic Combining (SC)}
\index[people]{Gadde, Lars-Erik}
\index[people]{Dubois, Anna}
\index[terms]{climate resilience}

\FloatBarrier
\begin{figure}[ht]
    \centering
    \includegraphics[width=\textwidth]{figures/5.16.HT1.png}
    \caption[Preview of TCA Workspace illustrations]{\textbf {Preview of TCA Workspace illustrations.} Horn Torus Semantic Form about Query Isomorph \textit{i}. Render of how TCA Workspace would render a Query Isomorph about this thesis located in a Horn Torus Semantic Form network graph.}
    \label{f5.16.HT1}
\end{figure}



\FloatBarrier
\clearpage



\subsection{My prior work}

My prior work in education technology and theological academia frames the epistemological diversity that motivates and informs this thesis.

Building on my research in EdTech UX design, branding, and creative direction, my approach expanded to include the use of design tools for managing operations. For example, I developed and implemented organizational resource maps to minimize lag in sharing multimedia assets within and between departments. I then expanded these resource management systems so they could also maps of organizations' written resources, like official company copy and positioning research. By visualizing the relationships of information within organizations I activated the knowledge that already existed in company assets, empowering all departments independently, while working better as a whole. 

In parallel, my engagement with theological research and art-making, particularly within the Harvard Divinity School Program for the Evolution of Spirituality art and spirituality working group, centred the themes of power and climate grief. In my exhibition \textit{Song Within a Sacrifice Zone}, and the artist talk and interview that followed, I grappled with the grief and disorienting overwhelm of the global warming crisis \citep{1496711,castano-suarez_biopower_2023,castano-suarez_song_2023-1}. Furthermore, in my presentation on biopower and the pastorate, and the subsequent panel discussion, I addressed abuses of power and the anti-oppressive potential of emerging information management techniques \citep{1496651,castano-suarez_biopower_2023,castano-suarez_biopower_2024}.

Ultimately, my prior work in design leadership pointed me to the need for more robust methods to organize information across groups of people who think and work differently from each other. This thesis addresses the innovation of methods more directly, representing an expansion of both the scale and semantic scope of my earlier work in organizational information mapping practice, all oriented in service of anti-oppressive climate resilience.
\index[terms]{climate resilience}
\index[terms]{climate justice}



\section{How this thesis is organized}

The overall structure of my thesis is as follows: In the following chapters, this thesis will move from a literature review to secondary methodologies and methods, preliminary work, visuospatial models derived from theory, methodological and theoretical proposals derived from visuospatial models, a reflection on the literature review, a discussion of my contributions, and a conclusive call to action. 

In Chapter 1, I introduced my thesis's background and context, research problems and significance, foundations of key ideas, research questions, thesis statement, contributions, and prior work.

In Chapter 2, Literature review, I present my mixed-methods scoping review. I frame the review with information about the climate crisis and various cognitive and social challenges it represents. I then present ways that approaches from design and computer science can help manage complexity. I then critique the terms ``blind-spots," ``black box AI," and ``stakeholders." I end my identifying the gaps I observed in my literature review.

In Chapter 3, Secondary methodologies and methods, I present how I worked within the Research-Creation and Critical Systems Methodologies. Here, I present my methods, sampling strategy, and analytical approaches, including Systematic Combining and Meta-Systematic Combining. 

In Chapter 4, Preliminary making, I present the art and design thought experiments that catalyzed this thesis: the composition of horn torus spatial information visualization. Next, I present my code and information analysis experiments in Personal Knowledge Management, topic models, and Small Language Models. 

In Chapter 5, From theory and method to visuospatial models and back, I present the ways Semantic Forms, Query Isomorphs, and how they work together. Next, I present TCA Workspace. Last I present the theoretical and methodological contributions I arrived at through making visuospatial models: Ontological Semantic Network Summaries (OSNS), Symbol-setting, Terroir of Text and Graphs, and TCA Researcher Grouping. 

In Chapter 6, Discussion of contributions, I retrace my research questions and present how my contributions respond to each.In this section, I also present the humanist post-structuralist design and multi-mathematical qualities of Semantic Forms, Query Isomorphs, OSNS, Symbol-setting, TTG, TCA Researcher, and the TCA Workspace platform overall.

In Chapter 7, Conclusion, I retrace the core problems I faced with research questions, present concluding syntheses, and offer a call to action.