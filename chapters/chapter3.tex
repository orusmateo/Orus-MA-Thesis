%\begin{savequote}[75mm]
%Nulla facilisi. In vel sem. Morbi id urna in diam dignissim feugiat. Proin molestie tortor eu velit. Aliquam erat volutpat. Nullam ultrices, diam tempus vulputate egestas, eros pede varius leo.
%\qauthor{Quoteauthor Lastname}
%\end{savequote}
\chapter{Secondary methodologies and methods}
\section{Research-creation methodology}

I used research-creation as a secondary methodology in this thesis to support my literature review. Canada’s Social Sciences and Humanities Research Council defines research-creation as ``an approach to research that combines creative and academic research practices, and supports the development of knowledge and innovation through artistic expression, scholarly investigation, and experimentation” \citep{government_of_canada_research-creation_2021}. My art and design “creation process is situated within the research activity and produces critically informed work in a variety of media (art forms)” \citep{government_of_canada_research-creation_2021}.
\noindent\index[terms]{Research-creation}

Research-creation is most often associated with artistic production in the context of the university, but, as Natalie Loveless asserts, ``its real potential rests in its demand for an inter-­or transdisciplinary perspective that, while marshalling the insights of emerging and developing fine arts research methodologies, exceeds the fine arts proper. 11" \citep[p. 6-7]{loveless_how_2019-1}. The research-creation methodology supports my mixed methods by allowing for the combination of art, design, and research in ways that challenge conventional separations of knowledge. Research-creation provides a framework for exploring and developing new forms of visual epistemology and diagrammatic reasoning while contributing to my work's transdisciplinary nature.
\index[terms]{visual epistemology} and \index[terms]{diagrammatic reasoning}
\noindent\index[terms]{Research-creation}

By working interdisciplinarily with research-creation, I risk ``the revelation of incompetence possible when the skills of one discipline prove insufficient in another context" \citep[p. 45]{loveless_how_2019-1}. Yet, I do not think that not working at the boundary of ignorance is an option in any form of research. My ignorance of topology and topic modelling at the outset of my research can be an asset. My thesis is also a user case-study of folks from similar disciplinary backgrounds to mine for developing knowledge work platforms. As Shoshona Felman asserts, the ``truly revolutionary insight—the truly revolutionary pedagogy discovered by Freud—consists in showing the ways in which ignorance itself can teach us something, become itself instructive" \citetext{\citealp[p. 79]{felman_jacques_1987}; \citealp[p. 38]{loveless_how_2019-1}}. As a Brand/UX designer and photographer, the novelty of the fields I encountered by research-creating this thesis adds an additional layer of value to this document.
\noindent\index[terms]{Research-creation}
\index[terms]{topology}

I propose that the origin of perspective in art before being codified into mathematical terms in DeCartes’ rectangular coordinate system, as retraced by Panofsky \citep[p. 57-58]{panofsky_perspective_1991}, is an example of research-creation. If the art of perspective can precede the math of perspective, the convergence of various disciplines through research-creation is of immense value and urgency in the context of sustainability transitions and other global problems that require interdisciplinary convergence. Loveless vibrantly frames research-creation itself as a matter of perspective, asserting that it ``mobilizes forms that anamorphically shatter single-point perspective, failing to cohere fully into art or scholarship, instead nurturing driven curiosity as its lure and guide: the desires articulating the eruptions of the drive(s) that animate each of us in unpredictable, but nonetheless accountable, directions. It demands the production of new, unruly, driven stories within the university as not only a bastion of privilege, but a site of intense transformative pedagogical power" \citep[p. 105]{loveless_how_2019-1}. Despite the divisive misinformation about climate urgency being `a matter of perspective’, we may, in fact, lean on the history of perspective itself as an example of joining efforts across disciplines.
\index[people]{Panofsky, Erwin}
\noindent\index[terms]{Research-creation}

\subsection{Chapman and Sawchuk}
My work embodies the principles of research-creation as defined by Chapman and Sawchuk \citep{chapman_research-creation_2012}, integrating creative practice with academic research to generate new knowledge. My research aligns with three out of their four main categories: research-for-creation, research-from-creation, creative presentations of research, and creation-as-research. However, this thesis works primarily on research-creation as research-for-creation and research-from-creation.
\noindent\index[terms]{Research-creation}

First, research-for-creation has been a foundational aspect of my work. My literature review of geometry, topology, information visualization, and computational text analysis provided the theoretical underpinnings necessary to develop Semantic Forms and Query Isomorphs. My exploration of mathematical concepts like Topological Data Analysis (TDA) and Persistent Homology informed the production of new visualization methods. Philosophical frameworks from Kant and Foucault, along with existing visualization techniques like Bertin's taxonomy of network graphs \citep[p. 52, 270]{bertin_semiology_2011}, inspired me to innovate forms such as the horn torus Semantic Form.
\noindent\index[terms]{Semantic Forms} \index[terms]{Query Isomorphs} \index[terms]{network graph} \index[terms]{Horn Torus Semantic Form}
\index[people]{Bertin, Jacques}
\index[people]{Foucault, Michel}
\index[people]{Kant, Immanuel}
\index[terms]{torus}
\index[terms]{topology}
\index[terms]{Topological Data Analysis (TDA)}

Second, in research-from-creation, the act of devising network graph composition forms led to theoretical insights for me. Making models of Semantic Forms and Query Isomorphs revealed to me how semantic forces influence the spatial organization of ideas in text graphs and how these forces might be operationalized through algorithms and interfaces. The creation of Ontological Semantic Network Summaries (OSNS) allowed me to theorize about visualizing ontological positions in texts as a way for researchers to choose sources and databases for their work better. Similarly, Terroir of Text and Graphs (TTG) emerged as a theoretical product of my making process, leading me to new perspectives on Knowledge Translation based on how ecological context and language influence each other.
\noindent\index[terms]{network graph} \index[terms]{Semantic Forms} \index[terms]{Query Isomorphs} \index[terms]{Ontological Semantic Network Summaries (OSNS)} \index[terms]{Terroir of Text and Graphs (TTG)}
\index[terms]{Knowledge Translation (KT)}
\index[terms]{composition}
\index[terms]{Ontological Semantic Network Summaries (OSNS)}

Overall, my thesis seeks to advance theoretical knowledge through creative practice, generate new insights from the act of creation, communicate research findings creatively, and position creation itself as a form of research. By integrating creation and research, I offer a set of contributions to both academic scholarship and creative practice, emphasizing the value of research-creation in addressing complex global challenges like the climate crisis.
\noindent\index[terms]{Research-creation}



\section{Critical Systems Thinking Methodology}
I used the Critical Systems Thinking Methodology in parallel with research-creation to support my literature review. Considering the interdisciplinary and multi-technological nature of design and information in addressing complex global challenges like the climate crisis, my thesis employs a diverse set of methods in alignment with Critical Systems Thinking (CST). CST, as proposed by Midgley, emphasizes two dimensions of criticality: power relations and methodological pluralism \citep[p. 143]{sevaldson_designing_2022}. My thesis aligns with both by proposing computational tools that can be used to resist the abuse of power \citetext{\citealp{1496711}; \citealp{castano-suarez_song_2023-1}; \citealp{1496651}; \citealp{castano-suarez_biopower_2024}} and by complementing systems analysis with research-creation and mixed methods literature review.
\index[terms]{Critical Systems Thinking (CST)}

\subsection{Pluralism of methodology and method}
My research seeks to embrace methodological pluralism by integrating mixed qualitative and quantitative methods, as discussed in my mixed methods literature review. 

My methods were also plural and complementary and aligned with CST's emphasis on non-reductionist investigation of connections and relations \citetext{\citealp{midgley_systemic_2000}; \citealp[p. 144]{sevaldson_designing_2022}}. By combining philosophical, computational, and multi-mathematical approaches, my research seeks to build methods for uncovering insights to better catalyze climate resilience.
\noindent\index[terms]{Critical Systems Thinking (CST)}
\index[terms]{climate resilience}

In brief, my thesis uses CST as a methodological framework for interdisciplinary research by addressing power relations and embracing pluralism of method and methodology, supporting justice-oriented methods for solutions to global challenges.
\noindent\index[terms]{Critical Systems Thinking (CST)}

\section{Research design for mixed methods}
I approached my work with mixed methods as a response to the diverse needs of the climate crisis. In their editorial \textit{Evidence synthesis for accelerated learning on climate solutions Berrang‐Ford et. al} write: “In order to comprehensively learn from the available evidence on climate solutions, it is, therefore, vital to mirror the methodological diversity in primary evidence by promoting the application of the full breadth of qualitative and mixed methods evidence synthesis methodologies” \citep[p. 3]{berrangford_editorial_2020}. 
\index[terms]{evidence synthesis}
\index[people]{Berrang-Ford, Lea}
\index[terms]{Knowledge Synthesis (KSy)}

My mixed methods seek to be integrative, in keeping with the mission to align otherwise diversified approaches. The separation of knowledge\slash\-s is exactly what I am working against through methods of integration. For this reason, my thesis moves between phases of literature review, contextual review, making, and reflection. 

I categorize my thesis as a work of anticipatory design instead of using more sense-exclusionary terms like ``design foresight." My ongoing efforts to use inclusive language extend across disciplines. My thesis is interdisciplinary in that I am informing disciplines with each other. Further, my work is also transdisciplinary in that the subject matter, tools, and anticipatory design move the study of art and design into math and computer science.
\section{Methods}
The following methods in my research were specifically qualitative. First, I surveyed symbols and information visualizations by collecting and cataloguing photographs, screenshots, and digital copies into categories. I categorized this selected content based on the specimen’s originality and complexity. For information visualizations, I categorized specimens by composition type. 
\index[terms]{composition}

In doing so, I analyzed information visualizations for their geometric compositions. I increased their dimensionality from two to three to be more capable of analyzing larger topic model network graphs. The increase in dimension allows for a more fulsome utilization of the neuroscience of the visuospatial as per Tversky’s research, but also provides a practical function in network graphs since larger graphs have vectors that tend to cross over each other, obfuscating the clarity necessary for pattern-finding. Beyond geometric analysis, I conducted a literature review of topology to better capture the means of analyzing network graph representations in my models. Such means include Topological Data Analysis (TDA) using Persistence Homology (PH). In my analysis of practices that seek to manage complexity, I arrived at the field of Systemic Design and its toolkit of designerly approaches 
\citep{jones_synthesis_2016,kjode_entanglement_2024,sevaldson_giga-mapping_2011,sevaldson_designing_2022}.
\index[terms]{network graph} 
\index[terms]{Persistence Homology (PH)}
\index[people]{Tversky, Barbara}
\index[terms]{Systemic Design}
\index[people]{Sevaldson, Birger}
\index[people]{Kjøde, Svein Gunnar}
\index[terms]{topology}
\index[terms]{Topological Data Analysis (TDA)}



\section{Sampling strategy}
I collected figures according to their originality of format, complexity represented in the visualization, and the symbolic quality of the composition. In other words, I considered whether the composition of the visualization could read as its own stand-alone unit. After a more general sampling according to the categories above, I selected a smaller number of the most diverse information visualizations based on their composition. 
\index[terms]{composition}

\section{Analytical approaches}
Overall, I surveyed information visualization compositions and the semantic affordances of their basic geometries using inductive, deductive, and both approaches. I produced a taxonomy of two-dimensional shapes and three-dimensional forms that act as morphological categories for most information visualizations, whether network graphs or otherwise. 
\index[terms]{network graph}


\subsection{Composition analysis of information visualizations, graphs, and symbols.}
I analyzed information visualization figures, network graph and otherwise, through inductive, deductive and abductive approaches. 
\index[terms]{network graph} \index[terms]{abduction}
\index[terms]{composition}

\subsubsection{Inductive, bottom-up analysis of deriving categories from specific examples}
%\noindent \textbf{Inductive, bottom-up analysis of specific examples to categories} \\
I derived observations from a collection of figures by categorizing them as images on one large virtual canvas, namely Affinity Designer. This inductive approach allowed me to narrow down images for more detailed consideration by dividing them into categories. 

In addition to the conventional text interpretation in a literature review, this deductive analytical approach to a sampling of figures involved contextualization and interpretation. A benefit of this approach to sampling symbols and information visualizations was identifying and interpreting figures that I considered highly unique. Examples of this interpretive approach follow: first, the pivotal example in my research was interpreting the Sri Yantra alongside its three-dimensional counterpart, the Meru Chakra \citep[p. 31]{buhnemann_mandalas_2003}, as image, symbol, and graph; second, was the interpretation of a network graph `whole’ as a unit made of smaller graphlets, which was foundational to arriving at Topological Capta Analysis (TCA) as a means of activating texts by tracking network graphlet isomorphologies; third, as a diversification of network graph `wholes’, texts like Manuel Lima’s classification of information visualizations as trees or circles \citep{lima_book_2014,lima_book_2017} lent themselves as corroboration for the value of a geometric approach to information visualization composition analysis, network graph or otherwise. 
\index[terms]{Sri Yantra} \index[terms]{Meru Chakra}
\index[terms]{composition}
\index[terms]{Topological Capta Analysis (TCA)}

\subsubsection{Deductive, top-down derivation of specific examples from larger categories}
%\noindent \textbf{Deductive top-down observation of categories to specific examples} \\
I analyzed top-down observation of composition forms in taxonomies of information visualization. Consulting the work of subject-matter experts who categorize information visualizations was of central importance. Notably, Anna Vital’s infographic \textit{How to think visually} \citep{vital_how_2018} acted as both an introduction and touchpoint to my work of examining the range of visually epistemological options available for information visualizations.
\index[terms]{composition} 

\subsubsection{Algorithmically-assisted sampling that was both top-down and bottom-up}
In my third approach for sampling figures, using Pinterest was inductive and deductive by feeding into its image-recommendation algorithm. My inductive process was the selecting and classifying images from the wide range of samples initially populated by Pinterest, while it had limited information about my image collecting. By saving certain types of images from certain categories, I then deductively co-created the parameters for new sets of images, which starts the process over again. As a member of the OCAD U Digital Futures graduate program, which has trained many game designers and theorists, this algorithmic feedback loop added a ludic serendipity to part of my research practice.
\index[terms]{abduction}

\subsection{Meta-systematic combining}
In this thesis, composition is both object of study and method. I accomplish this through an expansion of Systematic Combining (SC). 
\index[terms]{Systematic Combining (SC)} 
\index[terms]{composition}

As discussed previously, Kjøde employs the valuable SC method in his work to visualize the entanglement of systems in Sustainability Transitions (ST) for several reasons: 
\begin{enumerate}
    \item SC facilitates simultaneous evolution of “theoretical framework, empirical fieldwork, and case analysis” \citep[p. 554]{dubois_systematic_2002}.
\item SC emphasizes a ``continuous movement between an empirical world and a model world” \citep[p. 554]{dubois_systematic_2002} using ``abductive logic” \citep[p. 553]{dubois_systematic_2002}.
\index[terms]{abduction}
\index[people]{Gadde, Lars-Erik}
\index[people]{Dubois, Anna}
\index[people]{Kjøde, Svein Gunnar}

\item SC prioritizes discovering new research dimensions over data verification, combining frameworks from multiple sources to reveal unknown aspects \citep[p. 552]{dubois_systematic_2002}.
\index[people]{Gadde, Lars-Erik}
\index[people]{Dubois, Anna}

\item SC is suitable for ``rapidly developing, interdisciplinary fields” that become ``a system of dynamic knowledge”  \citep[p. 46]{kjode_entanglement_2024}. It can handle interrelated or contradicting ideas \citetext{\citealp[p. 46]{kjode_entanglement_2024}; \citealp[p. 8]{sevaldson_discussions_2010-1}}, making it appropriate for the ``complex, interconnected and emergent nature of the sustainability transitions field” \citep[p. 45]{kjode_entanglement_2024}.

\end{enumerate}
\index[terms]{Sustainability Transitions} \index[terms]{abduction}
\index[people]{Sevaldson, Birger}
\index[people]{Kjøde, Svein Gunnar}



In appreciating the value of SC, I propose an expansion of this method in an effort to develop new methods to bolster its theoretical impact on Knowledge Activation. I propose that meta-systematic combining (MSC) is a geometric and topological analysis and combination of composition types which can work with a computational analysis of texts and graphs (CATG). 
\index[terms]{Knowledge Activation (KA)} 
\index[terms]{composition}

Due to the same pressing urgency of complexity management, my expansion of the SC approach integrates not only the graphical representation of systems but also the analysis and combination of their compositions. My aim in this thesis is ``designing the designing” \citep[p. 185]{pangaro_design_2011} of complexity management operationalized as the development of new platforms for the composition-informed Computational Analysis of Texts and Graphs (CATG). 
\index[people]{Pangaro, Paul}

\section{Ethical considerations}
The use of technology in this thesis does not preclude the rapid and radical reduction of technology use as a means to return to more traditional land-based living; in fact, I actively work towards it in my life as someone in the midst of re-encountering my own Indigenous ancestry. I strive to benefit from colonially extractive technology as little as possible and to spend time honouring the land as much as possible. It is to this end that I position my investigation as a search for increased efficacy of information systems and the equitably sustainable research that they support. Considering the ecological cost of AI, I limit LLM use to lower-carbon options like local AI models and text-only outputs whenever possible. 
\index[terms]{Large Language Model (LLM)}
\index[terms]{Indigeneity}

While my work is distinctly visual and builds on affordances of visual processing, which benefits some kinds of accessibility, I recognize that focusing only on visual accessibility is insufficient. In a later section, I critique some of the ableist language used by sources in the literature and contextual review. 

In my appreciation for Sevaldson’s ethics of Systemic Design and SOD, I turn to key moments in his Designing Complexity (2022) that capture values which also guide my work. Designers must consider the ethical consequences of their work. We must always work within sustainability parameters to limit, or hopefully stop, the ways we contribute to and accelerate over-consumption. It is ethically impossible to be a designer today without balancing the production of commercial advantage and climate impact. The geometric value of a circular composition heuristic notwithstanding, designing with a holistic circular economy in mind implies ``that the designer takes responsibility for the whole process, all material systems, the life cycle and the recycling” \citep[p. 36]{sevaldson_designing_2022}. Every person in a given project must keep in mind the people `not in the room’, the ``people who are deprived of expressing their interests, like children, seniors with dementia, or refugees; it could also be future generations, other species, or people who are affected by the effects that are only visible to the expert” \citep[p. 97]{sevaldson_designing_2022}.
\index[terms]{Systemic Design}
\index[people]{Sevaldson, Birger}

Design can be wielded to be catastrophically destructive. Sevaldson warns that doing the ``wrong thing in an excellent way results in great devastation” \citep[p. 87]{sevaldson_designing_2022}. To illustrate this point, Sevaldson offers the famous example of Nazi art. Nazi art can be argued to be bad design or low quality art, but its political efficiency is evident \citep[p. 87]{sevaldson_designing_2022}. 
\index[people]{Sevaldson, Birger}

The value of Systemic Design tools notwithstanding, the balance of ideals and lived experience demands important choices from us. As Sevaldson writes, ``understanding the social systems open [sic.] up a way to involve and engage that might bridge the gap between desire and sustainability, between refinement and solidarity, between individual needs and the social, as well as between doing the right thing and making profits. The interesting thing is that this way may open up new possibilities. By recreating and reconnecting these contradictions, new ways of acting within and changing the social system might appear” \citep[p. 42-43]{sevaldson_designing_2022}. 
\index[terms]{Systemic Design}
\index[people]{Sevaldson, Birger}

Overall, I seek to activate knowledge and texts for innovation, and for the ways KA can power anti-oppression, social justice, eco-justice, and environmental sustainability. Sevaldson notes that Béla Heinrich Bánáthy ``is known for placing an empty chair in the middle of conversations, representing future generation” \citep[p. 97]{sevaldson_designing_2022}. In the spirit of keeping space open for the voices of the `other,’ I invite responses to my work that help me do this better. 
\index[people]{Sevaldson, Birger} \index[people]{Bánáthy, Béla Heinrich}