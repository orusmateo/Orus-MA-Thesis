Questions to expect


%In conclusion, I propose that we urgently develop both the spatial network graph representations of large text corpora and their hypergraph representations as a way of mitigating the harms of global warming.


\textbf{Copy from Document
}



We are in a crisis of understanding. The magnitude and complexity of the climate crisis and its many related crises demand that we change the tools and methods we use to think. This thesis proposes a novel framework for activating large texts, particularly in the critical domain of climate resilience. In this thesis, I propose a response to the climate crisis that builds on the ways we reason with symbols and graphs as a means to accelerate interdisciplinary knowledge work. 

The climate crisis is accelerating which imposes a shrinking timeline for innovating ways to face it. Furthermore, while we may need to create new tools and methods, we may already have solutions for climate crisis mitigation that lie unactivated in the silos of research repositories and their marginalia.

My thesis positions itself at this intersection of urgent global need, and emerging technological possibility. In a time in history when time saved in research could mean the survival of innumerable humans and other forms of life, the development of better ways to work with knowledge could hardly be more urgent. 





Epistemologically, TCA Workspace functions as a sort of particle accelerator of ideas, where researchers can examine and activate not books, but living webs of thought.
// tension between Semantic Forms and Query Isomorphs




To address this fundamental information crisis, we must establish not only new knowledge but also activate existing knowledge from various disciplines. Solutions to the climate crisis may already exist but remain hidden in the overwhelming amount of information available. To rethink how we integrate knowledge, we must take an interdisciplinary approach.




In my research, I was unable to find studies, methods or platforms that propose or accomplish the following tasks simultaneously:
\begin{enumerate}
    \item[-] Distinguishing the various modalities of knowledge and ranks them by their expense, including distinct definitions between modes of Knowledge Activation, including Surfacing, Synthesis, Translation, and Production. The limited subtlety in this subject makes climate resilience research more costly than it needs to be. It is imperative to distinguish and quantify the expense of various modalities of Knowledge Activation by their expense to reduce unnecessary costs in climate resilience research.
\item[-] Distinguishing three-dimensional forms by the various ways they capture and reveal reasoning and other semantic relationships as layouts for text graphs and graph chunks. The limited research in this area restrains exploration of Visuospatial Knowledge Activation in various fields, including neuroscience, learning psychology, and design. Visuospatial models facilitate understanding because they capture and reveal reasoning and semantic relationships in overall text graph composition and graph chunks.
\item[-] Distinguishing how capturing and revealing semantic relationships in text graphs can be distinctly non-computational with the \textit{option} of being computational. In other words, I am pursuing solutions that do not impose but rather offer the option for computational analysis of texts and graphs (CATG), which keep humans in the loop (HITL). The limited distinction and combination of computational machine-driven and human-driven ‘manual’ methods in the visuospatial modes of Knowledge Production (a) hinders the use of valuable training data for LLMs that can support climate resilience research (b) demonstrates that the practical applications of existing visuospatial modes of Knowledge Production lack a strong connection to research findings that indicate its value, and (c) limits the ways valuable technology can be used by climate resilience researchers. We must distinguish human-driven and machine-driven modes of VKA to take better advantage of each mode and the effectiveness of their combined use.  
\item[-] Deployment of contemporary tools for the non-computational Human Analysis of Texts and Graphs (HATG) that relies on manual placement of words and visual elements in three-dimensional space (e.g. Systemic Design, Systems Oriented Design, Systematic Combining, Boundary Critique). The limited options in this area prevent us from using the full range of our human ability to reason in visuospatial modes of Knowledge Activation and the optical opportunities beyond the restrictions of two-dimensional information representation. We must build better tools for non-computational HATG using manual placement in three-dimensional space to more fully apply the creative potential of human VKA beyond two-dimensional limits.
\item[-] Methods that use HATG to train algorithms or LLMs. We must drive research in this area to increase the impact of LLMs in climate resilience research and overall research. 
\end{enumerate}










\section{Climate crisis: what is at stake}
Despite the misinformation about climate change, there is a nearly complete consensus that human causes are to blame \citep{Cook_2016,lynas_greater_2021}. According to Pörtner et al., none “of the 20 2011–2020 Aichi biodiversity targets and none of the mileposts on climate trajectories intended to limit warming to 1.5°C have been met” \citep[p. 1]{portner_overcoming_2023}. The World Meteorological Organization (WMO) reports that the “past nine years, 2015–2023, were the nine warmest years on record” \citep[p. 3]{world_meteorological_organization_state_2024}. Despite this, greenhouse gas emissions continue to increase and “now exceed 55 billion tonnes of carbon dioxide equivalents” per year \citep[p. 1]{portner_overcoming_2023}. Key targets have not been met, and emissions continue to rise.






Thus, we have to rethink how humans, our ecosystems, and planetary health are connected. 

//This is not new, and Indigenous wisdom has been an example of this for innumerable years. 




Clinging to the hope that we are yet discussing humanities and not posthumanities, I posit that the climate crisis is a crisis of understanding, institutionally and socially. 











The relevance of information overload to Knowledge Production is not new. Bawden and Robinson note that overload was mentioned by name as a problem at the Royal Society’s Scientific Information Conference in 1948. In it, Maurice Line comments: ``Not for the first time in history, but more acutely than ever before,” scientists feared that the overwhelming flow of information would exceed their ability to manage it, putting “science itself [...] under threat” \citep[p. 183]{bawden_dark_2009}. Information overload and the difficulty of managing the vast amount of potentially relevant material is not a new challenge, but the rate and scale at which new information is produced and made available poses an increasingly serious challenge to researchers at a highly critical time. This complex of problems requires new strategies and new tools.



\noindent \textbf{Coping by using design} \\
Conversely to \textit{overload}, methods exist for \textit{coping} with more complex information visually. Considering the cognitive overload of the climate hyperobject, I turn here to the practice of design. Specifically, Systems Oriented Design (SOD) is a graphing methodology developed by Birger Sevaldson which hones ``the ability to cope with much larger amounts of information” \citep[p. 34]{sevaldson_designing_2022}. Sevaldson asserts that visualization ``is simply the best way to understand complex issues” \citep[p. 34]{sevaldson_designing_2022}. They stress the designer’s ``inherent ability to work with complexity” and their ``advantage in visual reasoning and thinking” \citep[p. 34]{sevaldson_designing_2022}. SOD, thus, uses visualization alongside designers’ innate capacity for visual reasoning to tackle the cognitive challenges posed by complex issues like climate change.
\index[terms]{hyperobject}
\index[terms]{Systems Oriented Design (SOD)}

\noindent \textbf{The visuospatial} \\
Barbara Tversky’s work in the neuroscience of the visuospatial examines the implications of how brains evolved to handle position in space before they evolved the ability to handle language. Neurologically and behaviourally, Tversky has found that ``spatial thinking is the foundation of all thought” and that “the foundation for spatial thought is also the foundation for conceptual thought” \citep{tversky_barbara_2022}. Building off Tversky, it is not an exaggeration to state that our sense of place in space precedes and fundamentally shapes communication, language, and reason.
\index[people]{Tversky, Barbara}




Terry Eagleton summarizes Barthes’s description of the shift in thinking from structuralism to poststructuralism as ``a movement from `work’ to `text’” \citep[p. 120]{eagleton_literary_2006}. Eagleton elaborates: ``It is a shift from seeing the poem or novel as a closed entity, equipped with definite meanings which it is the critic’s task to decipher, to seeing it as irreducibly plural, an endless play of signifiers which can never be finally nailed down to a single centre, essence or meaning” \citep[p. 120]{eagleton_literary_2006}. Herein lies the role of the network graph as a means of expressing decentralized meanings using signifiers like word, point, and line. Drucker points out that the ``shift to expressive metrics and graphics” is crucial in moving from ``\textit{expression of constructed, interpretative information}” to ``\textit{constructed expression of perceived phenomena}”\citep[p. 130]{drucker_graphesis_2014}. Interpretation in this sense involves more than just ``constructedness and inflection” \citep[p. 130]{drucker_graphesis_2014}.


Drucker offers a departure from the term data, which claims to convey fact. In practice, this is a leap of reasoning. Drucker characterizes ``all information as constructed: as expressing the marks of its inflection in some formal way” \citep[p. 130]{drucker_graphesis_2014}. Drucker differentiates \textit{data} from \textit{capta}, which ``is not an expression of idiosyncracy [sic.], emotion, or individual quirks, but a systematic expression of information understood as constructed, as phenomena perceived according to principles of observer-dependent interpretation” \citep[p. 131]{drucker_graphesis_2014}. Drucker goes on to clarify that to ``do this, we need to conceive of every metric ``as a factor of X,” where X is a point of view, agenda, assumption, presumption, or simply a convention” \citep[p. 131]{drucker_graphesis_2014}. Overall, by ``qualifying any metric as a factor of some condition, the character of the ``information” shifts from self-evident ``fact” [, or data,] to constructed interpretation motivated by a human agenda [ie. capta]” \citep[p. 131]{drucker_graphesis_2014}. Drucker challenges the notion of objective data, proposing instead the concept of `capta’ to emphasize that all information is inherently constructed, shaped by observer-dependent interpretation, and should be understood as a product of specific viewpoints rather than as self-evident facts.







// What has worked in surfacing using text and graphs in Sowa: 

Analogy utilizes all four kinds of mapping, while logic uses only the first three kinds. Sowa asserts that these mechanical and neurophysiological mechanisms underlie all kinds of structure mapping \citep[p. 28]{sowa_challenge_2004}.
\index[people]{Sowa, John F.}

To operationalize analogical reasoning in computational systems, Sowa and Arun K. Majumdar developed the \textbf{VivoMind Analogy Engine (VAE)}. VAE employs three structure-mapping methods and conceptual graphs in analogical reasoning to compare and map conceptual structures: (1) \textbf{matching labels} ``compares nodes that have identical labels, labels that are related as subtype and supertype such as Cat and Animal, or labels that have a common supertype such as Cat and Dog” \citep[p. 21]{sowa_analogical_2003}. (2) \textbf{matching subgraphs} ``compares subgraphs with possibly different labels. This match succeeds when two graphs are isomorphic (independent of the labels) or when they can be made isomorphic by combining adjacent nodes” \citep[p. 21-22]{sowa_analogical_2003}. (3) \textbf{matching transformations} ``If the first two methods fail, Method \#3 [matching transformations] searches for transformations that can relate subgraphs of one graph to subgraphs of the other” \citep[p. 22]{sowa_analogical_2003}. This method processes analogies of analogies and which requires more time \citep[p. 29]{sowa_challenge_2004}. VAE is an example of the ways Computational Analysis of Texts and Graphs can support human reasoning via analogy, setting the stage for interdisciplinary computational reasoning.
\index[people]{Sowa, John F.}







\subsubsection{Benefits of visualizing networks in three dimensions}
The mission of conciliating information from various disciplines into computational approaches for Sustainability Transitions requires the management of a high degree of complexity. Network graphs in three or more dimensions can manage more entities and relationships and should be considered as a semiotic framework for complex knowledge work like interdisciplinary and computational Sustainability Transitions research.




Ware and Mitchell report the following about their experiments: ``With 2D viewing and 2D spring layout, a 33-node graphs yielded comparable error rates. Thus, we find roughly an order of magnitude increase in the size of the graph that can be “read” (where we consider “reading” to be the identification of short paths), when 3D viewing is available using stereo and motion depth cues” \citep[p. 10]{ware_visualizing_2008} \footnote{Ware and Mitchell distinguish the number of links as a key factor to consider: ``The degree of correlation we found was surprisingly high and the regression equation suggests that adding 25 more links for a given number of nodes results in an additional 1\% increase in error rate” \citep[p. 13]{ware_visualizing_2008}. This 25:1 ratio can guide the dimensionality reduction of larger high-dimensional graphs for optically legibility and accessibility. Furthermore, Ware and Mitchell go on to discuss a variety of other factors that should be considered when working with three-dimensional network graphs.}





\subsection{Research gap}
My literature and contextual review identified research gaps, which are implementational, practical, and theoretical.

First, I will address the gap I identified in implementation. While three-dimensional modes of information visualization exist, the range of forms utilized seems to be limited in number and implementation complexity. First, I did not observe cone diagrams \citep{taylor_creating_1990,bezold_overview_1993,hancock_possible_1994} that were developed into large network graphs.  Second, I observed cylindrical network graph composition \citep{hinderling_50_2017}, but not applied to categorizing information. Third, spheres, or forms organized around a central origin vertex, have been developed as tools for topic category visualization, but with no options to query for relationships between ideas using a visual graphlet interface \citep{sunter_3d_2023,noauthor_infranodus_2024, weichart_obsidian-3d-graph_2023}. Fourth, I observed the use of some spatial graphlets to visualize text, but I found no option for searching for similar graphlets \citep{ortiz_mind_2024}. I could not find any forms that used tori, series of cones, or complex geometries to reveal semantic relationships. 
\index[people]{Bezold, Clement}
\index[terms]{composition}

Second, there is a gap in practical knowledge where research findings do not seem to be finding practical application. Theoretical questions exist regarding the spatial semiotic representation \citep{saint-martin_semiotics_1990}, interdisciplinary symbolic representations of knowledge \citep{anderson_drawing_2018}, text interface development \citep{drucker_graphesis_2014}, ontological graphs \citep[p. 4]{sowa_knowledge_2000}, interdisciplinary topic mapping \citep{adler_great_1952-1}, conceptual unification of theoretical models \citep{wilson_consilience_1999}, and the visual implementations of Systematic Combining \citep{kjode_entanglement_2024}. While these practices are independently advanced and advancing, it would seem using them in conjunction with three-dimensional topic model network graphs, climatically or otherwise, is not currently used to chart deeper relationships between ideas across academic disciplines. In short, I could not find literature that contends with the climate hyperobject using computational means that leverage human visuospatial reasoning. 
\index[people]{Drucker, Johanna}
\index[people]{Saint-Martin, Fernande}
\index[people]{Kjøde, Svein Gunnar}
\index[people]{Adler, Mortimer J.}
\index[people]{Anderson-Tempini, Gemma}
\index[people]{Sowa, John F.}
\index[terms]{Systematic Combining (SC)}
\index[people]{Wilson, Edward O.}
\index[terms]{consilience}
\index[terms]{hyperobject}


Third, a theoretical and literature gap exists between models and observed phenomena. Bertin proposes three-dimensional information visualization forms \citep[p. 270]{bertin_semiology_2011}, and Saint-Martin proposes a more sophisticated perspective on three-dimensional arrangement \citep{saint-martin_semiotics_1990} yet the range of spatial topic model network graph composition types seem to be narrow, as detailed in the implementation gap. Furthermore, it would seem there is an absence of taxonomies addressing the semantic value of distinct forms for plotting the composition of information visualizations in three-dimensional space while static or changing in movement. Therefore, I propose such a taxonomy, illustrated by examining spatial topic model network graphs. My taxonomy of Semantic Forms is foundational to my proposal for the development of CATG in digital scholarship, Systemic Design, Sustainability Transitions, and other disciplines. 
\index[terms]{Systemic Design}
\index[people]{Bertin, Jacques}
\index[people]{Saint-Martin, Fernande}
\index[terms]{Semantic Forms}
\index[terms]{composition}



Second, in \textbf{research-from-creation}, the act of devising network graph composition forms led to theoretical insights for me. Making models of Semantic Forms and Query Isomorphs revealed to me how semantic forces influence the spatial organization of ideas in text graphs and how these forces might be operationalized through algorithms and interfaces. The creation of Ontological Semantic Network Summaries (OSNS) allowed me to theorize about visualizing ontological positions in texts as a way for researchers to choose sources and databases for their work better. Similarly, Terroir of Text and Graphs (TTG) emerged as a theoretical product of my making process, leading me to new perspectives on Knowledge Translation based on how ecological context and language influence each other.



//Examples of unearthing and Revealing:
- Closest I got was Proof of concept with laudato si infranodus.

\subsubsection{Defining Semantic Forms}
In proposing Semantic Forms, I am not proposing comprehensive technical specifics for how someone will algorithmically build them. I propose what we might do with Semantic Forms while making some technical assumptions about what is algorithmically possible. For example, I choose to make the assumption that the activation of text and graphs using Semantic Forms can involve the parsing of texts, groups of texts, graphs, and groups of graphs into dimensionally versatile graphs like network graphs, topic models, and LLM vector embeddings. I conjecture that rich TCA graphs can be used to train LLMs by parsing visuospatial forms of knowledge production.
\index[terms]{Large Language Model (LLM)} 

The sense of visuospatial affords humans a higher bandwidth for information processing than just two-dimensional representation \citep{tversky_barbara_2022}, which could hold a key for managing the complexity of the climate crisis, or at the very least engaging with it with more agency. This section, however, is about the visuospatial models I made and not about their application to climate. 






"Analogy Engine"

"The VivoMind Analogy Engine (VAE), which was developed by Majumdar, is a high-performance analogy finder that uses conceptual graphs for the knowledge representation. Like SME, structure mapping is used to find analogies. Unlike SME, the VAE algorithms can find analogies in time proportional to (\textit{N} log \textit{N} ), where \textit{N} is the number of nodes in the current knowledge base or context. SME, however, requires time proportional to$N^3$ (Forbus et al. 1995). A later version called MAC/FAC reduced the time by using a search engine to extract the most likely data before using SME to find analogies (Forbus et al. 2002). With its greater speed, VAE can find analogies in the entire WordNet knowledge base in just a few seconds, even though WordNet contains over $10^5$ nodes. For that size, one second with an \textit{(\textit{N} }log\textit{ N )} algorithm would correspond to 30 years with an $N^3$ algorithm." \citep[p. 21[{sowa_analogical_2003}


















































Notes to integrate
Adam: 

// Why it matters is more important than what it is. 

 - urgencies, climate

 - Is it demonstration, illustration, metaphor, reinforcement? 

Is the substance of the paper pointing to things I found? Am I doing the connection work between the things I found, as opposed to letting the viewer look at a pile of found objects? 

 

 If I don't say the thing I want it to say, it means something different to everybody. 

 

 e.g. aim of a good curatorial statement: you don't have to go to the exhibition, but you really want to. 

e.g. Michael Prokopow's book does a good job at showing: 

 - interesting idea in the paintings

 - why the paintings are important

 - something about sitting in front of the object that is different from understanding the story behind them. 

he's not trying to have the *one* version of it, not trying to make it perfect, but trying to not look away. The artist thinks that is important. If you spend time with the work and Michael's writing, you have that urgent and meaningful nugget very clear. 

There is a thing in the middle, the void can be filled, the amount of work needed to look here, rather than to build here is worth it. 

Query Isomorphs are about articulating the space *between*. 

We need to go from empty consilience space, to form, to navigation, to arriving at the piece of information that change the whole thing.

VS

 a Manifesto. Goal is to motivate people to go and build the thing in an empty space. 

 

 Motivation is not people's problem. Question is how. 

 When capitalist power seems immutable, we do have access to information. could reconfiguration of information show us the thing that we imagine are here already, that can be used in the face of this immutable power. 

How do we solve the climate crisis?

we need information, we need a tool?

 what if it already existed?

what if it existed in the intersection of chemical sciece, music, and anthropology. 

there seems to be no good reason why it is not at that junction and something else. 

so we need a way to reconfigure and look at the interstitial areas.

new problem in areas of high consilience and low consilience. easier to move between chemistry and material science, which share a lot of discourse. Chemistry and music, not so much. music and physics, some. 

We need to go in the middle, but we get the angry mob problem. lots of activity and no direction. 

These Semforms etc are a tool. How do we take the corpi and make conections, without everyone explaining their own discipline in a tlinear time-bound. This is that these data maneuvers do. 

In Digital Humanities, they can start to look at the intersections, stregnths and flaws of relational databases. Makes sense that relational database between music and chemical science will have different axes than chem science and amterial science. but we could use the same tool and it should be coherent. 

If I made the tool to maintain the axes and coherence, than I have the first draft technical tool for enabling high data interdisciplinary exchange, that I think has the possibiliity for finding the piece of information that already exists for solving the climate crisis. 

This if giving someone with the motivation to dig through the mountain. 

 If we all dig into it, like the SETI project. Might be me, but statistically not likely. Someone else will do the mining, depending on how we work together, we share the benfits. Motivation is not the problem, it is the tools for individual action. 

Old time-bound tools made sense when experts could share aspects of their disciplines in a two hour panel, but information is changing too fast, and those days are over. 

Slides: 

 footnotes of the citations

First urgency is, the house is burning down. 

 We don't have time to run to home depot, what are we going to put over the fire. And the fire extinguisher has a lock on it, and I only have straw blankets. 

The details, e.g. this Semantic Form taxonomy, are in service of the argument. 

// I know I spent a lot of time talking about xyz, but remember, *this* is what I want to do. 

 

[Beat]

 Exhibition and Presentation Discussions: 

 //choose the order of when to discuss this.

 Remind people that I've shown this in public. The point is not that I can imagine pretty shapes. I have shown this and people have had responses to it. 

 Why did I show it in an art gallery and not xyz? 






 Emphasize the the value to knowledge systems: 

 Consilience, 

 

 reflects older relational ways of knowing.

Developing the valleys between the quadrants as its own unifying accelerating work/language/metadiscipline

Relate to core motivations: new ways of orienting and navigating knowledge, and we need it for climate.

\textbackslash{}\ not super clear in the thesis. 

Expect: 

 Ali:

 How is this not Interdisciplinary Visual Humanities. Didn't you just rehash Interdisciplinary Scholarship (1960s) from a different perspective? 

Greg:

 How does this apply in government, scholarship etc? Didn't you just make it messier rather than easier. 

Adam:

 Why does it look like this? does that matter? If you want to say X, why does it look specifically like this? subtext is about choices made, why don't you explain the unexplained choices you made? And if that is how we get disciplinary tropes, aren't you making disciplinary tropes?

X: 

 Naming things. Why do these things have names. Why these names specifically? Isn't this colonization all over again? In the goal of interdisciplinary exchange, why would I repeat disciplinary norms that I and others have established are problematic? 

 

 Y: 

 Give me one piece of knowledge that has been unearthed? 

 

 Z: There's lots of work here, when does it become worth it? 

A: Who are the users? Librarians, Digital Humanities Scholar.

 

 B: You're imagining people to use this, doesn't it look like AI is going to use this? If we are not going to use it then how can we trust it? 

 // Graph transparency








Things 3: 

PRESENTATION:
- Angle ranges of degree difference in network graphs of two or more dimensions.  
- Cone: Requires one 180deg line
- Cylinder: requires two 180deg line
-  Sphere: requires no 180deg lines, can include many. 
- Ring Torus: requires one 180deg line through hole
- Horn Torus: requires one 180deg line through centre point
- This is about dimensional addition in graphesis, or visual epistemology, into a visuospatial epistemology:
-  for the inclusion of more affordances: capaciousness of interface, 3D graphs which allow "reading" more nodes
- opens the door to recognition of more integrative patterns, 
- For sensorial diversification toward more neuroevolutionary alignment (CITE Tversky)
- For more consilience. Computational and mathematic approaches are already consilience across disciplines in the sciences, are there opportunities for similar consiliences in the humanities, and across both sciences and humanities? 
- Globally connected place-based text graphs (a:Behind the scenes computational HITL CATG-vector graphs for Small Language Models working together with network graph Retreaval Augmented Generation. b: HATG, visuospatial epistemology interface enhancement for current visual forms of knowledge production, including Systemic Design approaches like gigamapping. c: Interfaces that faciliate both HATG and HITL CATG like TCA Workspace which builds on low-dimensional and high-dimensional graphs)
- Critical AI research that considers AGI is has indeed not arrived, and that does not greet AI as a new being, can still be a spiritual work. There are innumerable beings among the work here already, so there is no shortage of mysticism. Furthermore, in the periphery of the mystical and at the intersection of philosophy, neurology, and design, is the morphology of visuospatial epistemology, including the Semantic Forms and Query Isomorphs. 
- Terroir of Text is not necessarily about Indigeneity, but it is at the very least about what it means to be *of an ecological biologically diverse place*.  
- Semantic Forms: as anchor contribution: fleet of visuospatial epistemology types, towards a taxonomy of information visuospatialization. 
- // Path: Ribecca, Vital, Tversky, Meru Chakra, Rucker. How all of these relate to Zettelkasten Obsidian and Digital Scholarship tools. 
- The poor, homeless. Justice and service. 
- Slides: 15 words or fewer
- Slides: repetition to drive the point home
- Slides: key making 


Questions: 
- how close to 180deg is too wide to be a Triangle Semantic Shape?
- What determines the outer edge of a Semantic Form. Averages
- 





CORRECTIONS:
- Remove biodynamic agriculture. Keep permaculture. 
- Presentation: 
- How the semantic forms are geometric and topological: degree difference is used to calculate conicality, sphericallity, etc. difference of degrees between each form. 
- Cote textbook for torus parts 
- Ethnic positioning: 
- visual disability: low vision requiring glasses for any task. Invisible disability 

- Tori: overlay radii to Wisstein from Wolfram






Photos: topology, Science magazine, Litmaps’ mission, 

To borrow an animistic analogue from Ahasiw, and his “Language of Spiders,” the cybernetic mycelium is ‘spidered’ into being. 

 

Particle accelerator of ideas revealed as a cybernetic mycelium is ‘spidered’ into being. 

 

AI-read-loud for editing
Replace adjectives for strong verbs


Spidering: Ahasiw Maskegon-Iskwew 
Reference 

Cite: “the only way out is through”
Let’s remember the buffalo who run towards the storm as a herd to get past it faster. 



Google Docs version of Ch 5
- Write Ch 5 
- Write Conclusion
- Proofreading with AI and  Grammarly 



ADD


Add PDF  Comments
Timeline graph

Interrogate me on each of these images.
- How they support the thesis
- How the lit review informs them
- How they represent a proposal for methods innovation
- Implications for interface design 



TTG: add Two-eyed seeing. https://earth4all.life/the-book/
Tes about expanding dimensions of sorts, this thesis can help 



Black box move to critique. The conflation that blackening is somehow worsening


Tidy- up tasks:
NotebookLM -Podcast. Hyperlink it into Appendices

Horizontal bar graph of terms and sources

Clearer entanglement graph

QR code to send email: 
Add email tracker for thesis leads. 




\textbf{Adam Transcript 
}


It's not you're not quantifying these arguments. No, I'm not. that you're you're trying to establish aationality. yes, and in order for it to be represented in the computer, you have to use some form of quantization. and that that's provisionally arbitrary, yes, as as a as a as a way of reframing rep representing a potentially reframing some information for interpretation by experts. Yes, right? 
And that you're not presenting the quantization as truth, but as as a technique for looking at information through a shift, exactly. Right? So it's not that it's you're not quantify or doing quantization. 
No, I'm not quantifying an idea. No, and I'm not measuring an idea.. Okay. 
Right. So, so this idea that there's that there is some relationality that's there that can be that can either be used or establish a discourse for interchange the same way that math provides discourse for interchange among most of the sciences and soft sciences. Right? 
that use math for expressing ideas of significance, you know, repeatability, right? and the kinds of things that they think are important for establishing ideas and, uh, replicablyability, right? So, so so so that's that's the thing. 
So, when I look at this, I I think one of the things that you could look at that and say, like, well, surely there's something and when you're talking about conciliants of like, oh, okay, there's a way that, uh that these things can share more efficiently. Yes, right? So, when you go from that to the cone, right? 
So that another way you could you could like the way I would work this is is you would say, okay, great. Well, if we imagine these our cones., on their own. these are those are cones. and uh they've been built up through disciplinary practice, right? And if we imagine this as territory, it's the hills are arbitrary. 
Yes. Bec because of just like this is where folks have landed. There isn't a particularly good reason for there to be a valley right between? 
right? The only reason that there's a valley is because there is a lack of activity here. Yes. 
So we could think about these as mountains and valleys. Yes. Or we could think about them as pillars. 
Right. Right? So if you think about these things as pillars, then it then we're not thinking about the middle area as being brought up. to to a level of the mountains, you're you're thinking about the potential of that actually being greater than the sum of it heart, right? 
Like, if that was the if the if the peak of the mountains was maybe the starting point or or provided perspective, then building up and then building beyond should actually should should actually be easier. So then, like, what's the what's the not? So, you're not interested in calling that a discipline. 
You're interested in proposing that there's knowledge there that's useful to many people, uh, and that that that that that knowledge between disciplines is potentially more valuable for uh for use in in the climate processes. Absolutely. Right? 
So that's the way that that's the way I would argue that. and then you could say like, okay, well, isn't that an old idea? And then you can go right, so then if you can go all right, and so this so if we look at like traditional or additional a smallledge systems, they've been talking this way about knowledge for, you know, a long time as long as we've articulated knowledge, right? And sort of like so so in this way, it sort of say like, so this seems so if we can see that from here through this sort of likeureau kind of kind of way, uh uh, is is there does this give us another way of articulating that it's not that the knowledge is important, but it actually it's these relational systems that have been established for hundreds of thousands of years. is really important for moving to the next place in either discovering or utilizing knowledge as we try to relate back to the planet. you know, I I don't know. 
There's such there's something in there about that that's that's kind of interesting. I and I think that's a a direction I I really appreciate, right? To to begin um to begin with this with without a mention of the the previous piece, uh of the um of these two, um I think the argument I think there's a more interesting argument around what this means for what you're saying rather than this is the first this is my first encounter with this. 
It's like, well, that that's cool. I see, yes. But actually, like that in terms of it's an interesting. 
So instead of instead of your first encounter with it being the interesting information, actually, that that's a detail and interesting information is this perspective of of looking at knowledge systems? Okay. right? I I think that's the it's one of the things in the thesis where where it's a it's a little bit tricky, it gets tricky to sort of see this perspective. you know, as you get into the middle chunk of it. 
Right? As to so like I think that together does sort of a nice job of saying. Right, okay, right. 
This is why it's informed in this way. right? It comes from here thinking about this, and then of course it gets into the the our details of and so forth. all this stuff. Okay, that's that's great. 
Um, yeah, but I but I I think that the thing is what's interesting about this you, like, what what are you trying to say and then what's interesting? in terms of goals of the thesisis, right? thinking of the goals of the overall project. and not necessarily, like, what is your research interest today? 
Yeah, that's kind of cool. Right, right? um it's it's kind of it's kind of neat, but uh when you're communicating research, the important thing is, like, what should I care. 
Right. What did what did you find that's interesting? Have you found it yet? 
No. Get back in the library. Yeah. 
Right? So, uh you know, so I think that that's always a kind of tricky thing for researchers is like, you wanna tell people about the cool stuff you found this week, because it's the cool stuff you found this week, but um if you can't tell it if you can't, like, really clearly link it to the meta arc of your research? Yeah. then it's it's uh it's a detailed detail. 
Thank you. All right? I think that that's always the hard thing to uh to to figure out. right? 
Um, yeah, details are gonna be fine, not worried about not not worried about details. Um, I think thinking about what's the what's the real core motivation and the yeah, you know, like the core contributions are cool, but the core contribution is that there's a new way of orienting and navigating knowledge. right? And uh, you know, someone like Ali is gonna ask him a question along the lines of, um you know, how is this not just like, um, uh intisciplinary digital humanities? 
he's he's gonna he his first question is probably gonna be like, well this this sounds you you're saying all the things about interdisciplinary scholarship without saying interdisciplinary scholarship. right? argue, didn't you just rehash what people tried to say in the 60s, but from a different from a different perspective? uh, so so he he'll probably ask something like that as it as his first question. um Greg will ask a question about well, how does this how does this apply? 
Yeah, right? So, you know, like, as a futurist working with a company or a government agency, uh, with all of all of these stakeholders, didn't you just make this messier rather than cleaner? what that's Greg and asked a question basically in that format.. um right? 
And then I'm just a I'm just gonna be a hot mess of of questions. Cool. you know. I'm gonna be a nervous about it, but that's fine. 
We'll we'll we'll get through.. my questions are are are probably are what are my questions gonna be along the lines of like, you know, uh why does it look like this? really, why does it look like this? does that matter of what you wanna say is this, why does it look like this? 
I think I'm probably gonna ask the question that sort of fall. Fair those lines. uh and and the subtext of that is on is I' probably not interested in the way it looks. No, of course yeah. 
Right? is that is that, uh, something along the lines of choices matter, and you made choices, and that um some of them you explain and some of them you don't. So, like, what about what about the unacknowledged choices that you made? 
uh, and doesn't if that's how we get disciplinary tropes, aren't you making new disciplinary tropes? or something? I don't know. 
There'll be they'll either I'm gonna ask something like that, I's gonna ask us something like that. uh right, that's that's what I would expect to happen at some point. Uh somebody's gonna ask you about naming things for sure, right? Why are those things they why why do those things have names? 
Why do those things have those names? and uh, isn't this colonization all over again? Yeah. right? 
And uh you know, in the in the goal of creating interdisciplinary exchange, uh why would you replicate disciplinary disciplinary norms, um you know, that you and other people have established our problematic. You know, and then the give me one piece of knowledge that's been nurhed. Uh, how much how much there's a lot of work here. 
When does it become worth it? Who who are the users of this? Is it is it the disciplinary folks? 
Is it the interdisciplinary folks? Is it the librarians? Is it the digital humanity scholar? 
Is it uh you know, is that this this is great and all, um, you'agining that people are people are going to use this. Doesn't it look like it's gonna be AI that uses this? right? 
And that if AI uses this, how are we ever gonna trust this word?. That'll be route two. They'arking outside in round one because it's not what you're talking about, but someone's gonna bring up a something like that.. um, this is great all, but AI, discuss. 
Uh, uh, I think that that will happen at some point. Yeah. Um Yeah. 
I think you could that I think those are gonna be they're they' will probably be a question I don't anticipate. Of course. But me, that's what I would expect the question That's great. 
All right. But if what I like to I like to think about what are the what are people going to ask, of course, if you prepare answers to all these questions? um, you'll be in good shape. 
Thank you. And then if none of those questions get asked, you know how to answer questions? Love answering questions. 
That's great. be great. Yeah, um I uh I hate to kind of yank in a different direction, but then I wonder if it's useful to include these. uh, and, uh, the reason I ask is is this this sort of an overview of the um these guys um which I can flip through pretty quickly, I think I can probably spend 20 seconds on I think you have to have these. Right. 
I I think if you don't have these, then you're not talking about your work. Yeah, that's this is sort of the corn making, isn't it? Yeah., uh, so so I think I think you have to have it there, and then, uh, and then in thinking about the MA is that you have the theory that is uh supported lightly by the work. 
Right? So this illustrates an aspect of the theory?, right? It's not that the theory supports the making. 
Right? So, uh, so I I would have it there, because it cause it illustrates some of the things that that you're talking about. um and it illustrates, it doesn't necessarily validate. Differentiation. 
Yeah. So, uh yeah, but I I would have that there, because it's it in a whole bunch of reasons. it's it's work that you made. It's relevant to what we're talking about. 
It looks great, right? So, there's lots of reasons to look at it. cool. Yeah. 
I would have all this stuff. uh I I at some point I would have just a horn Taurus because that was sort of a big moment in the like thinking about form that encapsulates a lot of things. Um. like, like Ruckers, like this one? Yeah. 
Or like I I would have something like that in there, and then the the types. Sorry, go on this and this. right? Like, like, this is the thing that that drew you to something. these were your interventions into it. like, this is uh this is this isn't you. 
No, this is not me. So I I I wouldn't I wouldn't that also not me. I know. 
Okay, great, right. Uh, but that's a thing that that you latched onto, and then you started to think about, like, well, if we were to put knowledge in here what's this point and what's this point? Right? 
Um right? So, like you you could imagine that there's relationality in here that you could envision with what we're really interested in. Like, you really saw this, but you were really interested in these, like, edge areas and, like, is this is is is this like, is this Nirvana? 
Is this Nirvana? Right? Like you were you know, and then you were thinking like, is this disciplinary synthesis? 
right? Like like and are those things the same like, is is Nirvana disciplinary synthesis? Right, exactly. 
Right? um I I did envision this more as like a like a point cloud with, uh, moving points. I didn't have the tools to make it. but but you spend a lot of time thinking about like what what is the meaning of that point? 
If we oriented knowledge in a way that it could meet here, what would that mean? Yes, exactly. right? Exactly. 
And that that's also sort of borne out in the in the coffee plant. Right? So you you're you're asking a very similar, not the same question and exploring it in a different way. 
So I think those things are are important to really did that because they were Those were the big questions that, um led to the other thing. That's true. Yeah. 
I'm I hesitate putting the, um the cake in that. um, this this feels a a secondary. uh, to me, I don't think there's enough time. Nope. um, these also feel secondary. Oh, okay, um I, uh, I think this ends up being secondary.. 
Uh, the conciliience piece, I think is important. Um, but I wonder if it's worth showing I wonder if it's showing if it's presented best as just showing this, or also showing these sort of these four proceeded by this. um because if I just see this, I don't understand that it's this. And the whole point of this is that it's both uh flat and not. 
So, do you have to think about what the argument is that you're going to make? Okay. and then think about what images to show. Okay, right? 
So uh yeah. I'm gonna land on at at least showing both of these guys, and, uh, then I will, uh work on the argument. Right. 
But, uh, if I had a minimum, it would be these two. Um. Yeah. 
Yeah, and the and the, like, why it matters is more important than what it is. Yes, yes, right? Right. 
Right. Yes. So, um, speaking to urgencies and uh, the impact that it might, uh that it could have it was felt further. 
Yeah. is it a demonstration and illustration? Is it a metaphor, is it a reinforcement uh you know, it that uh the So, when you've got work like this, one of the one of the problems with, like a lot of scholarship, uh in any field is that it it uh it basically does what I call pointing. Yeah, so you go to a paper and it says, hey guys, I found all this cool stuff. 
Right? And and that's it, that's the substance of the paper. Right. 
And the the person who's presenting the paper, it's self evident to them, why it's important, and they do none of the connection work. Ah, so the audience is left to imagine what's interesting about it. Right? 
And that that's and I'm risking not here. Yeah, right? So, what when you've got all of this stuff you could put it all together and it's interesting. 
Yeah, right? So, if you have a show with with objects and no curatorial statement, it's neat. but it's not it doesn't say the thing that you want it to say. Right. because it's says something different to everybody. 
Right? There's a reason why you put all that stuff together, because you want to walk people through. the thinking. right? So, uh, a good curritorial statement. a really good one, you don't have to go to the exhibition, which you really want to. 
H. Right? Uh, a lot of curatorial statements are written as kind of like advertising copy of like, come, check it out.. 
Right? Well, I don't really care about this. I'm reading Michael Prottow's book. 
Yeah, he's from here. Yeah. So he's got this book on Hervin Anderson. 
He's this like badass, um Caribbean, UK, DS. uh, I kind of like meeting. I's fine. you know, but I I'm not gonna go I'm not gonna go outating exhibition. I just not. 
Uh uh uh, I want to see his painted after reading this book. Right? Because it because I get why they're important. 
I get that there's a really interesting idea about. I get that there's something there's something about sitting in front of the object that's different than understanding the story about it. right? that there's that that he pieces together all of this stuff that they're like, oh, this is why he's important. 
And there's a whole lot of personal stories and there's a whole lot of history and there's there's a lot of descriptions of the materials in the pieces and all all that stuff, but that but there there's this thing about uh that's that's core to it where, like, he keeps painting the same thing over and over again. Yeah. Right? 
And uh the he's not trying to get good at it. He's not trying to have the one version of it. I mean, he's he's trying to not look away. 
Oh, it's so cool, that's so cool, right? Wow. right? So he's trying to not look away. 
Uh, and that is a thing that he that that the artists thinks is super important. right? So that that like he stays with things when other people would leave them, because he thinks that that's important and that somebody comes out of that. Right? 
And it takes a while to kind of under understand that, but that is clear as day once you spent a little bit of time with the work and with Michael's writing.. Right, so so see so that's what you wanted to do in the in the presentation. It's like, why? 
what what's the things that there there's there's this thing in the middle don't know what it is, but it's think we can I think we can establish that there is a thing there that the void can be filled. or it's already filled and it's just under uh, it's under appreciated, and the amount of work that it would take to look here rather than to build here is worth it. Right. And so so the when you're talking about ISOores, you're trying to articulate the space between. 
Exactly, right? And that that's what that's what you want people to understand, right? It's like when we're talking about this space, we need to understand what it's form is. because we need to go we need to go from empty to form to navigation to the piece of information that changes the whole thing. so so that that's why you've got these levels of detail in here, right? 
If I I like, you could just write a manifesto of it and say, introdisciplinaries it will solve the crime crisis. Exactly, right? But it's like, uh well, where's the meaning? 
Where's the visuals? How do we pointing itold? You know? 
I'm not saying there's no roll there, but I just it felt like I would be I don't know, but but if you write a manifesto and you tell people to go there, you don't really tell them where it is. or what to do. Oh, okay, yeah, right? And you don't tell them you don't like okay, go over to that empty space and then build the thing that I envision. right? 
You can get them real motivated to do that, and then you get like, uh you get the rain mosh bit of, uh uh sheep, but um no, Woodstock 99. Yeah, exactly right? It was a lots of people were motivated and it was rapid and that. uh right. 
So the that's not that isn't the right thing. motivation isn't people's problem, right now. People are motivated to work on climate crisis. The question is how and when capitalists, power seems concentrated and immutable. we have access to information. and could a reconfigure it reconfiguration of information, uh show us the things that we imagine are here already that can be used in the face of us in theable power. 
Exactly. That's what you looked at. right? That's the part you don't want to lose. 
No, and and I get it yeah, I could I could have got dererailed and well, I found this first and then I found this thing. and then I found that there.. right? So, so but then but then the thing is that like, if if you sort of set it out as the problem, so, like, how do we how do we solve the climate crisis? We need some information. 
We need we need a we need a tool. and then the next question was, what if it already existed? what if it was, what if it didn't exist in chemical sciences? but it existed at the intersection of chemical science and music and uh anthropology. 
Exactly. And there doesn't seem to be a good reason why it's not at that juncture than somewhere else. So then, what we need is a way of rapidly reconfiguring or like making new maps of territories of these configurations of knowledge bases. that lets us look at the interest of areas. 
Sounds great. Now, when you've got areas of high conciliience with areas of low conciliience, we have no problem. Right? 
So so that like, how do we move between these places become it's easier to move between, you know, chemistry and, uh, and uh, engineering.. Right? Uh, we were about chemistry and material science. did a lot of discourse. right? 
chemistry and music, not so much. physics, music, a little bit more. um, uh, so the like, hey, we should go into the middle there. again is is the angry mob problem? Yeah, right? Lots of people know direction, no ways of orient. not there. 
Here's some tools for once you get there, how do we how do we start to take the the corpi and create aationality? without having to do a bottom up, everybody explained their disciplines to each other in a in a a linear time mouth when. you just get the experts from each discipline to sit down, explain things, then they're dead by the time. They have that exchange. 
This is what big data does for us. I put all this stuff together, and then we can start to we can start to think about orientation, and then we can start to do some of the the uh uh, library sciences work in digital humanities where it looks at these corpi in order to figure out what some of the intersections are, where some of the strengths and flaws of the relational databases are sort of refine them. It makes sense that a relational database between music and chemical science will have different uh ways uh different axes than chemical science and material science. 
But you could use the same tool uh and it should be coherent. Right. so so that if the th so if I have made the tool that allows these corpi to be there and for axes interchange while while retaining coherence, then I have the conceptual and first draft technical tool for enabling high data, uh, uh interdisciplinary exchange, that I think has the possibility of finding the piece of information already exists for solving the five classes. Right. paints the robe, you know? 
Yeah, that's where you're doing all this stuff. Yeah, yeah. I'm curious about the map. curious about the pretty pictures. 
Yeah, who care who cares which database? Wire w why are we here? Why are we here? 
That's. you gave someone who had the motivation a way to dig through the mountain. In one of us, like the SEI project, if we all dig into it, uh it might be you, statistically, it's not gonna be exactly. It's gonna be somebody else. 
Somebody else is gonna, like do the Bitcoin mining that as long as we're all working together, and doing it, one of us pops out a Bitcoin and maybe we share it or whatever, but we're like one of us, you know, gets the cryptographic hash and then finds the signal in the night sky. Yeah, exactly. Right, so motivation isn't the problem. 
It's it's the tools for individual action. I have the old time bound tools made sense when experts could share aspects of their disciplines in a two hour panel, those days are open. something. what you. I think that's what you have to tell them on. 
Right. something something like that. Those are mine. Those are my words. you have a different way of saying this? 
Yes. um I similar. I I uh I wanna bring in, um I wanna bring in the uh the citations in there, like I wanna be able to mention who, um who brought white or what part of uh of this. um I would do that in passing at the talk, in passing. yeah. Because it's in the paper. 
It is. Yeah, right? you've got lots of writing on on the things, so I think if you talk about like the high level thing and just say, right, this is like Sacha or whoever, right? 
Yes. You know, then then I think that's um I think that's all that's required for the talk. you've established that you know the discourse and is in its relationships. Yeah. chopper fine, or whatever, it's all there. 
Oh, or the index or the index? Right? So, so I I would have it as a as a passing note as, you know, look, you have footnot on on each slide when you're talking about this. 
It's like, oh, okay, it's these two citations. It's this, right? Um, so look, you could show a thing and then say, right, these are this is the synthesis of these ideas, and these ideas, right? 
Thank you. Yeah. I think that's the way I think that's the way to do it.. 
It's funny knowing what people want, you know, or uh, I guess, like a proposing that I understand what people want to hear Uh, just like, what will motivate people is is such a is a question that I don't dabble, you know? um personally or at at a right time, so it's uh it's uh it's really cool to hear my uh my paperful through through this kind of my this this lens of oh, people would want to act or it would have connect with them if it is done in this way, it's really cool to hear if that way. Yeah, yeah. 
Thank you. Yeah. uh yeah, I mean, the problem with the format is if you had an hour, it's then you don't need to prep really. You just sort of, right? 
Hour is easy. ten minutes is hard, ten minutes is square. Yeah. It's it's sort of the hardest. 
Right, a one minute talk is kind of it's a different, difficult and easy, put together, but it's kind of easy, because it's just sort of dough, but in ten minutes, so much the mistakes. Wow. It's sort of the worst. 
Yeah, uh, yeah, yeah, if you've got like one to two hours, no worries. Right, yeah. At that left Hitch talking, he talks for an hour. had that at the end of it, he's like, well, I did pretty good. 
I got through 20 percent of the text I intended. Yeah, he's like, that's probably I get through 10 percent of the text I wish to present. So this is pretty I got through a lot of material. 
It's like, oh, that's allive. It's so funny.. Right? 
so funny, okay. Yeah. So this is this is uh this is me relying on your on your judgment here. what what about these folks? 
These are these are sort of these are the set forms with the infoormms, but presented as a spatial gig map. Right. One, two, and three. 
This guy really wanted, because this is more about, um place based, small language models and uh graphs. So it's uh it's about how we uh relate these ideological stos to the way that the interact with uh place based interactions. Okay. 
Uh, so this this feels like a it's it's not the matched to it, and I it this feels more like a a pass forward kind of thing... great, yeah, if I include this, I don't think I need to include. So that that's also kind of a plus. Sorry, let me let me. 
One. two. That um I would show it quickly. Yeah. right? 
Because it cause it's a good synthesis of a bunch of the different things. you're talking about, and then that you've presented? So, um I think it makes more sense how you get there, even if you show it for 10 seconds. Right? 
This is this is a quick mention. Yeah. Yeah, so I I would I would keep it there cause it the so so one of the things that that you want to do is you kind of wanna show people two years of thinking. you know, right? 
Uh, and so if you if you make too big of a hop, then then it's gonna s people aren't gonna be able to make those connections. Oh, like without tracing it through.. Right. 
Uh, the trick is that granularity of the trace. Yes. Yes. 
That's hard. Good luck. the thing that we all struggle with. how much detail did I take you from here to here effectively? Yeah. 
They'll find another question. Sure. So great. 
Yeah. I think if you left that out, then people would people wouldn't know why that last thing was there. Oh, sorry, if I left this out, they would not know why this is there. 
Yeah. Yeah, that' trucks, right? So I think that's and then they would ask about the difference between that and that. 
I I also think, as I if I if I were to be really brutal here, that I that I could just end with these folks, and that this is a nice tie in, like the the sort of narrative arc of getting from the the branches to here, feels like a nice tie be close. So then do that and leave these as, like, bring those as extra slides when someone se these are extra slides. when someone says, where do you think it's going next? like, well, I'm clever at this, and I sort of see like, I'm seeing this and Ext slides. 
Ext slides. stop is it. So it up we sort of have we talked about what's gonna happen in the in the defense? Do you know when it's over? 
Oh, um my understanding is that when I get calls back in, oh, can you know me? oh, I thought to know. Yeah. 
So the the defense is determining whether or not you're working at the master's level. period. One blowp point that's what it is. That's what it is. 
Okay, uh, so, it's when I? No, it's over when they're satisfied. Oh, so they can just keep asking questions. 
They can keep asking questions.. Um, it's hard to note they might be asking questions and yet be totally satisfied. Oh, yeah, yeah. 
Right? Uh, and you kind of tell him the room, like if they're just babbing and playing with you, then there's not then they're satisfied. It is clear when they're vibing and in their notense. 
And they're no, no, I I don't yeah, it's sort of clear. But it but uh, it's clear when they're satisfied when someone says, how do you imagine this being used in the future? Ah, fine. 
Right? So it's like, oh, that was great. Where did we go from here? 
Yeah. Oh, fine. right? That's the leg I have no moreative gift question, like, so, like, I'll the the committee is charged with coming in with questions for point of clarification. 
So the first round of questions is almost always point of clarification. I was reading your paper and I didn't understand this. Could you talk more about this thing, right? 
I didn't understand why this and this, this. it's the first round is gonna be some detail of some throwaway sentence you made that someone raises objection to, or or like, or whatever, right? The it it'll be that. Once they get the answer, at some point people will go like, okay, great, that makes sense, that makes sense, that makes sense. 
But like, what if this was, you know, what if you were Google instead of Google? Okay, right? um you're done. 
Oh, fine. I mean, keep answering the question. it like you you you you can just sort of sit back and go, I passed. It's great, so great. 
Okay, thank you. That's like you keep that in mind. Yeah, you're done when someone says, what's how do you what's the future of this? 
That's great. Or does this have a future? Sure. right? 
Mhm. yeah. Yeah, it's great. Yeah, and then then you know, you still might be an hour away from. 
Uh, about the day. Uh, right, but uh but yeah, that that's that's all that's all the defenses. It's just like you know, talk to me about the thing. 
Is there is there a glaring error or omission in any of this? If no, that's like the first, right? If there's any fires, they talk about the fires. 
Yes? Okay. That's great. cause I definitely want to know that first. 
Great. Oh, yeah, yeah. Yeah, yeah. the the first question is is almost always the like, this is really great work, and, you know, I like how you did this, but. yeah, d do you know about this or have you done this, or it seems like there's this thing that you're not talking about.. that's usually the the first question.. 
Yeah. and then and then, uh and then the second person is like, mm I was wondering something similar, but actually, I had the similar concern over here. Yes, right? that short. 
Yeah, and there's lots of space, there's lots of space. yeah, yeah, you did yourself no favors. I did not.. It was just fun. 
That's that's what you that's what you do. but uh but yeah, that's what's gonna happen. It's gonna it's gonna be like that. So, um and and to a certain extent, the first two to three questions? 
Yes. will be almost entirely unrelated to your presentation. Like Oh, yeah, people will come in for questions.. Yeah. 
So the presentation is really helpful. Yeah. It almost doesn't matter. 
Sure, in a way. You can so it matters in the sense that you can get it really wrong. Oh, yeah. but uh, it's it's never the what will be wrong is it will draw people's attention to a part of the project. 
Yeah. that isn't the interesting part for you. I' right. Right? 
So so, like, your job is to get people, like, right. the house is burning down. Exactly. That's what we're talking about. 
It's. We're not talking about whether or not these shapes are possible, or if these shapes exist within disciplinary discourse, the house is burning down, we don't have time to run to Home Depot. What are we gonna put over the fire? 
Yeah, and I I left that fire extinguisher laps. Or, uh, somebody put a lock on it, and then I pay the fee. Yeah. 
Right. So, so now what do I do? I've got all these straw blankets and it's on fire. 
Yep. So, uh uh yeah, that's right? So that that's that that's what that's the thing with the difference.. 
That's really funny.. right. So it's it just like you you wanna draw people's attention back to like what what's really important here.. and it's not the details that's something else. The details are in service already tough, but uh of the of the core argument. um, but sometimes it makes a lot of sense to spend a lot of time on one detail. because it's a cause it turns out to be pillar rather than a wall hanging? 
Yeah, right? But, uh yeah, for the most part, you wanna you you wanna think you you want the discourse to go in another direction.. So that's what the presentation is meant to be, is like remember I know I know I spent a lot of time talking about this, remember. this is what we wanna do. 
Like, uh, you might you might want to remind people that you've shown this in public. right? The whole point of this isn't that, you know, I I can imagine pretty shapes. It's that I show them to other people and they have responses to them. 


\textbf{Google Gemini: }
How can we make the climate crisis better

how can we use information better to find solution to the climate crisis

how can we make a difference with so much information out there. 


Knowledge activation techniques
visual and spatial methods

if you have felt like you're drowning
swimming through information but thirsting for knowledge

moving from raw data to actual wisdom

Strategy and efficiency for how we use information

Visual tools to make connections clear so we can see the big picture

\textbf{Semantic Forms} as a way to visualize complex topics using three dimensional shapes

Seeing key ideas clustered in particular ways, seeing relationships, seeing the map of the argument

can be research papers, impact of climate policies, 

useful for policy makers, or anyone 


\textbf{Query Isomorphs
}e.g. seeing impact of deforestation in local communities. Seeing all the studies that specifically connect to social or economic outcomes. 

beyond keyword searches. 

like a filter that lets you see the exact connection you are interested in, even in a giant ton of data


Seeing new connections between disciplines

Whole new way of doing research


\textbf{Ontological Semantic Network Summaries (OSNS) 
}
creating a family tree for ideas
maps out key concepts within a field
shows how they are all connected
how they evolved
key thinkers
see the lineage of an idea
how it has been influenced by other ideas
roadmap of a field
e.g. tracing the history of climate modeling, and how different approaches have emerged, have been debated, etc. 

How do we go about creating these visual representations esp with such complex topics? 

\textbf{Symbol-setting
}Scientists, artists, Indigenous knowledge holders to create symbols to communicate complex ideas in a more universal engaging way.

power of symbols: powerful emotions, complex meaning, 
huge potential for cross-cultural understanding and collaboration. 
essential for finding gloabl solutions.

visual communication to bridge diff ways of knowing and build a more unified movement of change. 


\textbf{Terroir of Text and graphs (TTG)
}
How our surroundings shape our thinking and language. 
even our written language, and how we visually represent information can be influenced by our environment. 
For more locally relevant, culturally sensitive climate solutions.

Different understanding of space and movement
deeper cultural and even cognitive differences based on place.

Recognizing these different ways of knowing can help recognize moving away from one size fits all solutions that often do not work in specific areas. 

Involving more and different groups together including Indigenous knowledge keepers. 

Bridging western science and traditional ecological knowledge. 

Symbols used in education, policy discussion, art, activism: build awareness and move into action. 

Creating a whole new visual language when it comes to climate solutions. 


Using AI to do some heavy lifting: 
\textbf{TCA Researcher grouping}
collaborations, teams for a mission
combinations of skills
matchmaking service for researchers

breaking down silos between disciplines, and connecting people



Building such a system:
\textbf{TCA Workspace}
More than a giant virtual lab for climate research
brings multiple tools and approaches together
Global library of knowledge for climate change. 
finding connections they might not have seen

connecting, sharing, co-creating solutions, 



\textbf{Always challenges when you are trying to push the boundaries of what is possible.}

Next steps we can take today



\textbf{How interconnected the approaches are
about creating an ecosystem of knowledge activation 
that can support us in finding solutions
recognizing that we need multiple perspectives, diverse ways of knowing, willingness to experiment if we are going to build solutions for the climate crisis. }


Visualizing complex research
AI connecting researchers
moving beyond feeling of being overwhelemed, empowering ourselves for a solution

Democratizing knowledge
accessible and usable for everyone, not select few experts
Against Siloing of knowledge
Climate crisis requires global response and breaking down silos

Shifting from passively consuming information 
to actively engaging with knowledge. 


Everyone has something valuable to contribute:

\textbf{Emphasis on limitations of conventional University research methods:
too slow
too siloed
too reliant on linear thinking
to effectively tackle climate.
like trying to solve a multidimensional puzzle with a one-dimensional approach.
}

Seeing connections we might miss
accelerate pace of discovery
break down disciplinary barriers that hinder progress

\textbf{More dynamic knowledge ecosystem. 
}
Using Semantic Forms to teach 
Query Isomorphs


\textbf{Ethical use of tools
}bias 
transparency
prioritize wellbeing and environmental sustainability



knowledge is not just something we consume, it is something we create, share, and activate, to shape a better future. 

The future isn't something that happens to us: it is something we create. Knowledge Activation is an approach we can use to build a more sustainable and just world. 



\textbf{GPT answers to consider
}
\subsubsection{\textbf{6. Curatorial Impact}}

\begin{itemize}
    \item \textbf{Making the Viewer Care}: The thesis draws inspiration from curatorial principles, aiming to compel stakeholders to explore the knowledge tools and frameworks presented, much like an engaging exhibition.
    \item \textbf{Providing a Clear Narrative}: It avoids leaving viewers with "a pile of found objects" by constructing a clear narrative around the tools’ purpose and the urgency of their application.
\end{itemize}
