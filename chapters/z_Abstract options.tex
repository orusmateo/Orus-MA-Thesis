

\textbf{Leonardo}




\textbf{From Studio MTO}

In an era where information overload impedes our capacity to address the escalating climate crisis, we are in a crisis of understanding. This thesis introduces novel methods of amplifying human visuospatial ability to analyze text and graphs by using human-in-the-loop computational methods. I propose the WorkSpace platform, which integrates several original contributions, including: semforms (three-dimensional topic model compositions), infomorphs (topological data analysis using graph chunks), and ontological summary graphing (revealing ontological positioning using semantic networks). WorkSpace builds on existing Systemic Design methods to reveal relationships between academic fields by activating text into knowledge across four modes: knowledge surfacing, synthesis, translation, and creation. WorkSpace is a new mode of knowledge activation that fosters the rapid interdisciplinary collaboration essential to addressing wicked problems like global warming.





\textbf{From Thesis}
The escalating severity of the climate crisis, combined with information overload, hinders our ability to respond as interdisciplinary researchers. In this thesis, I propose anticipatory designs for a novel platform aiming to reveal information we already know in ways that facilitate dialogue between academic disciplines.

I introduce original contributions, including Semantic Forms (3D text-graph compositions), Query Isomorphs (dimensionally versatile graphlets), Ontological Semantic Network Summaries (to reveal ontological frameworks), and the Terroir of Text and Graphs (topological analysis of the relationship between knowledge artifacts and ecological place).

By modeling Query Isomorphs across Semantic Forms, Systemic Design methods, and Language Model vector graph renders, I demonstrate opportunities to use the versatility of visuospatial reasoning in conjunction with Topological Data Analysis for information complexity management. The tools I present in this thesis comprise a platform that would accelerate the synthesis of existing knowledge and extend the applicability of that knowledge across disciplines.



\textbf{Keywords}

\noindent visual epistemology, anticipatory design, Systemic Design, Sustainability Transitions, interdisciplinarity, transdisciplinarity, consilience, isomorphology, evidence synthesis, humanistic information design, Topological Data Analysis, Systems Oriented Design, gigamapping, Systematic Combination, Language Models (LM), Large Language Models (LLM), Small Language Models (SLM), Artificial Intelligence (AI)


