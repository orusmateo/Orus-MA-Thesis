Document: 

Title: 
What may be known: methods for activating large texts and graphs in the climate crisis

Subject: 
Digital Futures MA thesis on systemic design, visuospatial reasoning, and human–AI methods for analyzing large text and graph corpora in the context of the climate crisis.

Author: 
Orus Mateo Castaño-Suárez

Keywords: visual epistemology, anticipatory design, Systemic Design, Sustainability Transitions, interdisciplinarity, transdisciplinarity, consilience, isomorphology, evidence synthesis, humanistic information design, Topological Data Analysis, Systems Oriented Design, gigamapping, Systematic Combination, Language Models (LM), Large Language Models (LLM), Small Language Models (SLM), Artificial Intelligence (AI)







\textbf{Figures and Alt Text}

*Text corrected


Semantic Form Illustrations: 

Query Isomorph-> Query Isomorphs


-> Semantic Form about Query Isomorph i and Sample s
- in the text
- in caption titles. 

starting p 126 and perhaps before. 






1/70: OCAD U Logo
Arrangement of three black squares and black text that reads "OCAD U" and "OCAD University". This is the logo of the Ontario College of Art and Design University. 


2/70
\textbf{*1.1 Preview of TCA Workspace illustrations ............................. 6 }
This is an information visualization of a three-dimensional horn torus network graph of variously hued nodes suspended in dark grey virtual space.

A three-dimensional anticipatory design visualization of TCA Workspace, a platform that would organize key terms from this thesis in virtual space according to the terms' semantic relationships.



3/70
\textbf{*2.1 Conceptual graphs in the lattice of theories ........................... 23 }
Information visualization using black text, arrows and shapes on a white background. 

This is a 2D illustration of how word types can be represented as various kinds of conceptual graphs, which in turn represent various semantic 'intersections' in Sowa's "lattice of theories".



4/70
\textbf{*2.2 “Four operators for navigating the lattice of theories” . . . . . . . . . . . . . . . . . . . . . . 24 }
This 2D illustration uses black text, arrows, and shapes on a white background to show how analogy helps navigate Sowa's 'lattice of theories.' It offers a zoomed-in view of the 'lattice of theories,' by focusing on a set of four lattice line intersections organized as diamond-shaped grid segments represented as black lines with four small black dots where lines intersect. Four operators are at play in this visualization: contraction, expansion, revision, and analogy.






5/70
\textbf{*2.3 Model-Theoretic Semantics ................................... 25 }
This 2D information visualization illustrates Sowa's understanding of how the World, the Model, and Theory derive and denote different outputs. On a white background, it depicts the relationships between the World (represented by an image of planet Earth on the left), the Model (shown as a lattice of theories), and the Theory (depicted as lines of model-theoretic semantics notation). This illustration shows that Approximation is derived from the World and the Model, as assessments of the model’s quality as "Good," "Fair," or "Poor." This illustration also shows that, conversely, Denotation is derived from the Model and the Theory, resulting in assessments of the theory’s validity as "True" or "False."





6/70
\textbf{*2.4 Graphlet configuration examples ................................. 28 }
This 2D information visualization on a white background illustrates examples of graphlet configurations, shown as both one-dimensional network lines and two-dimensional network shapes. The image displays a representative sample of configurations, specifically 3-node, 4-node, and 5-node graphlets.




7/70
\textbf{*2.5 The Sri Yantra .......................................... 33 }
This 2D illustration on a white background is of the Sri Yantra, an important example of Vedic sacred geometry. It features a complex arrangement of various shapes made with curved and rectilinear black lines. 


8/70
\textbf{*2.6 The Meru Chakra, side view ................................... 33 }
This photograph on a black background shows the Meru Chakra, a three-dimensional representation of the Sri Yantra. This golden cone-like form, stepped in multiple tiers, is shown along with a blurred reflection of itself below.


9/70
\textbf{*2.7 The Meru Chakra, top and side views .............................. 33 }
This illustration presents two photographed views of the Meru Chakra: a top view and a side view. The top view highlights how the Meru Chakra is an extruded version of the Sri Yantra. The side view emphasizes its mountain-like conical form.



10/70
\textbf{*2.8 “0-,1-,2-, and 3-simplex from left to right” ........................... 35 }
This 2D illustration on a white background displays four simplices from topology, arranged from left to right. The 0-simplex is shown as a single black dot. The 1-simplex is shown as a line connecting two black dots. The 2-simplex is shown as a grey triangle formed by three lines and three dots. Finally, the 3-simplex is a three-sided pyramid rendered with grey triangles and a dotted line to indicate its hidden back bottom edge.





11/70
\textbf{*2.9 Sowa query graph within the larger database inference system . . . . . . . . . . . . . . . . . . 44 }
This information visualization on a white background illustrates Sowa’s database inference system. The diagram shows a cartoon figure using a computer, with an arrow pointing from itself to a query graph. This query graph, along with representations of Conceptual Schemata and Type Definitions, all point towards a central, glowing circle labelled “Schematic Joins.” From this central circle, an arrow points to an example of an original query graph that has been expanded with background knowledge.


12/70
\textbf{*2.10 The Tree of Porphyry ....................................... 46 }
This information visualization on a white background illustrates John F. Sowa's version of the Tree of Porphyry, originally translated from Peter of Spain (1239). This diagram uses black lines and text to depict categories of inherited traits in Aristotle's ontology as a hierarchical structure.



13/70
\textbf{*2.11 Consilience across four disciplinary boundaries . . . . . . . . . . . . . . . . . . . . . . . . . 50 }
This 2D information visualization on a white background illustrates Wilson's concept of consilience across four disciplinary boundaries. The diagram uses black lines to divide the space into four quadrants: environmental policy, ethics, social science, and biology. The principle of consilience is illustrated by five concentric circles whose lines intersect with each quadrant line, representing how new frameworks can inform multiple domains.


14/70
\textbf{*2.12 The Semantic Research Assistant (SRA) query-handling workflow . . . . . . . . . . . . . . . 52 }
Flow chart on white background of the Semantic Research Assistant (SRA) query-handling workflow. This figure shows the user journey of how investigators and patients query databases, are understood as fragments, and combined into a logical query. Querying is shown to involve a hierarchy of ontologies represented as a tall cone and several smaller cones, all pointing upward like a mountain or 'meru'. 



15/70
\textbf{*2.13 Detail based on the Lenat et al. SRA query-handling workflow . . . . . . . . . . . . . . . . . 52 }
This information visualization detail on a white background focuses on the cone-like forms within the Semantic Research Assistant (SRA) query-handling workflow. Several smaller cones, each representing a domain model, are arranged around a larger central cone. This central cone unifies the smaller domain models through a middle ontology, depicted as its lower half, and an upper ontology, represented by its upper pointed half.




16/70
\textbf{*2.14 Diagram of the SOD Creative Process Framework . . . . . . . . . . . . . . . . . . . . . . . . 63}
This flowchart on a white background illustrates the Systemic Oriented Design (SOD) Creative Process Framework. It uses black arrows, green ellipses, and yellow ellipses to depict the relationships between creation and validation within the framework.



17/70
\textbf{*2.15 Diagram of a design process with iterations ........................... 63 
}
This information visualization on a white background illustrates a design process with iterations, depicted as a black spiral line moving from the upper left corner towards the center of the image. The spiral is divided using alternating shades of light and dark grey into eight repeating, cyclical phases, radiating outwards from the center like a clock. These phases include: project description, ideas, research, matrix, dinners, sketches/testing, evaluating, and specifications.



18/70
\textbf{*2.16 An example gigamap ....................................... 64 
}
This visually dense information visualization is an example of a gigamap, presented on a blue background with white lines and text. It features nested groups of smaller information visualizations, including Venn diagrams, architectural metaphors of semantic relationships, and systems analysis diagrams.



19/70
\textbf{*2.17 Key to example gigamap ........................................ 64 
}
This illustration serves as a key to the previous gigamap (Figure 2.16). It depicts the gigamap's various semantic regions by using a range of hues, shades, and tints of blue, each labelled using white text.


20/70
\textbf{*2.18 Rucker horn torus model representing “an oscillating universe with circular time” . . . . . . . 68 
}
This visualization on a white background is Rucker's horn torus model of "an oscillating universe with circular time." The horn torus is depicted using black lines and text. It illustrates cycles of time as circles cycling from the centre of the horn torus outward around the torus's tube body; space is represented as a line highlighting the horn torus's outer circumference.


21/70
\textbf{*2.19 Illustration of “signatures of toroidal structure in the activity of a module of grid cells” . . . . . 68 
}
This Information visualization on white background shows the toroidal structure revealed in the activity of grid cells in a study of how rats navigate space. Data and cell activity are represented with gradating patterns in hues of yellow, green, blue, and indigo.


22/70
\textbf{*2.20 The three standard tori ...................................... 68 
}

This geometric illustration on a white background shows the three standard tori: the ring torus, horn torus, and spindle torus. Each type is presented with an outer view, a cutaway, and a cross-section to reveal its internal structure.


23/70
\textbf{*4.1 Anatomy of the ring and horn tori ................................ 88 
}
This illustration depicts the anatomy of the ring and horn tori on white background using black and red lines and dots, and black text. The tori attributes represented are the Major Radius, Minor Radius, Major Diameter, Minor Diameter, Outer Diameter, and Inner Diameter.





24/70
\textbf{*4.2 Visualization of the Horn Torus Semantic Form as a flow of re-creation . . . . . . . . . . . . . 89 
}
This experimental information visualization on a white background uses black, teal and purple lines to depict the Horn Torus Semantic Form as a flow of re-creation. As a purple arrow in this cross-section, this figure illustrates transformation from the 'Fullness of being', represented as the torus's center point, outward to the 'Potential for being', shown as the outermost point of the horn torus's outer circumference. A teal arrow represents the direction back to 'Fullness of being'.


25/70
\textbf{*4.3 Horn torus sculpture as Semantic Form information physicalization of consolation and desolation.... 90 
}
This horn torus reveals the composition of a sculptural information physicalization as a Semantic Form expressing the dynamics of consolation and desolation. 

On the left, another experimental information visualization on a white background uses black, teal and purple lines to depict the Horn Torus Semantic Form. As a purple arrow in this cross-section representing 'Desolation', this figure illustrates transformation from the 'Fullness of sense of being', represented as the torus's center point, outward to the 'Absence of sense of being', shown as the outermost point of the horn torus's outer circumference. A teal arrow representing 'Consolation' indicates the direction back to 'Fullness of sense of being'.

On the right, a photograph of my sculpture Schrödinger's Theology (2022) offers a physical representation of this the consolation-desolation information visualization. The sculpture is a horn torus crafted from Kentucky Coffee Tree branches and unbleached cotton thread, serving as a contemplative information physicalization of the visualization described alongside it.




26/70
\textbf{*4.4 Field of horn tori as point plot heatmap with negative space as a sequence of double cones . . . 91 
}
This experimental information visualization on a white background presents an interface designed to connect large amounts of information using interconnected horn tori in a spatial layout. It depicts a field of Horn Tori Semantic Forms as a point plot heatmap, with the negative space forming a sequence of Double Cone Semantic Forms.



27/70
\textbf{*4.5 Larger field of horn tori as point plot heatmap .......................... 91 
}
This expanded view of the previous experimental information visualization on a white background shows an interface designed for connecting even larger datasets. At this scope, the interconnected horn tori are imperceptibly small, forming a more fine-grained spatial heatmap of information relationships. 



28/70
\textbf{*4.6 Obsidian network graph of my research database and detail . . . . . . . . . . . . . . . . . . . 92 
}
This two-part figure displays an Obsidian network graph of my research database. Both parts are presented as a pale grey network graph on a dark grey background. The left side provides a zoomed-out view of all database nodes and their relationships. The right side offers a zoomed-in, detailed view of the same graph.


29/70
\textbf{*4.7 Topic model of themes shared between two texts.
........................... 94 
}
%InfraNodus topic model relating two texts. 
This screen capture of the InfraNodus platform shows a network graph made of the themes shared between two texts. The network graph, set against a light grey background, displays node groups in various colors, with each color representing a distinct cluster of ideas.



30/70
\textbf{4.8 Missing Concepts listed within the Blind Spots tab in the Text Analytics Panels of InfraNodus . 95 
}
This figure shows two screen captures side by side of black text on a light grey background. Each screen capture is of the InfraNodus module, which lists concepts that are present in one text but not the other. Each missing concept is labelled with a small square, hued to represent which theme it corresponds to from the groups of ideas in Figure 4.7. On the left is a list of the concepts present in the Synthesis Report of the IPCC Sixth Assessment Report but not in Laudato si'. On the right is a list of missing concepts present in Laudato si', but not in the Synthesis Report of the IPCC Sixth Assessment Report. 


31/70
\textbf{*4.9 Conceptual Gateways listed within the Blind Spots tab in the Text Analytics Panels of InfraNodus 95 
}
This figure shows two screen captures side by side of black text on a light grey background. Each screen capture is of the InfraNodus module, which lists concepts that can act as conceptual gateways to extend discourse from one text to the other. Each missing concept is labelled with a small square, hued to represent which theme it corresponds to from the groups of ideas in Figure 4.7. On the left is a list of the concepts present in the Synthesis Report of the IPCC Sixth Assessment Report but not in Laudato si'. On the right is a list of missing concepts present in Laudato si', but not in the Synthesis Report of the IPCC Sixth Assessment Report. 

The use of the term 'blind spots' is the UX convention adopted by InfraNodus. I critique the use of the term 'blind spots', along with a set of other oppressive terms, later in this thesis. 


32/70
\textbf{*4.10 Overview of the \textit{Laudato si’ }graph of concepts .......................... 96 
}
Screen capture of light grey topic model interface with network graph showing idea 'communities' labelled in different bright colours, node proximity, and connected edges. This topic model is of Laudato si' and was made with a Small Language Model. 


33/70
\textbf{*4.11 Detailed view of the \textit{Laudato si’ }graph of concepts ........................ 96 
}
Screen capture of light grey topic model interface with zoomed-in view of network graph showing one idea 'community' labelled with black text, red nodes and edges, varying node size, and varying node proximity. This topic model is of Laudato si' and was made with a Small Language Model. This detailed view shows how a user can select a Node by its ID.






34/70
\textbf{*5.1 Network graphs as Semantic Shapes ............................... 101 
}
List of one-dimensional and two-dimensional network graph shapes drawn as black nodes and black arrow lines on light grey and white backgrounds. From left to right we start with a single example of one-dimensional network graph which resembles the one-dimensional form, the Line. Next we have three examples of two-dimensional network graph configurations: the Rectangle is made of a grid where nodes occupy intersections and arrows are the columns and rows; the Triangle is a tree graph where nodes branch out from a single node to more new nodes and edges, resembling a triangle; the Circle is a single node in the centre with nodes radiating from its middle outward. 

These four 'radialities' represent my geometric analysis of node composition in my survey of information visualizations. I call these my Semantic Shapes.




35/70
\textbf{*5.2 Example of circle and triangle Semantic Shape nested in a larger Circle Semantic Shape network graph..........102 
}
Three-part figure of a white network graph shown on a black background, with an image at the top left, top right, and bottom. On the left is a zoomed-out screen capture of a network graph representing the relationship between notes in the Markdown platform called Obsidian. On the right is a detailed view of the same graph. On the bottom is a more detailed view of the figure at the top right, but it includes an overlay of a large circle and three smaller triangles to identify the Circle and Triangle Semantic Shapes in this Obsidian graph.




36/70
\textbf{*5.3 Semantic Forms derived from the Circle Semantic Shape and other Semantic Forms . . . . . . 105 
}
Grid of seven squares drawn with thin black lines on a white background laid out on two rows—four squares on the top, three on the bottom. Each square is labelled with the first seven letters of the alphabet 'a' through 'g'. Square 'a' shows the circle Semantic Shape. Square 'b' shows how the Semantic Circle gets extruded along the Semantic Line to form the Semantic Cylinder. Square 'c' shows how the Circle and Triangle form the cone. Square 'd' shows how the Sphere is a Semantic Circle on two axes. Square 'e' shows the Cylinder wrapped onto itself with the Circle to become the Ring Torus. Square 'e' shows the Double Cone as two single cones connected base to base. Square 'g' shows the Horn Torus as being derived from and related to the Cylinder, the Circle, and the Sphere.



37/70
\textbf{*5.4 Knowledge graph disc and six Semantic Forms .......................... 105 
}
Grid of seven squares drawn with thin white lines on a dark background laid out on two rows—four squares on the top, three on the bottom. Each square is labelled with the first seven letters of the alphabet, 'a' through 'g'. Square 'a' shows a Logseq knowledge graph as an example of the circle Semantic Shape. Squares 'b' through 'g' inclusive show three-dimensional network graphs with nodes in dark grey and edges labelled in various tints of aqua, green, and purple. Each network graph is shaped like the corresponding Semantic Forms in Figure 5.3. To be doubly clear about which form is which, each Semantic Form is emphasized with a semi-transparent geometric solid container to its corresponding network graph. Square 'b' shows a network graph Cylinder, 'c' shows a Ring Torus, 'd' shows a Cone, 'e' shows a Double Cone, 'f' shows a Sphere, and 'g' shows a Horn Torus. 


38/70
\textbf{*5.5 Illustration colours considered .................................. 113 
}
Swatches of colours and corresponding hex codes in black text listed on a white background. 

This is the list of colours considered for my anticipatory design information visualizations.

Swatches are arranged left to right in one row with five categorical colours listed on the left and ten sequential colours listed on the right.


Some swatches have a bright red letter 'X' over them to indicate that they were not selected for my design system.




39/70
\textbf{*5.6 Illustration colours selected .................................... 113 
}
Swatches of colours and corresponding hex codes in black text, listed on a white background. 

This is the list of colours selected for my anticipatory design information visualizations.

Swatches are arranged left to right in one row with four categorical colours listed on the left and three sequential colours listed on the right.




40/70
\textbf{*5.7 Sample \textit{s }ideas list and Query Isomorph \textit{i }ideas list ........................ 114 
}
Two lists of terms are listed on a dark grey background with white grid lines to indicate a spatial interface. Each term is hued with my chosen sequential figure colours: light yellow for highly summative, yellow-green for mid-summative, and green for low-summative.

The first list, Sample s, is on the left and includes essential terms in my thesis. Each term on both lists is labelled as being one of my thesis contributions, a neologism, or both. 

On the right is a shorter list of terms chosen from Sample s to illustrate the morphological versatility of network graph chunks in Semantic Forms using Query Isomorph i. 


41/70
\textbf{*5.8 Sample \textit{s }ideas on a rectilinear timeline .............................. 115 
}
Sample s terms are listed on a dark grey background with white grid lines to indicate a spatial interface. Each term is hued with my chosen sequential figure colours: light yellow for highly summative, yellow-green for mid-summative, and green for low-summative.

Sample s terms are arranged on a rectilinear timeline of when they first appeared in my research.



42/70
\textbf{*5.9 Sample \textit{s} ideas on a timeline transposed onto a circle. . . . . . . . . . . . . . . . . . . . . . . . 116 
}
Sample s terms pairings are listed on a dark grey background with white grid lines to indicate a spatial interface. Each term is hued with my chosen sequential figure colours: light yellow for highly summative, yellow-green for mid-summative, and green for low-summative.

Sample s terms are arranged on a circle by transposing their placement from the rectilinear timeline. 



43/70
\textbf{*5.10 Query Isomorph \textit{i }in 2D and 3D ................................. 117 
}
Query Isomorph i term-node pairings are arranged in two configurations: one in 2D on the left on a white background using black text and lines, and one in 3D on the right, in which terms are placed across three axes and shown in light green on a dark grey background with white grid lines to indicate a spatial interface.  


44/70
\textbf{*5.11 Query Isomorph \textit{i }on a Circle Semantic Shape Logseq network graph . . . . . . . . . . . . . . 119 
}
Query Isomorph i term-node pairings are shown in light green with white edge lines and arranged across two axes onto a network graph shown as a flat Circle Semantic Shape. This Circle Semantic Shape, overlaid with Query Isomorph i, is placed into a spatial interface render, shown as a dark grey background with white grid lines.  


45/70
\textbf{*5.12 Cylinder Semantic Form about Query Isomorph \textit{i }........................ 121 
}

The Cylinder Semantic Form is shown as a network graph on a dark grey background with white grid lines to indicate a spatial interface. 

Query Isomorph i term-node pairings are connected with white edge lines, and arranged spatially according to the properties of the Cylinder Semantic Form and hued with my chosen sequential figure colours: light yellow for highly summative, yellow-green for mid-summative, and green for low-summative.

The high-summative term Computational Semiosis is shown as light yellow and placed onto a dotted line that crosses along the barrel of the Cylinder Semantic Form. 




46/70
\textbf{*5.13 Ring Torus Semantic Form about Query Isomorph \textit{i} and Sample \textit{s} with rectilinear edges . . . . . . . . . . . 123 
}


The Ring Torus Semantic Form is shown as a network graph on a dark grey background with white grid lines to indicate a spatial interface. 


Sample s term-node pairings are arranged spatially according to the properties of the Ring Torus Semantic Form, and hued with my chosen sequential figure colours: light yellow for highly summative, yellow-green for mid-summative, and green for low-summative. 
 
 Additionally, Query Isomorph i are connected with white rectilinear edge lines, and the high-summative term Computational Semiosis is shown as light yellow, and placed onto a dotted line of the same colour, which crosses along the tube body of the Ring Torus Semantic Form. 




47/70
\textbf{*5.14 Ring Torus Semantic Form about Query Isomorph \textit{i }with arced edges .............. 125 
}

The Ring Torus Semantic Form is shown as a network graph on a dark grey background with white grid lines to indicate a spatial interface. 


Sample s term-node pairings are arranged spatially according to the properties of the Ring Torus Semantic Form, and hued with my chosen sequential figure colours: light yellow for highly summative, yellow-green for mid-summative, and green for low-summative. 
 
 Additionally, Query Isomorph i are connected with white edge lines curved to align with the body of the Ring Torus Semantic Form. The high-summative term Computational Semiosis is shown as light yellow, and placed onto a dotted line of the same colour, which crosses along the tube body of the Ring Torus Semantic Form. 



48/70
\textbf{*5.15 Cone Semantic Form about Query Isomorph \textit{i }......................... 125 
}

The Cone Semantic Form is shown as a network graph on a dark grey background with white grid lines to indicate a spatial interface. 


Query Isomorph i term-node pairings are connected with white rectilinear edge lines, and arranged spatially according to the properties of the Cone Semantic Form. 

Query Isomorph i term-node pairings are hued with my chosen sequential figure colours: light yellow for highly summative, yellow-green for mid-summative, and green for low-summative. 

The high-summative term Computational Semiosis is shown as light yellow and placed onto a dotted line that crosses along the rotational axis of the Cone Semantic Form. 



49/70
\textbf{*5.16 Double Cone Semantic Form about Query Isomorph \textit{i }..................... 126 
}

The Double Cone Semantic Form is shown as a network graph on a dark grey background with white grid lines to indicate a spatial interface. 

Query Isomorph i term-node pairings are connected with white rectilinear edge lines, and arranged spatially according to the properties of the Double Cone Semantic Form. 

Query Isomorph i term-node pairings are hued with categorical colours that indicate their node community: purple for Theology, organge for Semiosis, Magenta for Graph Math, and Teal for 3D Information visualization composition.

The high-summative term Computational Semiosis is shown as light yellow and placed onto a dotted line that crosses along the rotational axis of the Double Cone Semantic Form. 




50/70
\textbf{*5.17 Sphere Semantic Form about Query Isomorph \textit{i }......................... 127 
}


The Sphere Semantic Form is shown as a network graph on a dark grey background with white grid lines to indicate a spatial interface. 

Query Isomorph i term-node pairings are connected with white rectilinear edge lines, and arranged spatially according to the properties of the Sphere Semantic Form. 

Query Isomorph i term-node pairings are hued with my chosen sequential figure colours: light yellow for highly summative, yellow-green for mid-summative, and green for low-summative. The high-summative term Computational Semiosis is shown as light yellow and placed in the centre of the Sphere Semantic Form. 

Three Cone Semantic Forms are nested within this Sphere Semantic Form, all sharing the Sphere's respective highly summative term, Computational Semiosis.




51/70
\textbf{*5.18 Horn Torus Semantic Form about Query Isomorph \textit{i }...................... 129 
}
The Horn Torus Semantic Form is shown as a network graph on a dark grey background with white grid lines to indicate a spatial interface. 

Query Isomorph i term-node pairings are connected with white rectilinear edge lines, and arranged spatially according to the properties of the Horn Torus Semantic Form. 

Query Isomorph i term-node pairings are hued with my chosen sequential figure colours: light yellow for highly summative, yellow-green for mid-summative, and green for low-summative. 

The high-summative term Computational Semiosis is shown as light yellow and placed in the centre of the Horn Torus Semantic Form. 

The conicalness of the horn torus's negative space is highlighted with a semi-translucent overlay.

Additionally, three categorical figures are included as circles which connect various nodes in the Horn Torus Semantic Form: the Semantic Forms arc is labelled in teal, the Query Isomorphs arc is labelled in magenta, and the TCA Platform Arc is labelled in orange.  




52/70
\textbf{*5.19 Horn Torus Semantic Form about Query Isomorph \textit{i }dimensionally reduced to 2D . . . . . . . 131 
}
The Horn Torus Semantic Form is dimensionally reduced into this two-dimensional visualization of arcs resembling sine waves. This figure is shown on a dark grey background with a white horizontal timeline to indicate a two-dimensional interface. 

Query Isomorph i term-node pairings are connected with white rectilinear edge lines, and arranged spatially according to the properties of the dimensionally reduced Horn Torus Semantic Form. 

Query Isomorph i term-node pairings are hued with my chosen sequential figure colours: light yellow for highly summative, yellow-green for mid-summative, and green for low-summative. 

The high-summative term Computational Semiosis is shown as light yellow and placed in the centre of the dimensionally reduced Horn Torus Semantic Form. 

Additionally, four categorical figures are included as circles which connect various nodes in the Horn Torus Semantic Form: the Semantic Forms arc is labelled in teal, the Query Isomorphs arc is labelled in magenta, and the TCA Platform Arc is labelled in orange, and the Theological arc is labelled in purple.


53/70
\textbf{*5.20 Horn Torus Semantic Form gigamap ............................... 134 
}
The Horn Torus Semantic Form is shown as a network graph on a dark grey background, with the spatial interface indicated by panes around the Horn Torus Semantic Form at three angles.

Each pane represents a different two-dimensional reduction of the Horn Torus Semantic Form. 


54/70
\textbf{*5.21 Cone Semantic Form gigamap about Query Isomorph \textit{i }..................... 135 
}
The Cone Semantic Form is shown as a network graph on a dark grey background, with the spatial interface indicated by panes around the Cone Semantic Form at four angles.

Each pane represents a different two-dimensional reduction of the Cone Semantic Form. 



55/70
\textbf{5.22 Spherical example of Spatial Information Visualization Composition (SIVC) in Design for Sustainability Transitions (DfST) ..................................... 137 
}
Information visualization on white background using black lines and text of a large black sphere containing three overlapping spheres resembling a spatial Venn diagram. This figure is based on Kjøde's "Floke programme and quadruple helix for stakeholder inclusion".



56/70
\textbf{*5.23 Conical example of Spatial Information Visualization Composition (SIVC) in Design for Sustainability Transtions (DfST) ...................................... 137 
}
Information visualization of a segmented cone on white background using black lines and text. This segmented cone represents four areas on a continuum of Systemic (Design) Practice. This figure is based on Kjøde's "Relating Systemic Design Practice to Socio-technical Systems Theory and the MLP". 


57/70
\textbf{*5.24 Cylindrical example of Spatial Information Visualization Composition (SIVC) in Design for Sustainability Transitions (DfST) .................................... 138 
}
Information visualization using black and grey lines and text, and multi-coloured forms. This figure is of a large cylinder containing the segmented cone illustration from Figure 5.23, and highlights the one-dimensional continuum of Systemic Practice in a multi-coloured double-arrow. This continuum is also shown as a four-lobed cycle of DfST Competencies, Toolboxes, Facilitation, and Knowledge, shown in pale yellow, green, red, and blue. This figure is based on Kjøde's "Praxeological framework for DfST relating to systematic transition initiatives". 



58/70
\textbf{*5.25 Kjøde’s DfST  Systematic Combining gigamap visualized as Cylinder Semantic Form network graph wiraph with Query Isomorph i.......................... 139 
}
A figure based on Kjode's "Praxeological framework" overlaid onto an illustration of my Cylinder Semantic Form and the nodes of Sample s, including Query Isomorph i. This figure is shown on a dark grey background.


59/70
\textbf{*5.26 Sphere Semantic Form as LLM Vector Embeddings . . . . . . . . . . . . . . . . . . . . . . . 140 
}
A figure of a Large Language Model (LLM) vector embedding graph as an example of a Sphere Semantic Form. The LLM vector graph is contextualized with white grid lines on a dark grey background. Categorical figures are indicated for term A in teal, term B in magenta, and the distance between their vectors in orange.


60/70
\textbf{*5.27 An Ontological Summary Graph of the Syntopicon’s Great Ideas onto the Tree of Porphyry ……… 143 
}
A figure of the Tree of Porphyry as black lines and text on a white background once more. Additionally, Adler's Syntopical terms overlaid onto the Tree of Porphyry in red text.



61/70
\textbf{*5.28 Torus vs Silos ........................................... 145 
}
A three-dimensional visualization of the Horn Torus Semantic Form as a spatial network graph spatially connecting long white rectangles on the 'ground' of the image and vertical cylinders standing upright on the rectangles. This image represents the way analysis of Semantic forms can connect across ecological place represented as agricultural monocrop by using flat rectangles. To echo the type of extractive capitalist structures that also silo knowledge work, one vertically positioned cylinder is placed at the end of each monocrop row. 

Additionally, three two-dimensional panes are placed around the central figure to represent dimensional reductions.

This nested information visualization illustrates an example of Terroir of Text and Graphs (TTG).



62/70
\textbf{A.1 Example Query Isomorph in two dimensions and three dimensions assembled from directed graphlets…….169 
}
A two-part figure of graph chunk configurations. To the left is a two-dimensional representation on a white background. To the right is a three-dimensional representation on a dark grey background. 



63/70
\textbf{A.2 Groups of imposition and types of imposition .......................... 170 
}
A grid of network graphs by Bertin showing their types and relationships. This illustration is made of black lines, arrows, graphs, and text on a white background.



64/70
\textbf{A.3 Groups of imposition and types of imposition as categories for my Semantic Shapes, Semantic Forms, and three-dimensional gigamaps ................................. 171 
}
This figure shows a composite image building on Bertin's "Groups of imposition and types of imposition". In it, I overlay my Semantic Shapes and Semantic Forms to show their relationships to figures in Bertin's taxonomy. 



65/70
\textbf{A.4 Implantation and imposition of network diagrams . . . . . . . . . . . . . . . . . . . . . . . . 172 }
This figure is a grid of network graphs by Bertin showing their types and relationships. This illustration is made of black lines, arrows, network graphs, and text, on a white background. 


66/70
\textbf{A.5 Implantation and imposition of network diagrams as categories for my Semantic Shapes, Semantic Forms,and three-dimensional gigamaps ............................ 173 }
This figure shows a composite image building on Bertin's "Implantation and imposition of network diagrams". In it I overlay my Semantic Shapes and Semantic Forms to show their relationships to figures in Bertin's taxonomy. 



67/70
\textbf{A.6 Graphical abstract ........................................ 174 }
This figure is an information visualization of variously hued rectangles and arrows on a white background representing the relationships between texts as written in this thesis. 



68/70
\textbf{A.7 Horn of Futures ......................................... 175 }
This figure shows Taylor's Cone of Plausibility warped to show the line of time not as a horizontal line, but as a spiral emerging from the middle of the figure. This figure is illustrated as a two-dimensional spiral,  representing a dimensional reduction of Taylor's Cone of Plausibility. 

This spiralling 'cone' is segmented into repeating phases like the "Diagram of a design process with iterations" found in Sevaldson, 2022b, p. 343. 

To represent Possible Futures, the widest spiralled triangle is represented with a black colour, Probable Futures in dark grey, Probable Futures in light grey, and Preferable Futures is shown in beige. The middle of the Preferable Futures range is labelled with a green line, and the middle of the light grey Probable Futures range is labelled with a red line. 



69/70
\textbf{A.8 Horn Torus point cloud ..................................... 176 }
This figure is a three-dimensional point cloud of blue dots rendered onto the virtual space's 'floor' and 'sides', marked as the X, Y, and Z vertices, labelled as thin black lines and text. 




70/70
\textbf{A.9 The Data Visualisation Catalogue ................................ 177 }
Grid arrangement of six columns and ten rows of information visualization types by Severino Ribecca. Each information visualization type is shown as a blue circle with a white illustration. Each blue circle is paired with a label indicating the type of information visualization shown in black text. 