\chapter{Conclusion}

In this thesis about the space-time of ideas, I offer my methodological prototypes for diachronic visuospatial topic modelling as anticipatory cybernetic design for climate resilience. In the ongoing work leading out from this thesis, I seek to develop multi/inter/trans-disciplinary research as a whole by building on Systemic Design and Machine Learning with computational topology.
\index[terms]{climate resilience}

To integrate various disciplinary perspectives, I set an expansive semantic container for more fulsome representations of complex ideas. Horn Torus Semantic Form Isomorphs were a wondrous, simultaneously divergent and convergent blossoming of ideas at the core of my investigation, but I have the sense that they are only the beginning. Beyond TCA workspace and any individual Semantic Forms, I will also expand my practice of identifying, categorizing, and applying Information Visuospatialization Compositions. This practice is the root and basis of my proposed methods for Surfacing, Synthesizing, Translating, and Producing Knowledge with HATG, HITL CATG, or both. I draw inspiration from the mycelium as a living teaching of transmutation and hope my cybernetic rhizome of isomorphic interbeing/s helps radically transmute our sustainability frameworks as much as it helps us with unlearning unhelpful axiological paradigms.
\index[terms]{torus}
\index[terms]{visuospatialization}
\index[terms]{Semantic Forms}


The sprawling scope of this cybernetic Knowledge Activation rhizome may seem to point away from my relational embodiment. However, my work towards synthesizing across disciplines parallels an urgency in my life to connect with the \textit{buried} histories of my family lineages, so many of which were withheld from me by the violence of armed conflict and oppression. Similarly to my personal mission of revealing inroads for honouring my physical and intellectual ancestors, I hope that the practice of developing Graphical User Interface design for Topological Capta Analysis helps the reader reveal co-benefits between their own intellectual and social lineages.
\index[terms]{Topological Capta Analysis (TCA)}



\section{Call to action: our collective future work}
In the face of the accelerating climate crisis, our ability to harness and activate vast bodies of knowledge to bridge disciplinary gaps has never been more critical. Drawing inspiration from the Horn Torus Semantic Form—a form symbolizing continuous cycles and interconnectedness— I ask: what is the re-origin point that will mobilize us in the face of climate catastrophe? 
%all our (entity) relations

There is a more fundamental driver underlying tools and methodologies: Trust. Abbott et al. emphasize that among our most pressing needs, we ``desperately need transparency and shared sacrifice to reinforce trust and solidarity” \citep[p. 24]{abbott_emergency_2023}. Similarly, Terry Tempest Williams entreats that to ``bear witness to this burning world” we must ``trust one another not to look away” \citep[47:00]{williams_practice_2024}. 
\index[people]{Williams, Terry Tempest}

Ever so delicate and ever so vital, trust becomes our starting point, endpoint, and the re-activating catalyst that propels us forward. We must begin, fail, and begin again with trust in one another, to not look away, to confront the climate crisis directly, and to collaborate across divides. As a network, we can transform vast amounts of information into insightful action for sustainability, reconciliation, and justice.

Perhaps a gift we could receive by facing the climate hyperobject, in all its dizzying scale, in which life is so vulnerable to global warming, is to shift our emphasis to the shared aims of the unified web of life. The only way out is through\footnote{``The only way out is through" is a paraphrase of Robert Frost \citep[p. 66]{frost_north_1917} as an homage to the bison, which can live through extreme heat \citep{world_wildlife_fund_meet_nodate} and have been known to respond to blizzards by facing them \citep{dapcevich_bison_2024}.}, and together.

\index[people]{Drucker, Johanna}
\index[terms]{interbeing}
\index[terms]{cybernetic rhizome}	
\index[terms]{rhizome} 
\index[terms]{TCA Workspace}
\index[terms]{Horn Torus Semantic Form}
\index[terms]{Horn Torus Semantic Form}


%The space-time of ideas: experiments in diachronic visuospatial topic modelling  as anticipatory cybernetic design for climate resilience.
% It is an exploration of diachronic visuospatial topic modelling , of the space-time of ideas.
% The space-time of ideas: anticipatory cybernetic design for climate resilience through diachronic visuospatial topic modelling  and its topological data analysis | HCI | CHASS | DH | ML | AI |

%While I seek to reveal aspects of the sustainable interdisciplinary Knowledge Activation rhizome of information technology,
%However, my thesis work is deeply tied to how I operate with a specific body and a particular history.