\chapter{Conclusion}
While I seek to reveal aspects of the interdisciplinary sustainability-oriented Knowledge Activation rhizome of information technology, I am operating from a specific body with a specific history. In my life, the urgency of synthesis across disciplines operates in parallel to connecting with buried histories from my family lineages which were written out of memory by homogenizing colonial European forces. My hope is that design for Topological Capta Analysis interfaces contributes to revealing both intellectual and social lineages in ways that harmonize but don't homogenize.

The Horn Torus Semantic Form of Isomorphs, that wondrous simultaneously divergent and convergent blossoming of ideas at the core of my investigation, is only the beginning. Beyond TCA workspace and any individual Semantic Forms, my larger pursuit is developing a practice of identifying, categorizing, and applying Information Visuospatialization Compositions for use in Knowledge Production, and the various forms of Knowledge Activation, computationally or otherwise. 

In my endeavour to integrate various disciplinary perspectives, I sought to find which means allow for the most fulsome representation of complex ideas, so as to set as capacious a container as possible. 

As demonstrated through the development of TCA Workspace, this thesis's seven contributions collectively represent better practices in Knowledge Activation by providing innovative methods for surfacing, synthesizing, translating, and creating knowledge with HATG and HITL CATG.

I draw inspiration from the mycelium as a living teaching of transmutation and hope my cybernetic rhizome of isomorphic interbeing/s helps radically transmute our sustainability solutions frameworks as much as it helps us with unlearning axiological paradigms that keep us from making change. 

\section{Call to action: our collective future work}
In the face of the accelerating climate crisis, our ability to harness and activate vast bodies of knowledge to bridge disciplinary gaps has never been more critical. Drawing inspiration from the Horn Torus Semantic Form—a form symbolizing continuous cycles and interconnectedness—we must ask ourselves, what is the re-origin point that will mobilize us in the face of climate catastrophe? 

There is a more profound need here beyond tools and methodologies that drive our efforts: Trust. Abbott et al. emphasize that among our most pressing needs, we “desperately need transparency and shared sacrifice to reinforce trust and solidarity” \citep[p. 24]{abbott_emergency_2023}. Similarly, Terry Tempest Williams entreats that to “bear witness to this burning world” we must “trust one another not to look away” \citep[47:00]{williams_practice_2024}. 

Trust becomes our starting point, our endpoint, and our re-activating catalyst that propels us forward. Trust in one another to not look away, to confront the challenges directly, and to collaborate across disciplines and divides. Together, we can transform vast information into actionable insights to close the divide between knowledge and justice. The climate crisis demands nothing less than our unified, trust-infused actions. The only way out is through\footnote{"The only way out is through" is a paraphrase of Robert Frost \citep[p. 66]{frost_north_1917} as an homage to the bison, which can live through extreme heat \citep{world_wildlife_fund_meet_nodate} and have been known to respond to blizzards by facing them \citep{dapcevich_bison_2024}.}, and together.

\index[people]{Drucker, Johanna}
\index[terms]{interbeing}
\index[terms]{cybernetic rhizome}	
\index[terms]{rhizome} 
\index[terms]{TCA Workspace}
\index[terms]{Horn Torus Semantic Form}
\index[terms]{Horn Torus Semantic Form}